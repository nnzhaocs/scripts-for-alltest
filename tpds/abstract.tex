\IEEEtitleabstractindextext{
	
\begin{abstract}

%\vspace{-5pt}

The rise of containers has led to a broad proliferation of container images.
%
The associated storage performance and capacity requirements place high
pressure on the infrastructure of registries, which store and serve images.
%
%The large volume of Docker images stored in registries and the high file
%redundancy among the images are key obstacles to efficiently scale these
%registries.
%
%\DIM{I don't understand what is the "key obstacles" here?}
%\NZ{Fixed} \Subil{reworded to include file redundancy as obstacle}
%
%The proliferation of container images and the associated storage performance
%and capacity requirements are key obstacles to efficiently scale Docker
%registries.
%
Exploiting the high file redundancy in real-world images
%through deduplication
is a promising approach to drastically reduce the large storage requirements of the
registries.
%
%Though real-world images contain a tremendous amount of duplicate data,
%
%However, modern registries are currently not able to effectively eliminate the duplicates.
%due to the compressed format of the images.

In this paper, we propose \sysname, a new Docker registry architecture, 
which
natively integrates deduplication into the registry.
%
\sysname supports several configurable \emph{deduplication modes}, which provide
different levels of storage efficiency, durability, and performance, as
required for different use cases.
%
To mitigate the negative impact of deduplication on the image download times, 
\sysname introduces a \emph{two-tier storage hierarchy} with a novel
layer prefetch/preconstruct cache algorithm based on user access patterns.
%
Under real workloads, 
for \emph{highest data reduction mode},
\sysname saves up to half of storage space 
compared to the current registry. 
For the \emph{highest performance mode},
\sysname can reduce the \texttt{GET} layer latency up to 13\%
compared to the state-of-the-art. 
%The remaining deduplication modes
%allow for different trade offs in performance and data reduction.
%the latencies within \gap of the original registry.
%
%\VT{We need to state performance impact of the highest data
%reduction mode and deduplication ratio for the highest performance mode.}
\end{abstract}
\begin{IEEEkeywords}
	Computer Society, IEEE, IEEEtran, journal, \LaTeX, paper, template.
\end{IEEEkeywords}}

%}
