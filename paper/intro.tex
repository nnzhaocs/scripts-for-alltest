\section{Introduction}

\vcomment{The text below is work in progress. Do not review yet.}

%Virtualization of computational resources is a
%wider spread technique that providides
%in many application fields, ranging
%$from HPC to XXX, to XXX.
%
%Cloud computing relies on compute, storage, and network
%virtualization to provide efficient, elastic, and inexpensive.
%Virtual machines acted as an enabler for Cloud Computing,

Virtual Machines serve as a corner stone of resources virtualization both in
Cloud and traditional deployments.
%
In the recent years, however, another approach to
virtualization---containers---became especially popular.
%
According to polls, over 87\% of enterprises are at various stages of adopting
containers and by 2020 it will be \$2.5 billion makrket.
%
Container is a groups of processes for which OS kernel
creates an illusion that they run alone on the machine.
%
Linux provides visibility isolation using namespaces
while performance isolation using cgroups.
%
Compared to virtual machines, containers are
light-weight---they use less memory and are much faster to start.


Though much research was done on containerization in general,
storage for containers still remains a raging war of opinions.
XXX: cite login and something esle.
%
A number of research and engineering projects proposed
various innovation. Cite.
%
However, one of the problems with proposing solutions,
configuring, and evaluation is that very little known
what user run in containers.


Containers are especially popular now, in part, thanks
to Docker technology.
%
Docker combines runtime packaging with containerization
using the concept of multi-layered images.
%
Images are stored at the cenralized registry
and are pulled from the registry when need to be run by a client.
%
Our estimate show that Docker Hub stores around 400,000 images
covering almost 2,000,000 layers---counting only public once.
%
Such a massive set of data allows to understand what applications
people run in containers, what they store in images, and other questions.


In this work we perform and exhaustive characterization of Docker registry
contents.
%
We downloaded over 55TB of Docker images and layers and analyzed layer size,
file size, file type, compression ratio and other distribution.
%
What can we learn from registry contents? How can it help?
%
Use cases: 
- Containerize every user application (like git).
- Containerize services.
- Containerize whole OSes - 
%
Number of images is growing and layers. Currently XXX.
%
A number of companies already provide registry as a service.

Why usefull?  Registry design, overall Docker improvement.
Understand what data people store in general
