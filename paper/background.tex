
\section{Background}

%\nancomment{
%TODO:\\
%1. add more detail info about driver? maybe:\\
%2. redunce some trival details
%}\\

Hypervisors based server virtualization technologies (e.g., VMware~\cite{xxx}, Xen~\cite{xxx}, and KVM~\cite{xxx}) have been extensively used by most of cloud platforms such as Amazon EC2, which consists of a virtual machine monitor (VMM) on top of a host operating system that allows dynamically partitioning of a machine and sharing the available physical resources such as CPU, storage, memory and I/O devices to support the concurrent execution of multiple guest operating systems instances within virtual machines (VMs) and provide users with benefits ranging from application isolation through server consolidation to improve disaster recovery and faster server provisioning~\cite{xxx}.
% 
\vcomment{The sentence above is 12 (!) lines long (in pdf)! We need to break it (and other such sentences) in
several sentences.  In a scientific paper, an optimal sentence is of about 3 lines long and almost never longer than 6 lines (I'm talking about 2-column format, as used by FAST).}
%
\vcomment{The paragraph above might be too detailed or out-of-scope of the paper. Let's see how it reads later.}




Nevertheless, hypervisor-based virtualization has a high performance overhead because of execution time overhead caused by executing privileged instructions and memory overhead caused by running multiple VMs.
%
Guest OSs are normally executed at a reduced privilege level~\cite{}.
%
Hypervisor intercepts traps from guest OSs and emulates the trapping instructions, which incurs execution time overheads, specially for applications which relies on I/O operations since I/O interrupt overhead is amplified for nested virtual machine.
%
Memory overhead includes space reserved for the VMs buffer and various virtualization data structures, such as shadow page tables~\cite{}, which increases with the number of virtual CPUs and the configured memory for the guest OSs~\cite{}. 
%
\vcomment{First, I think we do not need to explain so much details of CPU overhead.
%
I think for this paper it is sufficient to list overheads: memory, I/O, CPU and have corresponding citations.
%
(Notice I/O and memory overheads are more prominent in hypervisor virtualization).
%
Second, for memory overhead, we need to mention that hyperviser-based virtualization keeps complete copy of every guest OS in memory (compared to containers).
%
Third, somewhere we need to state that VM start time is long (give estimates) (comared to containers).}





Recent container-based virtualization (such as Linux Containers(LXC)~\cite{xxx}, OpenVZ~\cite{bibid}, and Docker~\cite{}) emerges as a lightweigh virtualization, which promises a near-native performance.
%
As opposed to virtual machines (VMs), container based virtualization works at operating system level and do not emulate another operating system.
%
In this case, all the virtual instances share a single operating system kernel, which significantly reduces theoverhead imposed through VMs.
%
Linux containers offer isolation (of PIDs, IPCs, mount pints, and network) through (PID and network) \textit{namespaces} while manages resource and controls processes via \textit{cgroups}~\cite{}.
% 
Docker container is a new popular container-based virtualization technology that extends LXC with higher level APIs and additional functionality.
%
\vcomment{Here, we briefly need to state what are added Docker functionalities and benefits,
rather than going into details about CoW. Basically, we need to say that Docker combines
sofware packaging with containerization which allows bla-bla-bla.}
%
Docker incorporates copy-on-write union filesystems (UnionFS) to avoid duplication and enable versioning.
%
It couples the above two components with a number of features, like portability, re-use, and reproducibility.




%Docker is an open source project that automates the deployment of applications inside Linux Containers, and provides the capability to package an application with its runtime dependencies into a container. It provides a Docker CLI command line tool for the lifecycle management of image-based containers. Linux containers enable rapid application deployment, simpler testing, maintenance, and troubleshooting while improving security. 

%, and  that make it developer-centric (and therefore distinct from traditional virtual machines that attempt to hew as much as possible to the metaphor of machine): like deployment 

%AuFS (Advanced Multi-Layered Unification Filesystem) as a filesystem for containers. AuFS is a layered filesystem that can transparently overlay one or more existing filesystems. When a process needs to modify a file, AuFS creates a copy of that file. AuFS is capable of merging multiple layers into a single representation of a filesystem. This process is called copy-on-write. 
 
%Docker using a high-level API that provides a lightweight virtualization solution to run processes in isolation. Docker is developed in the Go language and utilizes LXC, cgroups, and the Linux kernel itself. Since it’s based on LXC, a Docker container does not include a separate operating system; instead it relies on the operating system’s own functionality as provided by the underlying infrastructure. So Docker acts as a portable container engine, packaging the application and all its dependencies in a virtual container that can run on any Linux server.
 
% packaging and delivery technology, combining lightweight application isolation with the flexibility of image-based deployment methods. 
 
%an extension of LXC’s capabilities that provides higher level APIs and functionality as a portable container engine. It aims to improve reproducibility of applications by enabling bundling of container contents into a single object that can be deployed across machines.

%Docker is an open source project that automates the deployment of applications inside Linux Containers, and provides the capability to package an application with its runtime dependencies into a container. It provides a Docker CLI command line tool for the lifecycle management of image-based containers. Linux containers enable rapid application deployment, simpler testing, maintenance, and troubleshooting while improving security. 

% Linux Containers LXC, a user-space control package for Linux Containers, constitute the core of Docker. 
 
 %Docker harnesses some powerful kernel-level technology and puts it at our fingertips. The concept of a container in virtualization has been around for several years, but by providing a simple tool set and a unified API for managing some kernel-level technologies, such as LXCs (LinuX Containers), cgroups and a copy-on-write filesystem, Docker has created a tool that is greater than the sum of its parts. The result is a potential game-changer for DevOps, system administrators and developers.
 
 %Docker provides tools to make creating and working with containers as easy as possible. Containers sandbox processes from each other. For now, you can think of a container as a lightweight equivalent of a virtual machine.




 
\subsection{Docker}
\vcomment{Nannan, we need to move ALL subsections to separate *.tex files.}

%Docker is an open platform for developers and system administrators to build, ship, and run distributed applications using Docker Engine, a portable, lightweight runtime and packaging tool, and Docker Hub, a cloud service for sharing applications and automating workflows. The main advantage is that,
%Docker allows packaging an application with its dependencies into a standardized, self-contained unit (a so-called container), which can be used for software development and to run the application on any system.
%Containers are an abstraction at the app layer that packages code and dependencies together. 
 
\nancomment{
	1. TODO: add a docker architecture pic}\\





%how container isolate.
Docker containers create a wrapped, controlled environment on the host machine in which applications can be run in isolated manner via two main Linux kernel features -- kernel namespaces which are used to split the view that processes have of the system and control groups (cgroups) that restricts the resource usage of a process or group of processes.
%
Currently, Linux kernel provides six different namespaces, PID, IPC, NET, MNT, UTS, and USER for process IDs, IPC requests, networking, file-system mount points, host names, and user IDs~\cite{xxx}~\cite{xxx}.
%
Controlled resources include CPU shares, RAM, network bandwidth, and disk I/O.
%
\vcomment{This paragraph seems to belong to the Section above (on containerization) rather than to Docker section}.





%how container deameon. 
As Figure~\ref{xxx} shows, Docker ecosystem includes various components, i.e., Docker daemon, Docker container, Docker image, and Docker Hub.
%
Docker Deamon, known as Docker engine, can create images from Dockerfiles (through docker build), launch containers (through docker run), and fetch non-local images from Docker registry as well as publish new images to Docker registry such as Docker Hub (through docker pull or docker push).
%
It controls isolation levels of containers including cgroups, namespaces, capabilities restrictions, and SELinux/Apparmorprofiles, monitor them to trigger actions such as restart, and spawn shells into running containers for administration purposes.
%
\vcomment{@Nannan, I think I see the structure  you propose for this section: describe Docker components
and their relationship using a figure. I think it is fine. But we desperately need the figure to write/revise this text consciously.}



 
 
\subsection{Docker storage}

\subsubsection{Docker image and layers}

% what is image containning
Docker image is an immutable and ``executable'' file that is essentially a snapshot of a container.
%
\vcomment{I don't think the statement above is accurate. Image is not a ``file'', right?}
%
Docker images packages an application only with its runtime dependencies, such as binaries, and libraries to maintain lightweight characteristics.
%
Images are read-only copies of file system data.
%
All modifications to the container that add new or modify existing data are stored in a top writable layer which was initialized when a new container is created.
%
The writable layer is deleted when the container is deleted.
%
Since each container has its own writable layer, and all changes are stored in this layer, multiple containers can share access to the same underlying image and yet have their own data state~\cite{}.



 
% how to build, relation to container

%There are two methods to build an image: The first method is called interactive building which is carried out by starting a base image as a container (via `Docker run'), running commands to install the desired software on the running container, then committing the changes creating a new image on the local Docker repository~\cite{}. The second method is called building from a Dockerfile which is carried out by creating a Dockerfile. A Dockefile is a script file that contains all the commands one would normally execute manually in order to build an image. Dockerfile starts by loading a base image, followed by the list of Docker formatted commands to install the desired software. The image is built via `docker build'. 

Docker images are composed of a set of individual layers along with metadata in the JavaScript Object Notation (JSON) format called manifest.
%
Each layer contains the data modifications relative to the previous layer, starting from a base layer/image (typically, a lightweight Linux distribution).
%
\vcomment{Need to exlain at which granularity are the changes maintained.}
%
Each layer has a parent, except for the base layers/images, which are the roots of the trees.
%
This structure avoid Docker pulling redundant layers.
%
\vcomment{Explain the last sentence in a bit more details.
%
Right now it is not clear who pulls what and why layers reduce pulling.
%
In fact, we use ``pull'' here for the very first time.
%
Maybe we need to explain it earlier, in Docker section.}




%
\vcomment{I think it would be good to have here a simple figure that depicts relationship between image, image manifest, and layers, and how sharing of layers happens.}
%
Image manifests describe the various constituents of a docker image, such as platform which the image runs on and the configuration the runtime uses to set up the container.
%
Moreover, it contains a list of file system layers'digests contained in this image.
%
Manifest has two versions: schema version 1 and schema version 2.
%
Schema version 2 uses a config field references a JSON file that contains the configuration information.
%
Schema version 2 allows multi-architecture images, through a “fat manifest” which references image manifests for platform-specific versions of an image. 





%uses a config field references a configuration object for 
% container, by digest. This configuration item is a JSON blob that the runtime uses to set up the container. 

%The second is to move the Docker engine towards content-addressable images, by supporting an image model where the image’s configuration can be hashed to generate an ID for the image. 

%The config field references a configuration object for a container, by digest. This configuration item is a JSON blob that the runtime uses to set up the container.

%This new schema uses a tweaked version of this configuration to allow image content-addressability on the daemon side.

%This second schema version has two primary goals. The first 

%\begin{enumerate}
	%\item Layers.
	%\item Manifest.
	%\item Config file.
%\end{enumerate}





\subsubsection{Docker client storage}
% what is .., benefit

\vcomment{need to make sure that we talk about Docker client storage, not registry or anything else.}
%
The data management of container is superintend either by Docker storage drivers (e.g. AUFS, OverlayFS, Btrfs, etc) or by docker data volumes.
%
Storage driver manage container storage and mounting.
%
It provides a view of layer data via a mount point that the container uses as its root file system.
%
It usually implements copy-on-write (CoW) technique which means it does not update the data.
%
Instead, it creates a new copy of that part of data which is stored somewhere else on the disk keeping the old part as it is.
%
\vcomment{In previous three sentences "it" is used 3 times :)
%
It's hard to follow the logic.
%
In general, in formal text, where possible pronouncs (it, they, etc.) should be avoided and actual nouns used instead.}
%
Docker volume is mechanism to automatically provide data persistence for containers.
%
A volume is a directory or a file that can be mounted directly inside the container.
%
The I/O operations through this mount path are independent of storage driver and executed directly on the host. 



Docker provides a variety of pluggable storage drivers that are based on Linux filesystem or volume manager, including advanced multi-layered unification filesystem (aufs), B-tree file system (Btrfs), device mapper, or OverlayFS.
%
All of them use Copy-on-write(CoW) strategy to maximize storage efficiency.
%
AUFS is the default storage driver for Docker container.
%
\vcomment{It is not default in RHEL/CentOS, where device-mapper is the default.
%
I think we can say
that Aufs and DM are most commonly used in modern distributions.
%
Notice, you also then need to adjust/remove/rewrite the sentences below.}
%
It is a union filesystem which layers multiple directories on a single Linux
host and presents them as a single directory (i.e., branch/layer).
%
The unification process is referred to a union mount.
%
AUFS can efficiently share images between multiple running containers, enabling
fast container start times and minimal use of disk space~\cite{xxx}. 

%OverlayFS (overlayer and overlay2) is a modern union filesystem that is similar to AUFS, but faster.

%However, CoW requires more space because it stores the old copies of data as well.

 %OverlayFS layers two directories on a single Linux host and presents them as a single directory. These directories are called layers and the unification process is referred to as a union mount. OverlayFS refers to the lower directory as lowerdir and the upper directory a upperdir. The unified view is exposed through its own directory called merged. While the overlay driver only works with a single lower OverlayFS layer and hence requires hard links for implementation of multi-layered images, the overlay2 driver natively supports up to 128 lower OverlayFS layers. 

%However, AUFS storage driver can introduce significant latencies into container write performance. This is because the first time a container writes to any file, the file has to be located and copied into the containers top writable layer. These latencies increase and are compounded when these files exist below many image layers and the files themselves are large.

% This capability provides better performance for layer-related Docker commands such as docker build and docker commit, and consumes fewer inodes on the backing filesystem. 

%Btrfs is a next generation copy-on-write filesystem that supports many advanced storage technologies that make it a good fit for Docker. Btrfs is included in the mainline Linux kernel.  Among these features are block-level operations, thin provisioning, copy-on-write snapshots, and ease of administration. You can easily combine multiple physical block devices into a single Btrfs filesystem. btrfs storage driver stores every image layer and container in its own Btrfs subvolume or snapshot. The base layer of an image is stored as a subvolume whereas child image layers and containers are stored as snapshots, which only contain the differences introduced in that layer. The container’s writable layer is a Btrfs snapshot of the final image layer, with the differences introduced by the running container. These differences are stored at the block level.

%Device Mapper is a kernel-based framework that underpins many advanced volume management technologies on Linux. Docker’s devicemapper storage driver leverages the thin provisioning and snapshotting capabilities of this framework for image and container management. The devicemapper driver uses block devices dedicated to Docker and operates at the block level, rather than the file level. These devices can be extended by adding physical storage to your Docker host, and they perform better than using a filesystem at the level of the operating system. The devicemapper storage driver uses dedicated block devices rather than formatted filesystems, and operates on files at the block level for maximum performance during copy-on-write (CoW) operations. Another feature of devicemapper is its use of snapshots (also sometimes called thin devices or virtual devices), which store the differences introduced in each layer as very small, lightweight thin pools. Layers which are shared in common between containers are only stored on disk once, unless they are writable. For instance, if you have 10 different images which are all based on alpine, the alpine image and all its parent images are only stored once each on disk. Snapshots are an implenentation of a copy-on-write (CoW) strategy. This means that a given file or directory is only copied to the container’s writable layer when it is modified or deleted by that container. Because devicemapper operates at the block level, multiple blocks in a writable layer can be modified simultaneously. Snapshots can be backed up using standard OS-level backup utilities. Just make a copy of /var/lib/docker/devicemapper/

%Aufs is a fast reliable unification file system with some new features like writable branch balancing. Btrfs(B-tree file system) is a modern CoW file system which implements many advanced features for fault tolerance, repairand easy administration. Overlayfs is another modern union file system which has a simpler design and is potentially faster than Aufs.

%Device mapper is a Linux kernel component; it provides a mechanism for mapping physical block devices onto virtual block devices. These mapped devices can be used as logical volumes. Device mapper provides a generic way for creating such mappings. Device mapper maintains a table which defines device mappings. The table specifies how to map each range of logical sectors of the device.

% BTRFS is a Linux file system which has a poten-%tial of replacing the current Linux default file system,

%The start value for the first line is always zero. For other
%lines, start + length of the previous line should be equal
%to start value of the current line. Device mapper sizes are
%always specified in 512 bytes sectors. There are different
%types of mapping targets linear, striped, mirror, snapshot,
%snapshot-origin, etc.

%BTRFS (also knows as ”butter FS”) is basically a copy on write file system. Copy-on-write(CoW) means it does not update the data ever.[8] Instead, it creates a new copy of that part of data which is stored somewhere else on the disk keeping the old part as it is. Anyone with a decent file systems knowledge would understand that CoW requires more space because it stores the old copies of data as well. Also, it has a problem of fragmentation. Then how can a CoW file system be used as a default Linux file system? Wouldn’t that reduce the performance? No need to mention the storage space problem. Let’s dive into BTRFS to understand why it has become so popular.

%The primary design goal of BTRFS was to develop a generic file system which can perform well with any use cases and workload. Most of the file systems perform well for a particular specific file system benchmark, and the performance is no that great for other scenarios. Apart from this BTRFS also supports snapshots, cloning,and RAID (Level 0, 1, 10, 5, 6). This is more than any-one has bargained for from a file system before. One canunderstand the design complexity because Linux file sys-tems are deployed on all kinds of devices from computersand smart phones to small embedded devices.[10]The BTRFS layout is represented with B-trees, morelike a forest of B-trees. These are copy-on-write friendlyB-trees. As CoW file systems require a little more diskspace, in general, BTRFS has a very sophisticated mech-anism for space reclamation. It has a garbage collectorwhich makes use of reference counting to reclaims un-used disk space. For data integrity part, BTRFS usescheck sums.

%The commonly used storage drivers are 

% different types

%\begin{enumerate}
%	\item aufs.
%	\item zfs.
%	\item overlay/overlay2.
%	\item devicemapper.
%\end{enumerate}

%\subsubsection{Storage driver}

% The storage driver controls how images and containers are stored and managed on Docker host using a pluggable architecture. 

%\begin{enumerate}
%	\item aufs.
%	\item zfs.
%	\item overlay/overlay2.
%	\item devicemapper.
%\end{enumerate}

%\subsubsection{}




\subsection{Docker Registry}

The Docker Registry, known as Docker Hub, allows users push their Docker images to it and pulling Docker images from it.
%
\vcomment{Docker Hub is just one instance of registry installed.
%
I.e., Docker Registy != Docker Hub. Rephrase above.
%
}
%
The repositories in Docker Hub is either public or private.
%
All the non-repositories'names are in ``$\langle namespace\rangle/\langle repository name \rangle$" format, where~\textit{namespace} is the user name.
%
The official repositories'names only contain ``repository name".
%
\vcomment{We need to exlain what is a repository.}
%
\vcomment{Somewhere we need to say that clients talk to registries using REST API.}



Before Docker v1.10, every image and layer were assigned a randomly generated UUID.
%
\vcomment{how long was UUID?}
%
Starting from v1.10 Docker registry implemented a content addressable method using an ID (digest), based on a secure hash of the image and layer data.
%
This new method avoids ID collisions and guarantee data integrity after pull, push, load, or save.
%
\vcomment{Hashes theoretically can still collide.}
%
\vcomment{I think we need to say which hash is used}
%
It also brings better sharing of layers by allowing many images to freely share their layers even if they didn't come from the same build.
%
Addressing images by their content also lets Docker engine more easily detect if something has already been downloaded during a pulling.
%
\vcomment{Are the last two sentences state the same or not?}



 

%Content addressability is the foundation for the new distribution features. The image pull and push code has been reworked to use a download/upload manager concept that makes push and pull much more stable and mitigate any parallel request issues. The download manager also brings retries on failed downloads and better prioritization for concurrent downloads.

%We are also introducing a new manifest format that is built on top of the content addressable base. It directly references the content addressable image configuration and layer checksums. The new manifest format also makes it possible for a manifest list to be used for targeting multiple architectures/platforms (...more to come on that later). Moving to the new manifest format will be completely transparent. 

%Because we have separated images and layers, you don't have to pull the configurations for every image that was part of the original build chain. We also don't need to create layers for the build instructions that didn't modify the filesystem.

%The new method gives users more security, provides a built-in way to 

%The Registry is a stateless, highly scalable server side application that stores and lets you distribute Docker images.

%\begin{enumerate}
%	\item Registry versions.
%	\item Content addressability.
%	\item Pulling images/pushing images.
%\end{enumerate}

%\subsection{Current public registries}
