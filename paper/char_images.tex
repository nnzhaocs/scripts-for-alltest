\subsection{Images}
\label{sec:images}

\paragraph{Image size}%{Layer and image size}
%\nancomment{cache parameters, bigger image/layer overhead for runtime}
\begin{figure}[!t]
	\subfigure[CDF of layer size]{\label{fig_layer_size}
		\includegraphics[width=0.4\textwidth]{graphs/layer-size-cdf.pdf}
	}
	\centering
	\subfigure[CDF of image size]{\label{fig_image_size}
		\includegraphics[width=0.4\textwidth]{graphs/image-size-cdf.pdf}%
	}

	\caption{Image/layer size distribution}
	\label{fig:image-layer-size}
\end{figure}
\nancomment{add pdf if can}
Recall that container image consists of multiple layers and each of these layer
is stored in Docker registry. In figure~\ref{fig_layer_size} we show the compressed/uncompressed layer
size distribution. We find that around 50\% of the layers are 1~MB in size and
90~\% of layers are less than 100~MB. \emph{This finding is particularly useful for
the Docker registry designers as it shows that selective or popular layers can
be cached in memory at the registry side.}

We see that 90\% of the layers are smaller than 177~MB in uncompressed format
and smaller than 63~MB in compressed format.
%Interestingly, about half of the layers are smaller than 4~MB, independent of
%the format. 
\nancomment{need check num}
That means that the registry stores a large number of small layers
which can benefit from compression.

To analyze the frequencies, we zoom into the 0--128~MB range (see
Figure~\ref{fig_hist_layer_size}).
More than 1 million and 800,000 layers are smaller than 5~MB in compressed and
uncompressed format, respectively. Beyond that, the frequency drops rapidly and
we only see around 100,000 layers between 5~MB and 15~MB.

%\acomment{Where is layers analysis when you say similarly? I wrote above
%para.. see for consistency.}
%Conatiner image size determines the runtime of container if it is not already
%cached locally. 
Similar to layers, we measured compressed and uncompressed image size. Figure~\ref{fig_image_size} show the image size distributions at a coarse GB
resolution. 
%and a finer resolution only covering images smaller than 1.5 GB.
%\acomment{Could not locate the figures you are talking about. I see two figures
%layers size and image size.}

We find that 90\% of the images have an uncompressed size less than
1.3 GB and compressed size of 0.4~GB.
%while compressed images are less than 0.48 GB. 
In the median, this decreases to 94MB and 17 MB, respectively.  The largest
uncompressed image is 498 GB which is a Ubuntu-based image.  

\emph{Figure~\ref{fig_image_size} shows that the majority of uncompressed images in
Docker Hub are small which aligns with the Docker philosophy to package
software and distribute software in containers but include only its necessary
dependencies.}

\paragraph{Compression ratio}
\begin{figure}
	\centering
	\includegraphics[width=0.4\textwidth]{graphs/compress-ratio-cdf.pdf}
	\caption{CDF of compression ratio.
	}
	\label{fig:compress-ratio}
\end{figure}

\begin{figure}[!t]
	\centering
	\subfigure[Image compression ratio]{\label{fig_hist_image}
		\includegraphics[width=0.215\textwidth]{graphs/image-compress-ratio-pdf.pdf}%
	}
	\subfigure[Layer compression ratio]{\label{fig_hist_layer}
		\includegraphics[width=0.22\textwidth]{graphs/layer-compress-ratio-pdf.pdf}
	}
	\caption{Histogram of image and layer compression ratio.}
	\label{fig:reference-cnt}
\end{figure}

Figure~\ref{fig:compress-ratio} shows the compression ratio of the container
images.    %\nancomment{there is redundancy and avg compression ratio}}

However, looking at Figure~\ref{fig_hist_image} and Figure~\ref{fig_hist_layer}
we find that there are layers for which the compression ratio is zero. Doing
compression and uncompression of such layers incur additional overhead at
docker engine side.  For such layers, we suggest not to do compression.
Docker engine can use a hybrid approach to compress
only those layers which yield better compression ratio. 
\emph{This way users pulling such layers do not need to uncompress
	and will experience reduction in container startup time.}

\emph{High compression
	ratio shows the potential of compression while distributing the container
	images.}


%\acomment{Found following para in directory count subsection..}
To further study the sizes and the impact of compression, we calculate the
compression ratios (see Figure~\ref{}).
%
%The ALS-to-CLS ratio is generally greater than the FLS-to-CLS ratio because
%small files in layers get larger when combined in a tar archive.
%
90\% of layers have a  compression ratio less than 4 and the median compression
ratio is 2.6. The largest compression ratio is 1026.
%
Half of the layers have a compression ratio around 3.
%
%
%The maximum FLS-to-CLS is 512,930 and maximum ALS-to-CLS is 1026.
%
Looking at the histogram (see Figure~\ref{}), we see
that around 600,000 layers have a compression ratio of between 2 and 3 while
more than 300,000 between 1 and 2.
%
%Two peaks in the graph correspond to 587,000 layers that have the FLS-to-CLS
%ratio of 3 and 331,000 layers that have the ALS-to-CLS ratio of 3.

Our size analysis reveals an interesting trade-off. Compression is
computationally expensive and is one of the major sources of latency when
pulling an image from Docker Hub~\cite{slacker}.  As the majority of layers is
small and has low compression ratios, it can be beneficial to store small
layers uncompressed in the registry to reduce pull latencies.

\paragraph{Layer count per image}
%\nancomment{more layer more matedata? overhead for union fs}

\begin{figure}[!t]
	\centering
	\subfigure[CDF of layer count]{\label{fig:layer_count}
		\includegraphics[width=0.215\textwidth]{graphs/image-layer-cnt}%
	}
	\subfigure[Histogram of layer count]{\label{fig:hist_layer_count}
		\includegraphics[width=0.21\textwidth]{graphs/image-layer-cnt-pdf.pdf}
	}
	\caption{CDF and histogram of Layer count per image.}
	\label{fig:image-size}
\end{figure}

As discussed in~\ref{sec-image-layers}, images consist of a set of layers.
It is important to understand the layer count of the images as previous
work found that the number of layers can impact the performance of
I/O operations~\cite{slacker}. Therefore, we count the number of layers
per image and plot the CDF (see Figure~\ref{fig:layer_count})
and layer count frequencies (see Figure~\ref{fig:hist_layer_count})for all
Docker Hub images.

The results show that 90\% of the images have less than 18 layers while
half of the images have less than 8 layers. 8 layers is also the most
frequent layer value with 51,300 images consisting of exactly 8 layers.
The maximum layer count is 120 in the \textit{cfgarden/120-layer-image}.
We also find that there are 7,060 images which only consist of a single layer.

\emph{As a rule of thumb, less number of layers is better for the union file
	system as it then needs to handle less metadata. However, as a tradeoff,
	reducing the number of layers significantly effects the data sharing among images.
	%As a result, we see less number of layers shared among different images.
	For an example, if a image consists of only single layer its data can not be shared
	with other images.}
%\begin{figure}[!t]
	\centering
	\subfigure[CDF of layer by layer count]{\label{fig_repeate_layer}
		\includegraphics[width=0.23\textwidth]{graphs/repeate_layer.pdf}
	}
	\subfigure[Histogram of images by layer count in images]{\label{fig_hist_repeate_layer}
		\includegraphics[width=0.223\textwidth]{graphs/hist_repeate_layer.pdf}
	}
	\caption{Compression rate distribution}
	\label{fig-repeat-layer-cnt}
\end{figure}

%\paragraph{Repeat layer count distribution}
\paragraph{Layer reference count}
%\nancomment{should create more shared layers to remove duplicates}

\begin{figure}
	\centering
	\includegraphics[width=0.21\textwidth]{graphs/shared-cnt-cdf.pdf}
	\caption{CDF of layer reference count.
	}
	\label{fig:ref_count}
\end{figure}

An interesting question is what is the sharing rate of layers across images.
We analyze all image manifests and count for each layer, how many times it is
referenced by an image. Figure~\ref{fig:ref_count} shows that around 90\% of
layers are only reference by a single image while 95\% are reference by not
more than 2 images. Similarly, 99\% of layers are shared among less than 25
images. 

Figure~\ref{} shows the absolute values, revealing that
almost 1.5 million images are only referenced once.  \acomment{Figure is
	missing} While there is again a large spectrum of reference counts, the maximum
is 33,428, the vast majority of layers is not shared. This hints that the
layer-based approach to improve storage efficiency is barely utilized and there
is room for improvement in how to construct more sharable layers.

\emph{These findings reveal that layer level CAS has not been very successful
	and most of the layers that exists in Docker Hub are not shared among images.
	Hence, there is a dire need of a better redundancy management.}

\paragraph{Directory count and File count}
%\nancomment{how union fs handle so large dirs}
\begin{figure}
	\centering
	\includegraphics[width=0.4\textwidth]{graphs/dir-cnt-cdf.pdf}
	\caption{CDF of directory count per image/layer.
	}
	\label{fig:reference-cnt}
\end{figure}

\begin{figure}[!t]
	\centering
	\subfigure[Histogram of directory count per image]{\label{fig_reference_cnt_cdf}
		\includegraphics[width=0.2\textwidth]{graphs/image-dir-cnt-pdf.pdf}%
	}
	\subfigure[Histogram of directory count per layer]{\label{fig_reference_cnt_pdf}
		\includegraphics[width=0.22\textwidth]{graphs/layer-dir-cnt.pdf}
	}
	\caption{Histogram of directory count distribution}
	\label{fig:reference-cnt}
\end{figure}

Next, we look at the directory (Figure~\ref{fig-dir}) and file count
(Figure~\ref{fig-file}) in images to determine if deploying images requires
handling of large amounts of metadata. Looking at directories, we see that 90\%
of images have less than 7,344 directories while the median is at 296. For
files, 90\% of images have less than 64,780 files with a median of 1,090.

This is consistent with our analysis of layer-based file and directory counts
and the number of layers per image. Again, we conclude that most images do not
require an extensive amount of metadata when being deployed as file and
directory counts are low except for few outliers.

%\begin{figure}[!t]
	\centering
	\subfigure[CDF of compression ratio]{\label{fig_cdf_compression_ratio}
		\includegraphics[width=0.23\textwidth]{graphs/cdf_compression_ratio.pdf}
	}
	\subfigure[Histogram of comp. ratios]{\label{fig_his_compression_ratio}
		\includegraphics[width=0.223\textwidth]{graphs/his_compression_ratio.pdf}
	}
	\caption{Layer compression ratio distribution
	\vcomment{Different colors are used in figure (a) and (b) FLS/CLS}
	}
	\label{fig-compression-ratio}
\end{figure}


%\paragraph{Layer compression ratios}