\subsection{Layers}
\label{sec:layers}

In this section we charaterize layer sizes; file and directory counts in
layers; layer compression ratios; and layer depths in images.

\begin{figure}[!t]
	\centering
	\subfigure[CDF of layer sizes]{\label{fig_layer_size_cdf}
		\includegraphics[width=0.23\textwidth]{graphs/layer_size_mb.pdf}
	}
	\subfigure[Histogram of layer sizes]{\label{fig_hist_layer_size}
		\includegraphics[width=0.223\textwidth]{graphs/hist_layer_size.pdf}
	}
	\caption{Layer size distribution
	\vcomment{Let's use CLS, ALS, and FLS abreviations}.
	\vcomment{CLS size should go first}.
	\vcomment{We need to use different types of lines (solid, dotted, etc.)
		or markers (round, triangular)}.
	\vcomment{In figure B it is not clear to which bar group corresponds
		  to which layer size. I suggest to try to rotate the graph
		  by 90 grads to fit all layer size labels.}
	}
	\label{fig-layer-size}
\end{figure}

%%%%%%%%%%%%%%%%%%%%%%%%%%%%%%%%%%%%%%%%%%%%%%%%%%%%%%%%%%%%%%%%%%%%%

\paragraph{Layer size distribution.}
%
We characterize layer sizes using three different metrics:
%
1)~Compressed Layer Size (CLS)---the format a layer is stored in the registry or
transferred to a client;
%
2)~Archived Layer Size (ALS)---layer in decompressed but archived format;
%
and 3)~the sum of containing file sizes (FLS).
%
Figure~\ref{fig_layer_size_cdf} shows the CDF of the three metrics.

The ALS and FLS curves are, expectedly, close to each other (within 5\% for
any given layer size) while compressed layers are typically smaller.
%j
90\% of the layers are smaller than 177~MB and 63~MB in uncompressed and
compressed formats, respectively.
%
Half of the layers are smaller than 4~MB no matter in which format.
%
As we described earlier in \S~\ref{sec:methodology}, we only
analyzed layers smaller than 2~GB in compressed format (this
excluded only XXX\% of the layers).
%
\vcomment{need to adjust Methodology to reflect this.}
%
The largest analyzed uncompresed layer
was of XXX~GB size.
%
Figure~\ref{fig_hist_layer_size} zooms into 0--128~MB range.
%
1,080, 985, and 820 thousands of layers are smaller than 5~MB
in compressed, archived, and unpacked formats, respectively.
%
%90\% of layer can be compressed with less than 63 MB and 77\% of images are
%less than 63 MB even without compression. 

%%%%%%%%%%%%%%%%%%%%%%%%%%%%%%%%%%%%%%%%%%%%%%%%%%%%%%%%%%%%%%%%%%%%%

\begin{figure}[!t]
	\centering
	\subfigure[CDF of compression ratio]{\label{fig_cdf_compression_ratio}
		\includegraphics[width=0.23\textwidth]{graphs/cdf_compression_ratio.pdf}
	}
	\subfigure[Histogram of comp. ratios]{\label{fig_his_compression_ratio}
		\includegraphics[width=0.223\textwidth]{graphs/his_compression_ratio.pdf}
	}
	\caption{Layer compression ratio distribution}
	\label{fig-compression-ratio}
\end{figure}

\paragraph{Layer compression ratio distribution.}

We calculated two compression ratios: ALS-to-CLS and FLS-to-CLS.
Figure~\ref{fig_cdf_compression_ratio} shows the CDF of the compression ratios.
%
ALS-to-CLS ratio greater than the FSL-to-CLS ratio.
%
90\% of images have a archival-to-compression ratio less than ~4 while 90\% of
images have a sum of file-to-compression ratio less than 30.
%
Half of the images have a compression ratio (both archival-to-compression and
sum of files-to-compression) around 3.
%
The maximum compression ratio are 512,930 and 1026 for the ratio of sum to file
size to compression size and the ratio of archival to compression size
respectively.
%
Figure~\ref{fig_his_compression_ratio} shows a histogram of layer by
compression ratio.
%
587,000 images have a ratio of sum of file size to compression size of 3 and
331,000 images have a ratio of archival size to compression size of 3, which
are two peaks shown in the graph.

Figure~\ref{fig-compression-ratio} suggests that layers have a great potential
for compression to save space.

%%%%%%%%%%%%%%%%%%%%%%%%%%%%%%%%%%%%%%%%%%%%%%%%%%%%%%%%%%%%%%%%%%%%%

\begin{figure}
	\centering
	\begin{minipage}{0.23\textwidth}
		\centering
		\includegraphics[width=1\textwidth]{graphs/file_cnt.pdf}
		\caption{File count distr.}
		\label{fig_file_cnt}
	\end{minipage}
	\begin{minipage}{0.23\textwidth}
		\centering
		\includegraphics[width=1\textwidth]{graphs/dir_cnt.pdf}
		\caption{Dir. count distr.
		\vcomment{Let's make thise figure subfigures.}
		}
		\label{fig_dir_cnt}
	\end{minipage}%
\end{figure}
%

\paragraph{File and directory count distributions.}

Figure~\ref{fig_file_cnt} and Figure~\ref{fig_dir_cnt} show the CDFs of file
and directory counts for layers.
%
90\% of layers contain less than 7,410 files, while half the of layers have
less than 30 files.
%
XXX\% of the layers have only one file while the largest layer contains 826,196
files.
%
The average is 2,200.
%
The maximum number of directories in a layer is 111,940, the minimum is 1, and
the average is 273.
%
90\% of the layers have less than 826 directories and half of the layer store
less than 11 directories.
%

%%%%%%%%%%%%%%%%%%%%%%%%%%%%%%%%%%%%%%%%%%%%%%%%%%%%%%%%%%%%%%%%%%%%%

\paragraph{Layer depth distribution}

After extracting and unpacking gzip compressed layer archival files, we
calculated the layer directory depth (i.e., the maximum directory depth).
%
Figure~\ref{fig_layer_depth} shows the cumulative layer probability by layer
directory depth.
%
Around 90\% of layers' directory depth is less than 10. 50\% of layers'
directory depth is less than 4. 

Figure~\ref{fig_hist_layer_depth} shows the histogram of layers by layer
directory depth.
%
About 313,000 layers' layer directory depth is 3, which is the peak value in
the figure.
%
The maximum repeat count is 444 while the median is 4. The average is ~5.

\begin{figure}[!t]
	\centering
	\subfigure[CDF of layers by layer directory depth]{\label{fig_layer_depth}
		\includegraphics[width=0.23\textwidth]{graphs/layer_depth.pdf}
	}
	\subfigure[Histogram of layers by layer directory depth]{\label{fig_hist_layer_depth}
		\includegraphics[width=0.22\textwidth]{graphs/hist_layer_depth.pdf}
	}
	\caption{Layer directory depth distribution}
	\label{fig-layer-dir}
\end{figure}
