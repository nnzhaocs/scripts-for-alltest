\begin{abstract}

Docker containers have become a prominent solution for supporting modern
enterprise applications due to the highly desirable features of isolation, low
%
%Containers are created from images that preserve software dependencies,
%environment configuration, and other parameters that affect the application's
%runtime.
%
Containers are created from images which are shared between users via a Docker
registry.
%
%Docker allows sharing of such images between users via a Docker registry.
%
The amount of data Docker registries store is massive; for instance, Docker
Hub---a popular public registry---stores at least a half million public images.
%
%As the amount of images stored in public and private Docker registries
%increases it becomes important to study images' characteristics.
%
Investigating the storage-centric properties of containerized applications can
reveal useful insights that can enable new optimizations, higher performance,
and identification of bottlenecks.
%
%The massive Docker Hub dataset offers a unique opportunity
%for such an endeavor.
%
In this paper, we perform the first in-depth analysis of a large scale Docker
%
%Our goal is to collect statistics from a large amount of Docker images and
%perform a large-scale characterization of Docker images.
%
We download 51TB worth of Docker Hub images and found only ~3\% (5TB) of its
containing files are unique files while others are redundant copies, which means
that current layer-level content addressable storage cannot efficiently reduce
duplicates and a file-level content addressable storage mode is suggested.
%
Furthermore, we applied chunk-level deduplication method on the 5TB unique files
and reduced the storage consumption to 1TB.
%
Moreover, we presents a comprehensive deduplication analysis followed by
different useful suggestions that docker registry designers need to know for
eliminating duplicates.
%
Last but not least, we collected a large range of statistics from our Docker
image dataset that can help to make conscious decisions when designing storage
for containers and Docker registry in particular.
%
%perform a large-scale characterization of Docker images.
%
%
%characterize them using multiple metrics, \eg image size distribution, layer
%size, the number of layers per image, and the amount of layers shared among
%images.
%
%\abc{add an example or two of the key findings?\nancomment{added findings to
%the end of intro}} For example, we find that small layers only have low
%compression ratios, suggesting that storing these layers uncompressed can help
%save computation while not sacrificing storage.
%
%Our findings help to make conscious decisions when designing storage for
%containers and Docker in particular.
\end{abstract}
