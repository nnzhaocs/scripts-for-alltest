\section{Related work}
\label{sec:related}

%\nancomment{do we need a relatedwork?}\\
%\vcomment{Absolutely!}
%\textbf{Analysis on Docker}:

Due to its increasing popularity, Docker has recently received increased
attention from the research community.
%
Slacker~\cite{slacker} studied 57 images from Docker Hub for a variety of metrics.
The authors used the results from their study to derive a benchmark to evaluate
the push, pull, and run performance of Docker graph drivers based on the studied
images. Compared to Slacker, our analysis focuses on the entire Docker Hub dataset.
%
Cito \etal~\cite{analysisdockergithub} conducted an empirical study for characterizing
the Docker ecosystem with a focus on prevalent quality issues, and the evolution of Docker 
files based on a data set of 70,000 Docker files. However, their study did not focus
on actual image data.
%
Shu \etal~~\cite{dockervulnerabile} studied the security vulnerabilities in Docker Hub
images based on a dataset of 356,218 images and found there is a strong need for more
automated and systematic methods of applying security updates to Docker images. While
the amount of images is similar compared to our study, Shu \etal focused on a subset
of 100,000 repositories and different image tags in these repositories.

Dockerfinder~\cite{dockerfinder} is a microservice-based prototype that allows searching
for images based on  multiple attributes, e.g., image name, image size, or supported software 
distributions. It also crawls images from remote Docker registry but the authors do
not provide a detailed description of their crawling mechanism.
%
Bhimani~\cite{dockerssd} \etal characterized the performance of persistent storage options
for I/O intensive containerized applications with NVMe SSDs. Unlike our study, their analysis
is focused on the execution of containers rather than on their storage at the registry side.

%\textbf{Analysis on file systems}:
%File system contents have been widely studied in different operating system environments.
%Douceur and Bolosky collected and analyzed over ten-thousand Windows file
%systems~\cite{largefscontent}. They found that the size of file and directory are fairly
%consistent across file systems, but file lifetimes and file-name extensions vary based on
%the job function of the users. 
%Agrawal, Bolosky, Douceur, and Lorch collected and analyzed a five-year file-system metadata
%over 60,000 Windows file systems, and presented the temporal trends relating to file size,
%file type, and directory size~\cite{fiveyearfsmetadata}.
%Sienknecht, Friedrich, Martinka, Friedenbach collected file system data from UNIX systems
%based on a dataset of 46 systems, 267 file systems, 151,000 directories and 2,300,000 files
%and found small files dominate in count~\cite{distributedatainfs}.
%%Traeger, Zadok, Joukov, and Wright surveyed 415 file system and storage benchmarks from
%106 recent papers~\cite{xxx}.  
