
\section{Methodology}
\nancomment{
	outline: \\
	1. our method to collect and download images \\
	(how, pull, extracting, manifest+layer tarball, server setup) \\
	2. our Dataset \\
	(size, table)
}
%Some embedded literal typset code might 
%look like the following :
%
%{\tt \small
%\begin{verbatim}
%int wrap_fact(ClientData clientData,
%              Tcl_Interp *interp,
%              int argc, char *argv[]) {
%    int result;
%    int arg0;
%    if (argc != 2) {
%        interp->result = "wrong # args";
%        return TCL_ERROR;
%    }
%    arg0 = atoi(argv[1]);
%    result = fact(arg0);
%    sprintf(interp->result,"%d",result);
%    return TCL_OK;
%}
%\end{verbatim}
%}
%
%Now we're going to cite somebody.  Watch for the cite tag.
%Here it comes~\cite{Chaum1981,Diffie1976}.  The tilde character (\~{})
%in the source means a non-breaking space.  This way, your reference will
%always be attached to the word that preceded it, instead of going to the
%next line.