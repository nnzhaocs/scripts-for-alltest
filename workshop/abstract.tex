\begin{abstract}

Due to their tight isolation, low overhead, and efficient packaging of the
execution environment, Docker containers have become a prominent solution for
deploying modern applications.
%
%Containers are created from images that preserve software dependencies,
%environment configuration, and other parameters that affect the application's
%runtime.
%
Containers are created from images which are stored in a Docker registry.
%
Image consists of a list of layers and layers are shared among images.
%
%Docker allows sharing of such images between users via a Docker registry.
%
Docker registries store a large amount of images and with the increasing
popularity of Docker, they continue to grow. For example, Docker
Hub---a popular public registry---stores more than half a million public images.
%
%As the amount of images stored in public and private Docker registries
%increases it becomes important to study images' characteristics.
%
%Investigating the storage-centric properties of Docker images can reveal useful
%insights about containerized applications and thereby prompt improvements in
%Docker design.
%
%The massive Docker Hub dataset offers a unique opportunity for such an
%endeavor.
%
%Our goal is to collect statistics from a large amount of Docker images and
%perform a large-scale characterization of Docker images.

In this paper, we download and analyze over 47~TB of Docker images from
Docker Hub and derive major trends and useful insights.
%
\LR{This is out of context, we haven't introduced layers yet. Can we put
a better highlight?}\NZ{addressed}
For example, we find that only 10\% of layers are referred to by more than one image.
%
We then evaluate the potential of file deduplication for the Docker registry.
%
Our analysis reveals that only ~3\% of the files in unpacked images are
unique.
%
Based on this finding, 
%we present the design for a Docker registry, which
%performs file-level deduplication across the stored images.
%
we conduct a simulation-based deduplication analysis 
%to 
%assess the feasibility of our design
and provide different suggestions to improve deduplication
performance and reduce the overhead on \textit{pull request}. 
%show that \LR{put 1 or 2 highlight results here}.\NZ{our one-node design is the major weakness of this paper}

%estimate performance, storage efficiency, and resource utilization

%
%Furthermore, we applied chunk-level deduplication method on the 5TB unique
%files and reduced the storage consumption to 1TB.
%
%Characterize them using multiple metrics, \eg image size distribution, layer
%size, the number of layers per image, and the amount of layers shared among
%images.
%
%For example, we find that small layers only have low compression ratios,
%suggesting that storing these layers uncompressed can help save computation
%while not sacrificing storage.
%
\end{abstract}
