\begin{abstract}

Due to the tight isolation, low overhead, and efficient packaging of the
execution environment, Docker containers have become a prominent solution for
deploying modern applications.
%
%Containers are created from images that preserve software dependencies,
%environment configuration, and other parameters that affect the application's
%runtime.
%
Containers are created from images which are stored in a Docker registry.
%
%Docker allows sharing of such images between users via a Docker registry.
%
The amount of data Docker registries store is massive; for instance, Docker
Hub---a popular public registry---stores at least a half million public images.
%
%As the amount of images stored in public and private Docker registries
%increases it becomes important to study images' characteristics.
%
%Investigating the storage-centric properties of Docker images can reveal useful
%insights about containerized applications and thereby prompt improvements in
%Docker design.
%
%The massive Docker Hub dataset offers a unique opportunity for such an
%endeavor.
%
%Our goal is to collect statistics from a large amount of Docker images and
%perform a large-scale characterization of Docker images.
%
In this study, we downloaded over 47~TB of real Docker images,
analyzed their content, and derived major trends and useful insights.
%
For example, we found that only 10\% of layers are referred by more than one image.
%
We then evaluated the potential of deduplication for the Docker registry.
%
Our analysis revealed that only ~3\% of the files in unpacked images are
unique.
%
We, therefore, designed Docker registry that
performs file-level deduplication of the images.
%
We present simulation-based estimates of performance, storage efficiency, and
resource utilization in this paper.

%
%Furthermore, we applied chunk-level deduplication method on the 5TB unique
%files and reduced the storage consumption to 1TB.
%
%Characterize them using multiple metrics, \eg image size distribution, layer
%size, the number of layers per image, and the amount of layers shared among
%images.
%
%For example, we find that small layers only have low compression ratios,
%suggesting that storing these layers uncompressed can help save computation
%while not sacrificing storage.
%
\end{abstract}
