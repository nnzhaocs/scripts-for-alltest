\section{Introduction}

\emph{Containers}~\cite{process-containers-linux} have recently gained
significant traction due to their low overhead and fast deployment and
the rise of container management frameworks such as Docker~\cite{docker}.
%
Polls suggest that 87\% of enterprises are at various stages of adopting
containers, and they are expected to constitute a \$2.5 billion
market by 2020~\cite{container-grow-by2020}.

Docker combines process containerization with efficient and effective packaging
of complete runtime environments in so called {\em images}.
%
Images are composed of shareable and {\em content addressable layers}.
%
A layer is a set of files, which are compressed in a single archive.
%
Both images and layers are stored in a Docker \emph{registry} and accessed by
clients as needed.
%
Since layers are uniquely identified by a collision-resistant hash of
their content, no duplicate layers are stored in the registry.

Registries are growing rapidly.
%
For example, Docker Hub~\cite{docker-hub}, the most widely used registry,
stores more than 500,000 public image repositories comprising over 2 million
layers and it keeps growing.
%
Over a period from June to September 2017, we observed a linear growth of the number
of images in Docker Hub with an average creation rate of 1,241 repositories per day.
%
We expect this trend to continue as containers gain more popularity.
%
%(as shown in Figure~\ref{fig_image_growth}.
%
This massive image dataset presents challenges to the registry storage
infrastructure and so far has remained largely unexplored.

In this paper, we perform the first, in-depth, large-scale redundancy
analysis of the images and layers stored in Docker Hub.
%
We download 47\,TB (167\,TB uncompressed) worth of Docker Hub images,
%
\LR{In the abstract it says 47TB?}\NZ{addressed}
%
which contain over 5 billion files in total.
%
%We started our analysis
%with collecting all the metadata over xxx images and  conducted the first
%comprehensive characterization of the image dataset (or union file systems). This analysis is particularly useful as no prior work exists on union file
%systems~\cite{xxx} and dataset used and created exclusively for Docker,
%which are different from either analysis on EX4 linux file system~\cite{xxx} or
%Windows file system~\cite{xxx}.
%%
%%We not only present the distribution for each metric but also pointed out the
%%challenges faced by Docker engine designer and registry designer based on
%%global knowledge (or analysis) of metadata. 
%We also provide useful insights (parameters or metrics)
%for the developers to better
%understand the file systems used by Docker container. 
%
%%Unfortunately, it's unknown whether  this coarse-grain layer-level content
%%addressable storage (LLCAS) can efficiently reduce duplicates, and how much
%%redundant data is stored in layer, image, and registry because there is no
%%prior research on registry dataset analysis.
%
%We continued with the deduplication analysis of the dataset.
%
Surprisingly, we find that only around 3\% of the files are unique
while others are redundant copies. This suggests that current layer-level content
addressable storage cannot efficiently reduce duplicates.
%
To analyze what are the redundant files and why there are so many,
we conduct a comprehensive redundancy analysis on the entire dataset.
%
We make three major observations:
%
\begin{compactitemize}
%
\item Only 10\% of layers are referred to by more than one image, 
meaning that layer-level content addressability is not enough to
effectively reduce storage utilization in the registry.
%
\item A large amount of files are shared across layers and images,
resulting in only 3\% of unique files.
%
\item \LR{This point is unclear. If the source code is duplicated, I would expect that to be the major
source of redundancy, not the resulting executables? Isn't that also what we're saying in Section 3?}
The majority of the file duplicates are executables and object files, which are
mostly created by source code duplicates.\NZ{I did see a lot of executables/intermediate representations have the same filename with the source codes/scripts. 
	For example, a.py and a.pyc. But i did not calculate the total number.
	How about we replace with: }
	We found source codes and scripts have a very high deduplication ratio (\textbf{$31.25\times$} for souce codes and \textbf{$50\times$} for scripts), which indicates that Docker developers are prone to replicate source codes or scripts (i.e., code reuse). Moreover, these source code duplicates and script duplicates would result in more executable duplicates and object file duplicates.

\end{compactitemize}

As one reason for the large number of duplicate files, we found that different Docker image are prone
to contain the same source code from external public repositories (e.g., GitHub~\cite{github}). As
there are no official images containing this source code, users manually add it to their images,
resulting in different layers which cannot be reused later.  

Based on our findings, we propos a file-level content addressable storage model (FLCAS
\LR{Can we come up with a more catchy acronym?}) for the Docker registry, which utilizes
file-level deduplication to remove redundant files. 
%
We simulate FLCAS for 0.9 million layers and provide different suggestions to improve
deduplication performance.
%
\LR{Do we have some numbers on overhead/saved storage space that we can put here?
If so, we should replace the two suggestions below by those numbers.}\NZ{We can just say we provide some hints for dedup implementation in registry by using a simple simulation.}
%
The simulation result show that (1)~processing layers in
parallel can largely improve throughput. For example, 80\% of file-level deduplication time is
less than 9.09 s per layer and by processing 60 layers in parallel, our one-node
prototype can process about 3 layers per second.
%
\LR{That sounds like we actually ran the deduplication and not just simulated it?
Did we perform more of an \emph{emulation}?}
%
(2) Fast compression methods can mitigate pull overhead caused by re-compression
because files are required to be re-compressed as a compressed layer archival file to serve
the incoming pull requests.
%


%%%%%%%%%%%%%%%%%%%%%%%%%%%%%%%%%%%%%%%%%%%%%%%%%%%%%%%%%%%%%%%%%%%%%%%%%%%%%%
%                                                                            %
%                                OLD INTRO                                   %
%                                                                            %
%%%%%%%%%%%%%%%%%%%%%%%%%%%%%%%%%%%%%%%%%%%%%%%%%%%%%%%%%%%%%%%%%%%%%%%%%%%%%%

%Finally, we proposed and implemented Docker registry design that performs
%deduplication.
%%
%In our thorough redundant analysis and characterization of the xxxx images,
%with xxxx layers and xxxx files, we investigated the following four research
%questions (RQs):
%
%\begin{compactitemize}
%%
%\item How much redundant data stored in layers, images, and registry? Although
%layer-level address content addressable storage is adopted by Docker, we do not
%know whether  this coarse-grain layer-level content addressable storage (LLCAS)
%can efficiently reduce duplicates, and how much redundant data is stored in
%layer, image, and registry.
%%
%\item What are the redundant files and why there are so many redundant files?
%We aim to identify what are the redundant files that users mostly replicate.
%%
%Such information will provide Docker designer knowledge (user behavior) to
%better develop and optimize Docker container and Docker registry storage
%system.
%%
%\item What are the challenges faced by Docker registry and engine designer? By
%characterizing and analysis all the image metadata, we aim to identify the
%challenges' faced by registry designer and guide designers'optimization and
%users' development.
%%
%\item How to reduce the redundant files? We aim to propose a file-level content
%addressable model to reduce the redundant files by using file-level dedup while
%maintaining a good performance.
%%
%\end{compactitemize}
%
%The significance of this work are (1) our empirical evidence that large amount
%of redundant files exist in layers, images, and registry and layer-level
%content addressable storage is not efficient to remove redundant files;(2)
%findings about what are the redundant files and why there are so many redundant
%files exist;(3) first in-depth characterization on image dataset (union file
%systems)(4) a file-level content addressable model that can efficiently remove
%redundant copies while maintain a good performance.

%For years, virtual machines served as a cornerstone of computing resource
%virtualization both on premises and in the cloud~\cite{rosenblum2005virtual}.
%%
%Recently, however, \emph{container-based} virtualization started to gain
%significant traction~\cite{process-containers-linux}.
%%
%According to polls, over 87\% of enterprises are at various stages of adopting
%containers; analysts also predict that by 2020, containers will constitute a
%lucrative \$2.5 billion market~\cite{container-grow-by2020}.
%
%
%
%At its core, container is a set of processes which are isolated by the operating
%system kernel in terms of visibility and resources. This allows containers to share
%the same kernel without being aware of each other.
%%
%For example, Linux performs visibility isolation (for user identifiers, file systems,
%network, etc.) using namespaces~\cite{man-namespaces} and enforces resource
%utilization constraints with control groups~\cite{kernel-doc-cgroups}.
%%
%Compared to virtual machines, containers use less memory and storage, are much
%faster to start, and typically cause less execution
%overhead~\cite{felter2015updated, Disco, HypervisorsvsLightweight}.
%
%The rapid increase in use of container technology was largely made possible by
%container management frameworks, with Docker being one of the most popular
%solutions~\cite{docker}.
%%
%Docker combines process containerization with efficient and effective runtime
%environment packaging.
%%
%Software is packaged in container \emph{images}, each consisting of several
%read-only \emph{layers} and a manifest which describes container metadata, \eg
%what layers make up an image and which command to run at container startup.
%%
%Read-only layers can be shared between different images and encapsulate
%file-system trees for dockerized processes.
%
%%Docker is another technology whose popularity grew rapidly in the recent
%%years~\cite{docker}.
%%
%%When Docker starts a container, it combines read-only layers (and an additional
%%writable layer to store changes) into a single namespace and starts the process
%%declared in the manifest in the new namespace~\cite{docker-driver-eval}.
%
%
%
%Docker images are stored in a centralized \emph{registry} and are pushed to and
%pulled from the registry by clients as needed.
%%
%Docker Hub~\cite{docker-hub} is the most widely used Docker registry
%installation which, according to our estimates, stores more than 400,000
%\emph{public} image repositories comprising a total of 2 million layers.
%%
%This amount is steadily increasing and we observed a linear growth of the
%number of images over a period from June to September 2017.
%
%
%
%While this massive dataset presents challenges to the registry storage
%infrastructure, it also provides opportunities to better understand how
%containers are used in practice.
%%
%Currently, there is little known about the contents, use cases, and workloads
%of production containers.
%%
%In part, this is due to the privacy concerns that organizations and individuals
%have when sharing details of their computing environments.
%%
%However, this knowledge is imperative to design and evaluate novel approaches
%to improve the performance and reliability of containers.
%
%
%
%
%In particular, storage for containers has remained a largely unexplored
%area~\cite{login-container-storage-options}.
%%
%We believe one of the prime reasons is the limited understanding of what data
%is stored inside containers.
%%
%This knowledge can not only help to directly improve the registry and container
%storage infrastructure but also allows to infer container use cases and derive
%representative workloads from that.
%%
%While existing work as focused on various aspects of
%containerization~\cite{slacker, dockervulnerabile, dockerfinder, analysisdockergithub, dockerssd}, analyzing the
%contents of images and layers has not received much attention.
%
%
%
%
%%Though much research was focused on various aspects of
%%containerization~\cite{prev-work-1, prev-work-2, prev-work-3}, storage for containers
%%remains an unexplored territory~\cite{login-container-storage-options}.
%%
%%To start designing a novel storage solution for containers,
%%or to optimize and fairly evaluate existing ones,
%%it is imperative to understand containers' real-world
%%use cases and workloads in sufficient details.
%%
%%Unfortunately, little is known about how containers are used in the real world.
%%
%%In part, this is due to the privacy concerns that organizations and individuals
%%have when sharing details of their computing environments.
%
%
%%Docker images are stored at the centralized \emph{registry} and are pushed to
%%and pulled from the registry by clients as needed. 
%%
%%The most known Docker registry installation is Docker Hub which according to
%%our estimates stores at least 400,000 \emph{public} images that consist of at
%%least 2,000,000 layers.
%
%
%
%
%In this paper we perform the first, comprehensive, large-scale characterization of
%Docker registry contents.
%%
%We downloaded over 50TB of Docker images from Docker Hub and analyzed
%traditional storage properties---\eg, file sizes and types, data compression
%ratios, directory depths---as well as Docker-specific properties---e.g., the number
%of layers per image and the amount of layer sharing.
%%
%%Our insight in this study is that this massive dataset can be used to understand what
%%applications run in containers, how much data they store, and the properties of
%%the data.
%%
%We found, for example:
%\begin{compactenumerate}
%	\item 90\% of the repositories only have a very small pull count (less than 333), which suggests that Docker hub is a good fit for caching few popular repositories or images.
%	\item majority of the images and layers in Docker hub have a smaller size. 90\% of images can be compressed with less than 500 MB and 70\% of images are less than 500 MB even without compression. 90\% of layer can be compressed with less than 63 MB and 77\% of layers are less than 63 MB even without compression.
%	\item Docker images has a great potential for compression to save space.
%	\item 90\% of images have less than 18 layers. Half of images have less than 8 layers. 
%	\item 10\% of layers are referenced more than one image.
%	\item Around 90\% of layers' directory depth is less than 30. 50\% of layers' directory depth is less than ~3.
%	\item Around 30\% of files are ASCII text files. 
%	About 11\% files are gzip compressed files 
%	Interestingly, about 1\% of files are empty. 
%\end{compactenumerate}
%%
%\vcomment{Here we need to stick an example or two of interesting findings. \nancomment{addressed}}
%
%%From our findings, we infer a set of propositions to describe how Docker is
%%used in the real world:
%%\lrcomment{Can we summarize our findings in a few propositions to put here?}.
%%
%%We believe our findings will improve the understanding of containers' data and lay
%%a solid ground for future storage optimizations at clients and registries in
%%Docker and beyond.
%
%After introducing the Docker background~(\S\ref{sec:background}),
%this paper makes the following contributions:
%\begin{compactenumerate}
%  \item We describe a comprehensive methodology to retrieve the complete set of
%  	images stored in Docker Hub~(\S\ref{sec:methodology});
%  \item We perform the first in-depth analysis of container images stored in
%    Docker Hub~(\S\ref{sec:char}).
%%  \item based on our analysis, we formulate propositions on how Docker is currently
%%    used to help guide optimizations and benchmark
%%    workloads~(\S\ref{sec:propositions}).
%\end{compactenumerate}
%
%After discussing related work~(\S\ref{sec:related}),
%the paper concludes~(\S\ref{sec:conclusion}).
%
%%The rest of the paper is organized as follows. We explain
%%relevant Docker details in Section~\ref{sec:background} and our methodology in
%%Section~\ref{sec:methodology}. We present dataset characterization in
%%Section~\ref{sec:results}, describe related work in Section~\ref{sec:related},
%%and conclude in Section~\ref{sec:conclusion}.
