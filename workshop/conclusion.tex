\section{Conclusion}
\label{sec:conclusion}

In this paper, we carried out the first comprehensive analysis of the container images
stored in Docker Hub.
%
We presented a methodology to exhaustively crawl and efficiently download Docker Hub
images. Using this approach, we downloaded a total of 346,243 images and 1,763,354 layers
resulting in 51 TB of data and
%through 8.8\% layers and 9.2\% images cannot be analyzed due to extracting errors. 
117 million files.
%
Based on this dataset, we carried out a detailed study of a variety of storage
metrics on both layers and images. Metrics included layer and image sizes, compressibility,
and popularity. Our findings reveal that there is room for optimizing how images
are stored and used. For example, we observed that compression may not always be beneficial for small
layers as it can affect pull latencies. Additionally, layers are rarely shared between
images which increases storage utilization. We plan to investigate such improvements
in the future.

%Our interesting observation is that there are a number of layers' compression ratio are quite
%low. Given that compression is computationally expensive and often slow down the \textit{pull}
%process, it can be beneficial to store small layers uncompressed in the registry to reduce pull
%latencies. 
%
%We analyzed the file count and directory count distribution for both layers and images. We 
%find that majority of images and layers has a moderate number of directories and files, with 
%less images and layers have extremely large or small number of directories and files.
%
%We also analyzed the layer count distribution for images. Majority of images has a moderate 
%number of layers. And very less layers are shared among different images.   
