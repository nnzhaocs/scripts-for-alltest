\section{Preliminary evaluation}
\label{sec:Evaluation}

%Our two-tier heterogeneous cache in the registry can improve the performance of the entire system
%by hiding the long latency imposed by the backend dedup system.
%Next, we present the preliminary evaluation of our user access history-based cache algorithm
%and the space efficiency of file cache.

\begin{figure}[t]
	\centering
		\begin{minipage}{0.23\textwidth}
			\centering
			\includegraphics[width=1\textwidth]{graphs/evaluation_hitratios.png}
			\caption{Hit ratio.}
			\label{fig:hitratio}
		\end{minipage}
		\begin{minipage}{0.24\textwidth}
			\centering
			\includegraphics[width=1\textwidth]{graphs/percentage_morelayers_afterdedup.png}
			\caption{File-level deduplication vs. compression efficiency. \NZ{figure needs redraw}}
			%\vspace{-3pt}
			\label{fig:cacheefficiency}
		\end{minipage}
\end{figure}
\vspace{-6pt}
\paragraph{Cache hit ratio.}

We simulate our user-access-history-based cache and 
replay the \texttt{dal} workload~\cite{dockerworkload} to measure the hit ratio.
We set our cache size is 20\%of ingress data and 
within the cache, we use 10\%of cache for buffering incoming \texttt{put} layer requests
 and the rest space in cache for caching prefetched layer slices from backend dedup system. 
%as shown in Figure~\ref{fig:hitratio}.
%Our algorithm exhibits an enhanced cache performance, with a high hit ratio
%up to 0.96.
Figure~\ref{fig:hitratio} shows the results. We observe a significant increase in the hit ratio, $74\%$ to $95\%$ as the duration threshold grows from $1$ to $10$ minutes. This is because prefetched layers are kept in the cache for more time.
The hit ratio stablizes at $96\%$ as the duration threshold increases from $15$ to $20$ minutes.
%Therefore,
%the highest hit ratio of our algorithm is around 0.96,
%and there are 0.4 of layers that are miss because 
%some users \emph{re-pull} the layers after they pull the same layers.
The layers responsible for the $4\%$ miss rate are the ones being
\emph{re-pulled} by the same user.
%We also see that 74\% of users can finish their \texttt{pull} layer request
%within a minute and 
%around 89\% of users can finished their \texttt{pull} layer request with less than 5 min. 
%We also see that the response time to \texttt{pull} layer requests is within $1$ minute for 74\% of users, and it is less than $5$ minutes for 89\% of users.
%Since we buffer layers upon a \texttt{push} layer request and prefetch layer {\em slices} from the backend servers, 
%we compare the hits on prefetched layer {\em slices} and the hits on buffered layers.
We see that, across different duration thresholds, 
the hits upon buffering incoming put requests is very low,
confirming that it takes a long time for a recently \emph{pushed} layer to be pulled.
We also observe that our average cache utilization is only 22\%, which means that
the cache only few peak periods is filled up because many users are connected 
and the requested repositories contains many layers, which means 
user accesses are non-uniform with peaks and troughs,
and our algorithm can adapt to workload changes.
%This trend is perfectly fit in our two-tier heterogeneous cache
%and the evaluation results can guide us to carefully choose 
%layer buffer size and file cache size.
%
%In terms of file count, it increases from \textbf{3.6$\times$} to \textbf{31.5$\times$} while
%in terms of capacity, it increases from \textbf{1.9$\times$} to
%\textbf{6.9$\times$} as the layer dataset grows from 1000 to 1.7 million layers.
%%
%This confirms the high potential for file-level deduplication in large-scale
%Docker registry deployments.

\vspace{-6pt}
\paragraph{Space efficiency.}
%As shown in Figure~\ref{xxx},
%we compare the cache hit ratio of LRU, Prefetch~\cite{xxxx}, and our 
%user-based cache replacement algorithm by replaying three IBM container registry workloads~\cite{dockerworkload}.
%Our user-based cache exhibits an enhanced cache performance, with hit ratio improvements ranging from 
%0.2 to 0.3 for all the three workloads compared to LRU.
We analyze the space efficiency for the file cache compared to a cache that naively stores
compressed layers.
%for an increasing number of files stored in the file cache 
%(see Figure~\ref{fig:dedup-ratio-growth}).
%
%Figure~\ref{fig:dedup-ratio-growth} shows the deduplication ratio growth over the layer dataset size.
%
In Figure~\ref{fig:cacheefficiency}, the x-axis values correspond to the sizes of $4$ random samples drawn from the whole dataset and the size of the dataset in terms of capacity and layer count.
For a traditional cache, the compressed layer tarballs will be kept as is.
While \sysname will store \emph{deduped} layers. 
%unique files in the file cache.
The y-axis shows how many more \emph{deduped} layers can fit in our file cache compared to naively storing compressed layer tarballs.
For the first two samples of the dataset, with size less than $20$~GB, 
there is no benefit to \emph{dedup} layers 
because the deduplication ratio is very low.
However, when the dataset size is $3$~TB, we can store $56\%$ more \emph{deduped} layers' unique files in file cache.
The number of extra \emph{deduped} layers that can fit in the file cache increases almost linearly with the size of the layer dataset.
%the bigger the dataset, the more deduped layers that can fit in the file cache.
%The number of extra layers increases almost linearly with the layer dataset size.
%In this case, there is a high potential for file cache when the cache size is big.
This verifies the benefit of the file cache when the cache size is large, which should be carefully selected to realize significant space savings.


%In this experiment, we show how many more layers can be stored in the file cache 
%after decompression and file-level deduplication.
%Figure~\ref{xxxx} shows the growth of deduplication ratio with different dataset sizes.


%We evaluated \sysname's performance improvement over traditional 

%%While \sysname\ can effectively eliminate redundant files in the
%%Docker registry, it introduces overhead which can reduce the
%%registry's performance.
%%
%%The overheads can be classified in two categories: 1)~\emph{background
%%overhead} caused by the computation and I/O that is performed during layer
%%deduplication; and 2)~\emph{foreground overhead} from extra processing on the
%%critical path of a pull request.
%%\begin{figure}
	\centering
	\includegraphics[width=0.48\textwidth]{graphs/res-time.pdf}
	\caption{Off-line file-level deduplication run time.}
	\label{fig:dedup-res}
\end{figure}

%
%
%%\paragraph{Hit ratios}
%%
%%\paragraph{Hit ratios with prefetching}
%%
%%%\subsection{} % what are the cost for a naive file-level deduplication
%%
%%\paragraph{Restoring performance breakdown}
%%
%%\paragraph{Simulation}
%%
%%To analyze the impact of file-level deduplication on the registry performance,
%%we conduct a preliminary simulation-based study of \sysname.
%%
%%Based on the simulation results, we estimated the overhead of \sysname\ on
%%\texttt{push} and \texttt{pull} layer request latencies.
%%
%%We then provide different suggestions on how the Docker registry can mitigate
%%the deduplication overhead.
%%
%%%%%%%%%%%%%%%%%%%%%%%%%%%%%%%%%%%%%%%%%%%%%%%%%%%%%%%%%%%%%%%%%%%%%%%%%%%%%
%%
%%
%Our simulation
%approximates several of \sysname's steps as described in Section~\ref{sec:design}.
%%
%First, a layer from our dataset is copied to a RAM disk. 
%%
%%
%%Note that there is no foreground pull or push requests since the simulation is \emph{off-line}.
%%
%The layer is then decompressed, unpacked, and the fingerprints of all files
%are computed using the MD5 hash function~\cite{MD5}.
%%
%The simulation searches the fingerprint index for duplicates,
%and, if the file has not been stored previously, it records the
%file's fingerprint in the index.
%%
%%To map a layer to its containing files, we create the layer recipe and add it
%%to a \emph{layer-to-file table}.
%%
%%The simulator then creates a file recipe.
%%
%%For each file in a layer, a layer digest
%%to its containing file content digest mapping record is also created 
%%
%%The \emph{layer-to-file table} also
%%records the file path within each layer associated with each file.
%%
%At this point our simulation does not include
%the latency of storing unique files.
%%
%To simulate the layer reconstruction during a \texttt{pull} request,
%we archive and compress the corresponding files.
%%
%%Only unique files are maintained in RAM
%%disk while the redundant copies are removed.
%%
%
%The simulator is implemented in 600 lines of Python code
%and our setup is a one-node Docker registry on a machine with 32~cores and 64\,GB of RAM.
%%
%To speed up the experiments and fit the required data in RAM
%we use 50\% of all layers and exclude the ones larger than 50\,MB.
%%
%We process 60 layers in parallel using 60 threads.
%%
%The entire simulation took 3.5 days to finish.
%%
%%The overall runtime is about 3.5 days.
%
%Figure~\ref{fig:dedup-res} shows the CDF for each sub-operation of
%\sysname.
%%
%Unpacking, Decompression, Digest Calculation, and Searching 
%are part of
%the deduplication process and together make up the Dedup time.
%%
%%\VT{@Nannan, in Figure ~\ref{fig:dedup-res} can you reorder the lines in the
%%legend so that the Searching goes after Digest calculation?}\NZ{addressed}
%%
%Searching, Archiving, and Compression
%simulate the processing for a \texttt{pull}
%request and form the Pulling time.
%%
%
%%\LR{What was the overall runtime for processing 0.9 million layers?}\NZ{addressed}
%%
%%\alicomment{How are we saving the location
%%of each file in the layer? It is not clear from the following sentences.}
%%\NZ{addressed}
%%
%%To improve searching performance, the
%%mapping table is stored in Hive database~\cite{xxx}. 
%%
%%\lrcomment{Why are we using Hive for this? It seems overkill to me, especially
%%for such small data. Even at scale, a KeyValue store would probably provide
%%better performance than clunky MapReduce-based DB.}
%%
%
%\paragraph{Push}
%
%\sysname\ does not directly impact the latency of \texttt{push} requests because
%deduplication is performed asynchronously.
%%ie the registry reliably stores a
%%copy of the layer as-is and then sends a response to the client.
%%
%The appropriate performance metric for \texttt{push} is the time it takes to deduplicate
%a single layer.
%%
%%Next, we look at the effects on \texttt{push} and \texttt{pull} latencies in
%%more detail.
%%
%%However, if there are intensive push requests while the registry is performing
%%deduplication, \sysname\ can still impact push latencies because it incurs
%%CPU, memory, and I/O overhead. %(similar to pull requests).
%%
%Looking at the breakdown of the deduplication time in
%Figure~\ref{fig:dedup-res}, we make several observations.
%
%First, the searching time is the smallest among all operations with 90\% of the
%searches completing in less than 4\,ms and a median of 3.9\,ms.
%%
%%The mapping table maintains 0.98 million layer-to-file digest mapping records. 
%%
%%\LR{Remove the following sentence? 1.7 million records is actually quite small
%%so even a single-node DB with one index is enough.}\NZ{addressed} Consider
%%that more than 1.7 million layers are stored in Docker hub and the number is
%%still increasing, it's better to choose a fast distributed database to provide
%%high searching performance and scalability.
%%
%Second, the calculation of digests spans a wide range from 5\,$\mu$s to almost
%125\,s.
%%
%%This is because the time mainly depends on the layer size, \ie the fewer and
%%smaller files a layer contains, the faster it is to compute all digests for
%%the layer.
%%
%%Typically, smaller layers contain a smaller number of smaller files, which
%%takes much less time to calculate their digests.
%%
%%While if the layer is bigger, the digest calculation overhead will be higher. 
%%
%90\% of digest calculation times are less than 27\,s while 50\% are
%less than 0.05\,s.
%%
%The diversity in the timing is caused by a high variety of layer sizes both in
%terms of storage space and file counts.
%%
%%Thus, we suggest that multiple-threading is needed to calculate the files'
%%digests simultaneously; 
%%
%%Fast CPUs as well as more powerful computing nodes are required to speed up
%%digest calculation.
%%
%Third, the run time for decompression and unpacking follows an identical
%distribution for around 60\% of the layers and is less than 150\,ms.
%%
%%Around 60\% of decompression and unpacking time are less than 0.15\,s. 
%%
%However, after that, the times diverge and decompression times increase faster
%compared to unpacking times.
%%
%%\VT{do we have some theory why?}
%%\NZ{decompressing the layers with bigger uncompressed size takes longer time.}
%%
%90\% of decompressions take less than 950\,ms while 90\% of packing time is less
%than 350ms.
%
%%Overall, we see that file digest calculation contributes a lot to the
%%overall deduplication latency especially when the layer size is big.  Moreover,
%%we see that the deduplication latency increases as the layer size grows.
%%
%Overall, we see that 90\% of file-level deduplication time is less than 35\,s
%per layer, while the average processing time for a single layer is 13.5\,s.
%%
%This means that our single-node deployment can process about 4.4\,layers/s on average
%(using 60 threads).
%%
%In the future we will work on further improving \sysname's deduplication throughput.
%%
%%In a large-scale registry deployment, this throughput can be improved
%%as more node are available to perform deduplication.
%%
%
%\paragraph{Pull} 
%
%From Figure~\ref{fig:dedup-res}
%we can see that 55\% of the layers have close compression and archiving
%times ranging from from 40\,ms to 150\,ms and both operations contribute equally
%to pulling latency.
%%
%%60\% of compression and archiving time are less than 0.15 s.
%%
%%While compression has the highest run time 80\% of compression time is less than 2.82~s. 
%%
%%\LR{Again, better to show the 90th percentile.}
%%\NZ{90\% of the compression time is less than 8\,s.}
%After that, the times diverge and compression times increase faster with an
%90\textsuperscript{th} percentile of 8\,s.
%%
%This is because compression times increase for larger layers and follow the distribution
%of layer sizes (see Figure~\ref{fig:layer-size-cdf}).
%%
%%80\textsuperscript{th} percentile of 2.82\,s.
%%
%Compression time makes up the major portion of the pull latency and is a
%bottleneck.
%%
%Overall, the average pull time is 2.3\,s.
%
%%
%%We see that archiving time and compression contributes equally to pulling
%%latency when their run time are lower than 0.15 s while compression time almost
%%equals to pulling latency when the compression time is greater than 0.15 s. 
