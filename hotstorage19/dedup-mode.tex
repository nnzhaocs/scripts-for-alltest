\subsection{Deduplication modes}
\label{sec:dedup-mode}

To satisfy different space saving and performance requirements,
\sysname provides two kinds of deduplication modes:
Basic deduplication modes and selective deduplication modes, denoted as \emph{B-modes} and \emph{S-modes}.
B-mode mantains a certain amount of layer replicas for each layer and deduplicates the rest layer replicas.
While s-mode mantains a number of layer replicas for different layer proportionate to their popularity.

\subsubsection{Basic deduplication modes}
Assume that registry storage system uses R-way replication.
Consider that layer dataset deduplication analysis~\cite{xxx} shows a file-level deduplication ratio of 2$\times$.
Thus, the average layer deduplication ratio for the layer dataset~\cite{xxx} is 2$\times$.

B-mode \emph{r} is defined as keeping \emph{r} layer replicas and deduplicating the rest of layer replicas, where $r \leq R$. 
The space saving is calculated by following equation~\ref{eq:b-savings}.
\begin{equation}\label{eq:b-savings}
Space savings = \frac{R-(r+\frac{1}{2}(R-r))}{R} = \frac{R-r}{2R}
\end{equation}

Layer pulling performance largely depends on registry server load.
Assume that the total number of P-servers and D-servers are \emph{P} and \emph{D} especially.
The total \texttt{pull} layer request load is \emph{L}.
Consequently, the average server load is $\frac{L}{P}$ and $\frac{L}{P+D}$ for \sysname and original registry respectively.
Note that original registry is \sysname's B-mode \emph{R}, which represents the best pulling layer performance with no space savings.
The performance degradation can be calculated as:
%$$
 %$\emph{f}_\frac{L}{P} \times \theta_{r}$. 
 
 \begin{equation}\label{eq:c-pull}
Performance degradation =  \Delta(\frac{L}{P} - \frac{L}{(P+D)})\times \delta_{R-r}
 \end{equation}
 where $\Delta(\frac{L}{P} - \frac{L}{(P+D)})$ denotes the performance degradation caused by server load increases.
 $\delta_{R-r}$ means the performance degradation caused by load variance among different servers impacted by replication level.

\subsubsection{Selective deduplication modes}
In S-mode, the number of kept layer replicas \emph{r} is proportional to their popularity ($r \leq R$).
Layer popularity is calculated as $\frac{\emph{l}_{pcnt}}{L}$, 
where $\emph{l}_{pcnt}$ is the total number of pulling request count to layer \emph{l}.  
Consequently, the number of kept layer replicas $r_{l}$ for layer \emph{l} is
%\emph{r} ($r \leq R$) is calculated as:
$r_{l} = R \times \Phi (\frac{\emph{l}_{pcnt}}{L})$, 
where $\Phi (\frac{\emph{l}_{pcnt}}{L})$ denotes a replication factor.

The space savings for S-mode is calculated as following equation~\ref{eq:s-savings}.
\begin{equation}\label{eq:s-savings}
Space savings = \frac{R-(\sum\limits_{l=1}^n(r_{l}+\frac{1}{2}(R-r_{l})))}{R} 
\end{equation}

The performance degradation is:
  \begin{equation}\label{eq:s-pull}
Performance degradation =  \Delta(\frac{L}{P} - \frac{L}{(P+D)})\times \prod\limits_{l=1}^n\delta_{R-r_{l}}
 \end{equation}
where $\prod\limits_{l=1}^n\delta_{R-r_{l}}$ means 
the performance degradation caused by load variance among different servers impacted by different replication levels.
 