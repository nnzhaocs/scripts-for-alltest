\subsection{Deduplication modes}
\label{sec:dedup-mode}

To satisfy different space saving and performance requirements,
\sysname provides two kinds of deduplication modes:
Basic deduplication modes and selective deduplication modes, denoted as \emph{B-modes} and \emph{S-modes}.
B-mode mantains a certain amount of layer replicas for each layer and deduplicates the rest layer replicas.
While s-mode mantains a number of layer replicas for different layer proportionate to their popularity.

\subsubsection{Basic deduplication modes}
Assume that the storage system uses R-way replication.
Consider that layer dataset deduplication analysis~\cite{xxx} shows a file-level deduplication ratio of 2$\times$.
Thus, we assume the average layer deduplication ratio is 2$\times$.

B-mode \emph{r} is defined as keeping \emph{r} layer replicas and deduplicating the rest of layer replicas. 
The space saving is calculated by following equation~\ref{eq:b-savings}.
\begin{equation}\label{eq:b-savings}
S = \frac{R-(r+0.5(R-r))}{R} = \frac{R-r}{2R}
\end{equation}

%Layer pulling performance is depends on layer access parallelism.

\subsubsection{Selective deduplication modes}
S-mode

$r = R \times \frac{\emph{W}_{pcnt}}{L} \times \delta$

S-mode ($\delta$, $\emph{f}_{pcnt}$) is defined as keeping \emph{r} layer replicas for the top-$\delta$ hot layers.
\begin{equation}\label{eq:s-savings}
S = \frac{R-(r+0.5(R-r))}{R} = \frac{R-r}{2R}
\end{equation}