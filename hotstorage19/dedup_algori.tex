\paragraph{User active time-based cache replacement algorithm}
%\label{sec:design_cache_algori}

\begin{algorithm}
	%\scriptsize 
	\caption{User-based cache replacement algorithm.}
	\label{alg:cache}
	\SetAlgoLined
	\KwIn{
		$S_{thresh}$: Threshold for layer buffer to trigger eviction.
	}
	\While{free\_buffer $<$ $S_{thresh}$}{
		$last\_usr \gets UsrLRU.last\_item()$\\
		\For{layer in reversed(LayerLRU.items())}{
			\If{layer exclusively belongs to last\_usr}{
				%{\scriptsize $/*$\textit{If layer is not shared with other users, layer is deleted}}\\
				\textbf{Evict} $LayerLRU[layer]$\\			
				$free\_buffer += sizeof(layer)$
		}
	}
	\textbf{Evict} $UsrLRU[last\_usr]$
	}
\end{algorithm}

We leverage observations from our study of the user access patterns to guide our cache replacement. This is because
a user's active time is more predictable as discussed in~\cref{sec:background}.
%   In~\cref{sec:design}, the incoming \texttt{push} layer requests are first buffered in the layer buffer and later evicted to the file cache.
%   If there is a \texttt{pull} layer request miss in both the layer buffer and the file cache, 
%   \sysname~will fetch the layer from backend storage system and store it in the layer buffer.
We service requests from the layer buffer, then file cache, and then from the backend storage system. 
When a cache (layer buffer or file cache) is under pressure due to 
shortage of free space,
the layer buffer or the file cache will simply evict or delete some layers or files and reclaim space.
Since our layer buffer and file cache both share the same cache replacement algorithm, we only present
our user-based cache replacement algorithm for the layer buffer as shown in Algorithm~\ref{alg:cache}.

Free space in the layer buffer is considered too low if ($free\_buffer$ $<$ $S_{thresh}$). 
In such a sitation, \sysname~will free the buffer space used by inactive users. 
\sysname~maintains two LRU lists: an LRU list of active users, and
an LRU list of recently accessed layers, referred to as \emph{UsrLRU} and \emph{LayerLRU}, respectively.
At first, \sysname~will select the least active user from \emph{UsrLRU}.
Next, \sysname~will \arb{what is???: reversely iterate} \emph{UsrLRU} until it finds a layer that is exclusively owned by that least active user. 
\sysname~then evicts this layer from the layer buffer. 
 Note that layers are shared among different active users.
 \sysname~will continue to evict until the free space in the layer buffer is above the specified capacity threshold.
 
%iteratively removes the least active users along with the least recently accessed layers 
% that exclusively owned by these users from the two LRU lists and evict the layers from layer buffer.
%, then   

\paragraph{User access history-based prefetch algorithm}

To improve our cache hit ratio for \texttt{pull} layer request, we proposed a user access history-based prefetch algorithm
by utilizing the uniqueness of registry image dataset hierarchy or organization: User $\rightarrow$ repositories $\rightarrow$ layers.
Users create repositories in Docker registry, \texttt{push} layers to their own repositories,
and \texttt{pull} layers from their own repositories or other public repositories. 
%When user \texttt{pull} an image from registry, it will first pull the manifest of the image while user 
%\texttt{push} an image to registry~\cite{docker}~\cite{dockerworkload}. 
Based on the above behavior pattern and hierarchy of image in Docker registry, 
we can record user access history and,
when a new use connects to registry, we can prefetch all the layers that have been accessed by this user
into cache ideally.
Gradually, cache hit ratio will almost reach 1.
However, limited to the cache size, we won't be able to fit all the active users' layers into cache.
But we can prefetch active users' popular layers in the cache based on the historical access information.

\begin{algorithm}
\scriptsize 
	\caption{Preconstruction}
	\label{alg:prefetch}
	\KwIn{\\
		%$\theta_{rp}$: Threshold for popular repulling clients. \\
		%{\tiny\texttt{/*when $L\_timer[layer] > L\_thresh$, layer is evicted or demoted to Flashcache */}}\\
		$\theta_{rpc}$: Threshold for repull layers to be preconstructed and cached. \\
		%$\theta_{R}$: Threshold for popular repulling layers. \\
		$RLmap$: Repository to layers map.\\
		$URLmap$: User to user profile map. \\
	}
	\While{true}{
		\emph{r} $\leftarrow$ \texttt{request received}\\
		\uIf{r = GET manifest}
		{
			newLayers $\gets$ \emph{RLmap[r.repo]} $-$ \emph{URLmap[r.client].layers} \\
			oldLayers $\gets$ \emph{RLmap[r.repo]} $\bigcap$ \emph{URLmap[r.client].layers} \\
			%\emph{layers} $\gets$ \emph{RLmap[r.repo]} $-$ \emph{URLmap[r.client]} \\
			%\emph{OnTimelayers}, \emph{NotOnTimelayers} $\gets$ \emph{OnTimeCalculation(layers)} \\
			%\emph{MEMcache} $\gets$ \emph{Prefetch(OnTimelayers)} \\
			%\emph{FLASHcache} $\gets$ \emph{Prefetch(NotOnTimelayers)} \\
			%\texttt{set} \emph{L\_timer[layer] for each layer in layers} \\
			\emph{cache} $\leftarrow$ \texttt{Preconstruct} newLayers \\
			\If{URLmap[r.client, r.repo, oldLayers].rp $>$ $\theta_{rpc}$} 
			{ 
				\emph{cache} $\leftarrow$ \texttt{Preconstruct} oldLayers \\
			}
		}
		\uElseIf{r = PUT layer }
		{
				%\texttt{update} \emph{URLmap[(r.client, r.repo, r.layer)]} \\
				\texttt{update} \emph{RLmap[r.repo, r.layer]} \\
				\emph{cache} $\leftarrow$ \texttt{cache} \emph{r.layer} \\
				%\texttt{set} \emph{L\_PUT\_timer[r.layer]} \\
				%{\tiny\texttt{/*when $L\_timer[layer] > L\_thresh$, layer is evicted/}}	 \\
			}
		\ElseIf{r = GET layer}
		{
				\uIf{r.layer in cache}
				{
					\emph{serve from cache} \\
					\emph{hit++} \\
				} 
			   %{ 
			   	    {\tiny\texttt{/* Otherwise, it's a repull layer miss   /}}\\
					\uElseIf{ files in r.layers are cached in file cache}
					{
						layer $\gets$ \texttt{Restoring from} \emph{file cache + Disks} 
					}
					\Else{
						layer $\gets$ \texttt{Restoring from} \emph{Disks} \\
					%\uIf{ $D_{rs}$ $>$ $\theta_{rs}$} 
					%{ 
						\emph{cache} $\gets$ layer
					%}		
				}
			%}
		
		}
	\texttt{update} \emph{URLmap[r.client, r.repo, r.layer]} \\
	\texttt{set} \emph{Timer[r.client, r.repo, layer]} \\
	}
\end{algorithm}

%
\begin{algorithm}
	\scriptsize 
	\caption{Eviction}
	\label{alg:eviction}
	\KwIn{\\
		$\theta_{C}$: Capacity threshold for layer cache to trigger replacement. \\
		%$T_{flash}$: Capacity threshold for FLASH cache to trigger eviction. \\
		%$\theta_{nrpe}$: Threshold for non-repull layer to be evicted.  \\
		$LFRUlayer$: LFRU of layers. \\
		%$URLmap$: User to user profile map. \\
		$Timer$: Timer for layers in layer cache.
	}
	
	\ForEach{layer in LFRUlayer} 
	{
		{\tiny $/*$\textit{Timeout eviction   /}} \\
		\If{Timeout == Timer[layer]}
			{
				\texttt{Evict} layer \\
			}
	}

	\While{freeSpace $<$ $\theta_{C}$}
	{
		{\tiny $/*$\textit{Replacement   /}} \\
		layer $\gets$ \emph{LFRUlayer.last\_item()}\\

		\texttt{Evict} layer \\
		\emph{freeSpace} $+=$ \texttt{sizeof} layer \\

  }
\end{algorithm}		
%		if layer.rrp < x:
%		remove the layers
%		
%		\If{layer exclusively belongs to last\_usr}{
%			%{\scriptsize $/*$\textit{If layer is not shared with other users, layer is deleted}}\\
%			%	last\_layer $\gets$ \emph{LayerLRU[last\_usr]}\\			
%			FLASHcache $\gets$ \emph{Demote(last\_layer)} \\
%			\emph{free\_MEM} $+=$ \emph{sizeof(last\_layer)} \\
%		}
%	}
%	
%}

%	\ForEach{layer in cache} 
%	{
%		{\tiny $/*$\textit{Non-repull layer eviction/}} \\
%		and URL[clients, repos, layer].rp <
%	}
%	for layer in cache:
%			if  and 
%				evict layer
%	for repo in timeout repo:
%	if repo.rrp < x:
%	remove the layer in the repo whoes repo all timeout and repo.rrp < x.
%	evictedLayer <-
%	
%	for user in timeout user:
%	if user.rrp < x:
%	remove the layers
	
	


%	\While{free\_FLASH $<$ $T_{flash}$}{
%	\emph{last\_repo} $\gets$ \emph{RepoLRU.last\_item()}\\
%	\For{last\_layer $\gets$ \emph{LayerLRU[last\_repo].last\_item()}}{
%		\If{layer exclusively belongs to last\_repo}{
%			%{\scriptsize $/*$\textit{If layer is not shared with other users, layer is deleted}}\\
%			%	last\_layer $\gets$ \emph{LayerLRU[last\_usr]}\\			
%			\emph{Discard(last\_layer)} \\
%			\emph{free\_FLASH} $+=$ \emph{sizeof(last\_layer)} \\
%		}
%	}
%}




%\begin{algorithm}
%    \tiny 
%	\caption{User access history based eviction}
%	\label{alg:prefetch}
%	%\SetAlgoLined
%	\KwIn{\\
%		$L_{thresh}$: Threshold for duration to keep a prefetched layer. \\
%		$RLMap$: Repository to layers map.\\
%		$URLmap$: User to layers map. \\
%	}
%	\While{true}{
%		\emph{r} $\leftarrow$ \texttt{request received}\\
%		\uIf{r = GET manifest}
%		{
%%			layerlst $\leftarrow$ URLmap[(r.client, r.repo)]
%			\emph{layers} $\gets$ \emph{RLMap[r.repo]} $-$ \emph{URLmap[r.client]} \\
%			\emph{OnTimelayers}, \emph{NotOnTimelayers} $\gets$ \emph{OnTimeCalculation(layers)} \\
%			\emph{MEMcache} $\gets$ \emph{Prefetch(OnTimelayers)} \\
%			\emph{FLASHcache} $\gets$ \emph{Prefetch(NotOnTimelayers)} \\
%			\texttt{set} \emph{L\_timer[layer] for each layer in layer} \\
%			{\tiny\texttt{/*when $L\_timer[layer] > L\_thresh$, layer is evicted/}}
%			}
%		\uElseIf{r = PUT layer }
%		{
%				\texttt{update} \emph{URLmap[(r.client, r.layer)]} \\
%				\texttt{update} \emph{RLMap[(r.repo, r.layer, put)]} \\
%				\emph{MEMcache} $\leftarrow$ \texttt{buffer} \emph{r.layer} \\
%				\texttt{set} \emph{L\_PUT\_timer[r.layer]} \\
%				{\tiny\texttt{/*when $L\_timer[layer] > L\_thresh$, layer is evicted/}}	 \\
%			}
%		\ElseIf{r = GET layer}
%		{
%				\eIf{r.layer in MEMcache or r.layer in FLASHcache}
%				{
%					\emph{serve from MEMcache or FLASHcache} \\
%					\texttt{update} \emph{URLmap[(r.client, r.layer)]} \\
%					\texttt{Reset} \emph{L\_timer[r.layer]}\\
%					\emph{hit++} 
%				}
%			   {
%					\emph{serve from backend storage system} \\
%					\texttt{update} \emph{URLmap[(r.client, r.layer, repulled)]} \\
%					\emph{RepulledLayers} $\gets$ \emph{RLMap[r.repo]} \\
%					\emph{FLASHcache} $\gets$ \emph{Prefetch(RepulledLayers)} \\
%					\texttt{set} \emph{L\_timer[layer] for each layer in layers} \\
%					{\tiny\texttt{/*when $L\_timer[layer] > L\_thresh$, layer is evicted/}}
%			}
%		}
%	}
%\end{algorithm}

%U_thresh
%
%while true do
%	r <- request received
%		if r = Get manifest then
%			layerlst <- Urlmap[(r.client, r.repo)]
%			layers <- choose_popular(layerlst) 
%			PrefetchedLayers <- Prefetch(layers) 
%			set L_timer[layer] for each layer in layers
%			/when L_timer[layer] > L_thresh, layer is evicted/
%		if r = Put layer then
%			update Urlmap[(r.client, r.repo, r.layer)]
%			Layerbuffer <- buffer r.layer
%			set L_timer[r.layer] for r.layer
%			/when L_timer[layer] > L_thresh, layer is evicted/		
%		if r = Get layer then
%			if r.layer in PrefetchedLayers then
%				serve from PrefetchedLayers[r.layer]
%				update Urlmap[(r.client, r.repo, r.layer)]
%				prefetch_hit++
%			else
%				server from backend storage system
%				update Urlmap[(r.client, r.repo, r.layer)]
%				layerlst <- Urlmap[(r.client, r.repo)]
%				layers <- choose_popular(layerlst) 
%				PrefetchedLayers <- Prefetch(layers) and 
%				set L_timer[layer] for each layer in layers
%				/when L_timer[layer] > L_thresh, layer is evicted/


As shown in algorithm~\ref{alg:prefetch}, \sysname~maintains a tree-like URLMap for recording 
user access history information. For example, if usr \emph{U} accesses repository \emph{R}'s layer \emph{L},
we will add a child node \emph{L} to parent node \emph{R} which is a child node of \emph{U}.
Each node records the follow history information: (\emph{Pull\_cnt}, \emph{Put\_cnt}, \emph{last\_access\_time}). Note that for layer node, \emph{Put\_cnt} $=$ 1 or $=$ 0.
Whenever a GET or PUT layer request is received, \sysname~will update URLMap for the associate user, repository, and layer. 
When a GET manifest request is received or a GET layer request is miss,
\sysname~will select a certain number of layers from client's previously accessed layers by lookup URLMap
and prefetch them into cache for later access.
To determine if layer is a popular layer and need to be prefetched into cache,
we only consider the layers that was recently accessed with a time period 
that is shorter than \emph{T\_thresh}, meaning that
\emph{current\_time} $-$ \emph{last\_access\_time} $<$ \emph{T\_thresh}.
Then we calculate the layer popularity as: 
$\omega_{pull} \times Pull\_cnt + \omega_{push} \times Push\_cnt$.
Since recently \emph{pushed} layer has a higher chance to get \emph{pulled} in the future,
we give a higher weight ($\omega_{push} > \omega_{pull}$) for the layer if it has a \emph{Push\_cnt} equal to 1. 

 
%We assume that users are independent with each other.

