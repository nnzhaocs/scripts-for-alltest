\subsection{User-based cache algorithm}
\label{sec:design_cache_algori}

\begin{algorithm}
	%\scriptsize 
	\caption{User-based cache replacement algorithm.}
	\label{alg:cache}
	\SetAlgoLined
	\KwIn{
		$S_{thresh}$: Capacity threshold for layer buffer to trigger eviction.
	}
	\While{free\_buffer $<$ $S_{thresh}$}{
		$last\_usr \gets UsrLRU.last\_item()$\\
		\For{layer in reversed(LayerLRU.items())}{
			\If{layer exclusively belongs to last\_usr}{
				%{\scriptsize $/*$\textit{If layer is not shared with other users, layer is deleted}}\\
				\textbf{Evict} $LayerLRU[layer]$\\			
				$free\_buffer += sizeof(layer)$
		}
	}
	\textbf{Evict} $UsrLRU[last\_usr]$
	}
\end{algorithm}

We use our observations of user access pattern to guide our cache replacement
since user active time is more predictable based on Section~\ref{xxx}.
In~\cref{sec:design}, the incoming PUT layer requests are first buffered in layer buffer and later evicted to file cache.
If there is a PULL layer request miss in both layer buffer and file cache, 
\sysname~will fetch the layer from backend storage system and store it in layer buffer.
Correspondingly, cache pressure happens in both file cache and layer buffer caused by shortage of free space.
Layer buffer or file cache will simply evict or delete some layers or files and reclaim space.
Since our layer buffer and file cache both share the same cache replacement algorithm, we only present
how our user-based cache replacement algorithm work for layer buffer as shown in Algorithm~\ref{alg:cache}.

Free space in layer buffer is too low if ($free\_buffer$ $<$ $S_{thresh}$). 
\sysname~will reduce the buffer space being used by the inactive users. 
\sysname~maintains two LRU lists: a LRU list of active users \emph{UsrLRU} and
a LRU list of recently accessed layers \emph{LayerLRU} respectively.
At first, \sysname~will select the least active user from \emph{UsrLRU}.
Next, \sysname~will reversely iterate \emph{UsrLRU} til it finds a layer that exclusively owned by the inactive user, 
then evict the layer from layer buffer. 
 Note that layers are shared among different active users.
 \sysname~stops eviction til there is enough free space in layer buffer.
 
%iteratively removes the least active users along with the least recently accessed layers 
% that exclusively owned by these users from the two LRU lists and evict the layers from layer buffer.
%, then   


