\section{Slimmer design}
\label{sec:slimmer}

%Our analysis in Section~\ref{sec:redundant_files} suggests that adding the
%support of file-level deduplication to Docker registry can significantly reduce
%its storage capacity requirements, especially in large-scale deployments.
%
%In this section, we first describe a high-level design of \emph{\sysname}---
%a Docker registry that supports file-level deduplication.
%We then proceed with a simulation-based evaluation of the expected performance
%implications.

%Integrated Caching and Deduplication

%Cache-assisted Inline deduplication system

Our goal is to improve the restoring performance when we apply deduplication on registries' backend storage systems.
Modern container registries such like Google container registry~\cite{googleregistry} use cloud storage as backend Docker image storage systems. Users can push and pull Docker images to their repositories stored on Cloud storage. To enable fast and high-availability access, container registries use regional private repositories across the world, and users can store images close to their compute instances and get fast response time. For example, IBM container registry setup spans five regions~\cite{anwarfast}. 
There are many on-cloud global deduplication software for reducing cloud storage costs. For example, StorReduce sits between user application and cloud storage, transparently deduplicating data inline.
Intuitively, all the above deduplication techniques can be used to remove the redundant data from Docker image storage system.  
However, Docker image dataset is different from the common data stream. They are compressed archival files.
To remove the redundant files from compressed layer files, the deduplication methods should recognize the compression formats and decompress them before running the data through the block-level or file-level deduplication. Otherwise, the deduplication ratio would be very low since compressed files have very low deduplication ratio. 

In this section, we present the detailed design of \sysname.

\subsection{Design}
\label{sec:design}

We designed \sysname\ so that the interface between the Docker clients and the
registry remains unchanged.
%
As such, no modifications are needed in the Docker clients.
%
Below we describe the actions that \sysname\ takes during the layer pushes and
pulls at the registry side.
%
For the sake of this paper, we explain only the main steps omitting smaller
details.

\paragraph{Push}
%
\sysname\ does not unpack the layer immedieately after receiving it from a
client.
%
Instead, \sysname\ saves the layer's compressed tarball in a persistent
\emph{staging area}.
%
A separate \emph{off-line} deduplication process iterates over the layers in
the staging area and performs the following actions for every layer:
%
\textbf{1)}~uncompresses and unpacks the layer's tarball into individual files;
%
\textbf{2)}~computes a \emph{fingerprint} for every file in the layer;
%
\textbf{3)}~checks all file fingerprints against the \emph{file index} to
identify if identical files are already stored in \sysname;
%
\textbf{4)}~stores non-deduplicated files in \sysname's storage system;
%
\textbf{5)}~creates and stores a \emph{layer recipe} that includes the path,
metatada, and fingerprint of every file in the layer;
%
\textbf{6)}~removes the layer's tarball from the staging area.

We believe that off-line deduplication is appropriate because it keeps push
latencies percieved by the Docker clients low.
%
Background deduplication process can be scheduled during the periods of low
load on the registry.
%
Layer recipies are identified by layer's digests (Section~\ref{sec:background})
and files are identified by their fingeprints.
%
These identifiers are used to store and retrieve corresponding objects in the
underlying storage.
%
For example, if a file system is used as a backend storage, \sysname\ creates a
single file for  every layer recipe (named by the digest) and a single file for
every in-layer file (named by the fingerprint).


\paragraph{Pull}
%
Pulling the layer cannot be postponed to the off-line process becasue the
Docker client is actively waiting for the layer. 
%
\sysname\ performs the following actions \emph{inline} during the pull request:
%
\textbf{1)}~if a layer is still in the staging area, \sysname\ services it
directly form there;
%
\textbf{2)}~otherwise, \sysname\ finds the layer recipe by the layer digest
(provided by the client);
%
\textbf{3)}~prepares a directory structure for the layer based on the layer
recipe;
%
\textbf{4)}~packs and compresses the layer's  directory tree into a temporary
tarball;
%
\textbf{5)}~sends the layer tarball back to the client;
%
\textbf{5)}~disards the layer tarball.



\subsection{Operations}

\paragraph{Architecture}

\sysname~ is composed of five main components(see Figure~\ref{onshareddir}):
a modified Docker client,
layer buffer,
file cache,
registries,
backend storage system that implements deduplication.
The modified Docker client sends put or pull layer/manifest requests.
After it sends the pull layer requests.
it receives multiple partial layers and decompresses them together as a whole uncompressed layer, then verifies the
integrity of the uncompress layer by calculating the digest of the uncompress layer.
Layer buffer is used to buffer all the put layer requests and 
cache prefetched layers for later use.
Layer buffer can be implemented on distributed in-memory store,~\eg Plusma.
File cache is a flash cache that implements file-level deduplication which stores unique files that belong to a certain amount of layers.
File cache is implemented on distributed log structured store, \eg CORFU.
The unique files after removing duplicate files are flash-friendly workload because
the files can be appended as a log and there is no modification to these files, meaning that there is
no small write.
Backend storage system with deduplication feature stores the layers and does deduplication.
Many modern storage system withs deduplication feature can be used as our backend storage system.
However, the only requirement is that the storage systems can recognize compressed layer file and decompress them before running deduplication.
And the storage systems also can restore partial compressed layer file in parallel can send them back to clients. 
Note that there is no consistency issue between either layer buffer, file cache or backend storage because layers won't be changed in the future.

At a high-level, registries act like a distributed cache which sits close to clients to provide fast performance. 
Layer buffer caches hot layers for active users.
To improve the capacity limitation of main memory cache, we 
utilize flash fast random read access and store unique files on a distributed flash-based store.  

\paragraph{Workflow}
%
The registry API is almost the same as original registry.
The user simply pushes and pulles image to registry. 

\textbf{Push.}
After receiving a layer from the client, 
\sysname~first buffers the layer in layer buffer. 
Meantime, \sysname will also submitted this layer to the backend storage system.
Our layer buffer and file cache use write through policies 
and since there is no modification to the layer and files, so there is no data consistency issue between
cache and backend storage system.
Storing a layer in layer buffer might trigger cold layer eviction.
The cold layer will first be evicted to file cache and deduplicated.
A deduplication process performs the following steps for every victim layer that is evicted from layer buffer.
\begin{compactenumerate}
	\item decompress and unpack the layer's tarball into individual files;
	\item compute a \emph{fingerprint} for every file in the layer;
	\item check all file fingerprints against the \emph{file index} to
	identify if identical files are already stored in the file cache;
	\item store non-deduplicated files in \sysname's file cache and update the 
	\emph{file index} with file \emph{fingerprint} and file location along with its host address;
	\item create and store a \emph{layer recipe} that includes the file path,
	metadata, fingerprint of every file in the layer;
	\item remove the layer's tarball from the layer buffer.
\end{compactenumerate}
Layer recipes are identified by layer digests and files are identified by their fingerprints.
%
These identifiers are used to address corresponding objects in the
underlying flash storage.
Usually, the underlying distributed flash-based stores such as CORFU transparently spread data among
different servers and upper level applications are unaware of data host address. 
We modified CORFU's read and write interfaces so that the file locations and 
its host addresses are exposed to our ~\sysname.
 The metadata generated during the above deduplication process
 are stored on redis~\ref{xxx}.
 Similar to layer buffer, storing unique files in file cache might also trigger cold file eviction.
 Since the backend storage system already stores the backup of the layers, we can simply remove
 the victim files from file cache.
 The cache replacement is governed by the user-based cache algorithms
presents in Section~\ref{xxx}. 
%\sysname\ handles push requests asynchronously.
%\sysname 
%does not immediately unpack the layer.
%Instead, it reliably stores the layer's compressed tarball in a persistent
%\emph{staging area}.
%A separate \emph{off-line} deduplication process iterates over the layers in
%the staging area and performs the following steps for every layer:
%%
%\begin{compactenumerate}
%	\item decompress and unpack the layer's tarball into individual files;
%	\item compute a \emph{fingerprint} for every file in the layer;
%	\item check all file fingerprints against the \emph{file index} to
%	identify if identical files are already stored in \sysname;
%	\item store non-deduplicated files in \sysname's storage;
%	\item create and store a \emph{layer recipe} that includes the path,
%	metadata, and fingerprint of every file in the layer;
%	\item remove the layer's tarball from the staging area.
%\end{compactenumerate}

%The advantage of off-line deduplication is that it keeps push
%latencies perceived by the Docker clients low.
%
%The background deduplication process can be scheduled during the periods of low
%load on the registry.
%
%Layer recipes are identified by layer digests (see Section~\ref{sec:background})
%and files are identified by their fingerprints.
%%
%These identifiers are used to address corresponding objects in the
%underlying storage.
%
%For example, if a file system is used as a backend storage, \sysname\ creates a
%single file for every layer recipe (named by the digest) and a single file for
%every in-layer file (named by the fingerprint).

\paragraph{Pull}
%
A pull layer request that finds its desired layer in layer buffer 
is a hit in layer buffer. 
Otherwise, if it finds its containing files in file cache, that is
a hit in file cache.
%When a layer is pulled by the client, if \sysname\ 
Then, \sysname~has to \emph{reconstruct} the layer based
on the layer recipe.
%
%A pull request cannot be postponed to an off-line process as the
%pulling client is actively waiting for the layer.
%
\sysname\ performs the following steps \emph{inline} for restoring a layer from file cache:
%
\begin{compactenumerate}
	%\item check if the requested layer is still in the staging area and if so,
	%service it directly from there;
	\item find the layer recipe by the layer digest
	provided by the client;
	\item get a server list that contains subsets of this layer's files 
	by iterating each file \emph{fingerprint} in the layer recipe and getting its corresponding host address
	through \emph{file indexing}.
	\item forward the pull layer request and layer recipe to each server in the server list.
	\item initiate a restoring process on each server in the server list. 
	\item the restoring process first prepares a directory structure for the layer based on the layer recipe;
	\item the restoring process fetches the locally available files for the layer and put into the directory tree, 
	\item the restoring process packs and compresses the layer's directory tree into a temporary tarball;
	\item the restoring process sends the layer tarball back to the client and then discard the layer tarball.
\end{compactenumerate}


\subsection{User-based cache algorithm}
\label{sec:design_cache_algori}

\begin{algorithm}
	%\scriptsize 
	\caption{User-based cache replacement algorithm.}
	\label{alg:cache}
	\SetAlgoLined
	\KwIn{
		$S_{thresh}$: Threshold for layer buffer to trigger eviction.
	}
	\While{free\_buffer $<$ $S_{thresh}$}{
		$last\_usr \gets UsrLRU.last\_item()$\\
		\For{layer in reversed(LayerLRU.items())}{
			\If{layer exclusively belongs to last\_usr}{
				%{\scriptsize $/*$\textit{If layer is not shared with other users, layer is deleted}}\\
				\textbf{Evict} $LayerLRU[layer]$\\			
				$free\_buffer += sizeof(layer)$
		}
	}
	\textbf{Evict} $UsrLRU[last\_usr]$
	}
\end{algorithm}

We use our observations of user access pattern to guide our cache replacement
since user active time is more predictable based on Section~\ref{xxx}.
In~\cref{sec:design}, the incoming PUT layer requests are first buffered in layer buffer and later evicted to file cache.
If there is a PULL layer request miss in both layer buffer and file cache, 
\sysname~will fetch the layer from backend storage system and store it in layer buffer.
Correspondingly, cache pressure happens in both file cache and layer buffer caused by shortage of free space.
Layer buffer or file cache will simply evict or delete some layers or files and reclaim space.
Since our layer buffer and file cache both share the same cache replacement algorithm, we only present
how our user-based cache replacement algorithm work for layer buffer as shown in Algorithm~\ref{alg:cache}.

Free space in layer buffer is too low if ($free\_buffer$ $<$ $S_{thresh}$). 
\sysname~will reduce the buffer space being used by the inactive users. 
\sysname~maintains two LRU lists: a LRU list of active users \emph{UsrLRU} and
a LRU list of recently accessed layers \emph{LayerLRU} respectively.
At first, \sysname~will select the least active user from \emph{UsrLRU}.
Next, \sysname~will reversely iterate \emph{UsrLRU} til it finds a layer that exclusively owned by the inactive user, 
then evict the layer from layer buffer. 
 Note that layers are shared among different active users.
 \sysname~stops eviction til there is enough free space in layer buffer.
 
%iteratively removes the least active users along with the least recently accessed layers 
% that exclusively owned by these users from the two LRU lists and evict the layers from layer buffer.
%, then   




\subsection{Performance Evaluation}

While \sysname\ can effectively eliminate redundant files in the
Docker registry, it introduces overheads which can reduce the
registry's performance.
%
%The overheads can be classified in two categories: 1)~\emph{background
%overhead} caused by the computation and I/O that is performed during layer
%deduplication; and 2)~\emph{foreground overhead} from extra processing on the
%critical path of a pull request.

\begin{figure}
	\centering
	\includegraphics[width=0.48\textwidth]{graphs/res-time.pdf}
	\caption{Off-line file-level deduplication run time.}
	\label{fig:dedup-res}
\end{figure}


\paragraph{Simulation}
%
To analyze the impact of file-level deduplication on performance,
we conducted a preliminary simulation of \sysname.
%
%Based on the simulation results, we estimated the overhead of \sysname\ on
%\texttt{push} and \texttt{pull} layer request latencies.
%
%We then provide different suggestions on how the Docker registry can mitigate
%the deduplication overhead.
%
%%%%%%%%%%%%%%%%%%%%%%%%%%%%%%%%%%%%%%%%%%%%%%%%%%%%%%%%%%%%%%%%%%%%%%%%%%%%
%
%
Our simulation
approximates several \sysname's steps described in Section~\ref{sec:design}.
%
First, a layer from our dataset is copied to a RAM disk. 
%
%
%Note that there is no foreground pull or push requests since the simulation is \emph{off-line}.
%
The layer is then decompressed, unpacked, and the fingerprints of all files
are computed using the MD5 hash function~\cite{MD5}.
%
The simulation searches the fingerprint index for duplicates,
and, if the file has not been stored previously, it records the
file's fingerprint in the index.
%
%To map a layer to its containing files, we create the layer recipe and add it
%to a \emph{layer-to-file table}.
%
%The simulator then creates a file recipe.
%
%For each file in a layer, a layer digest
%to its containing file content digest mapping record is also created 
%
%The \emph{layer-to-file table} also
%records the file path within each layer associated with each file.
%
Note that at this point our simulation does not include
the latency of storing unique files.
%
To simulate layer's reconstruction during the \texttt{pull},
we archive and compress corresponding files.
%
%Only unique files are maintained in RAM
%disk while the redundant copies are removed.
%

We implemented the simulation in 600 lines of Python code
and setup a one-node Docker registry on a machine with 32~cores and 64GB of RAM.
%
To speed up the experiments and fit the required data in the RAM
we used 50\% of all layers and excluded the ones larger than 50MB.
%
We processed 60 layers in parallel using 60 threads.
%
The overall runtime of the simulation was roughly 3.5 days.
%
%The overall runtime is about 3.5 days.

Figure~\ref{fig:dedup-res} shows the CDF for each sub-operation of
\sysname.
%
Unpacking, Decompression, Digest Calculation, and Searching 
are part of
the deduplication process and together make up the Dedup time.
%
\VT{@Nannan, in Figure ~\ref{fig:dedup-res} can you reorder the lines in the
legend so that the Searching goes after Digest calculation?}
%
Searching, Archiving, and Compression
simulate the additional processing for a \texttt{pull}
request and form the Pulling time.
%

%\LR{What was the overall runtime for processing 0.9 million layers?}\NZ{addressed}
%
%\alicomment{How are we saving the location
%of each file in the layer? It is not clear from the following sentences.}
%\NZ{addressed}
%
%To improve searching performance, the
%mapping table is stored in Hive database~\cite{xxx}. 
%
%\lrcomment{Why are we using Hive for this? It seems overkill to me, especially
%for such small data. Even at scale, a KeyValue store would probably provide
%better performance than clunky MapReduce-based DB.}
%

\paragraph{Push}

\sysname\ does not directly impact the latency of \texttt{push} requests because
deduplication is performed asynchronously, \ie the registry reliably stores a
copy of the layer as-is and then sends a response to the client.
%
The appropriate performance metric for push is the time it takes to deduplicate
a single layer.
%
%Next, we look at the effects on \texttt{push} and \texttt{pull} latencies in
%more detail.
%
%However, if there are intensive push requests while the registry is performing
%deduplication, \sysname\ can still impact push latencies because it incurs
%CPU, memory, and I/O overhead. %(similar to pull requests).
%
Looking at the breakdown of the deduplication time in
Figure~\ref{fig:dedup-res}, we make several observations.
%
First, the searching time is the smallest among all operations with 90\% of the
searches completing in less than 4ms and the median of \VT{XXX}ms.
%
%The mapping table maintains 0.98 million layer-to-file digest mapping records. 
%
%\LR{Remove the following sentence? 1.7 million records is actually quite small
%so even a single-node DB with one index is enough.}\NZ{addressed} Consider
%that more than 1.7 million layers are stored in Docker hub and the number is
%still increasing, it's better to choose a fast distributed database to provide
%high searching performance and scalability.
%
Second, the calculation of digests spans a wide range from 5$\mu$s to almost
125s.
%
%This is because the time mainly depends on the layer size, \ie the fewer and
%smaller files a layer contains, the faster it is to compute all digests for
%the layer.
%
%Typically, smaller layers contain a smaller number of smaller files, which
%takes much less time to calculate their digests.
%
%While if the layer is bigger, the digest calculation overhead will be higher. 
%
90\% of digest calculation times are less than 27s while 50\% are
less than \VT{XXX}s.
%
The diversity in the timing is caused by a high variety of layer sizes both in
terms of storage space and file counts.
%
%Thus, we suggest that multiple-threading is needed to calculate the files'
%digests simultaneously; 
%
%Fast CPUs as well as more powerful computing nodes are required to speed up
%digest calculation.
%
Third, the run time for decompression and unpacking follows identical
distribution for approximately 60\% of the layers and is less than 150ms.
%
%Around 60\% of decompression and unpacking time are less than 0.15\,s. 
%
However, after that, the times diverge and compression times increase faster
compared to unpacking times.
%
\VT{do we have some theory why?}
%
90\% of decompressions take less than 950ms while 90\% of packing time is less
than 350ms.

%Overall, we see that file digest calculation contributes a lot to the
%overall deduplication latency especially when the layer size is big.  Moreover,
%we see that the deduplication latency increases as the layer size grows.
%
Overall, we see that 90\% of file-level deduplication time is less than \VT{XXX}s
per layer, while the average processing time for a single layer is \VT{XXX}.
%
This means that our single-node deployment can process about 3 layers/s on average.
%
In future we will work on further improving \sysname's deduplication throughput.
%
%In a large-scale registry deployment, this throughput can be improved
%as more node are available to perform deduplication.
%

\paragraph{Pull layer latency} 

For a \emph{pull layer} requests, \sysname\ will first fetch 
all its containing files by consulting the \emph{layer-to-file table}, 
restore the layer archive file and then send the compressed archive back
to the clients.

Figure~\ref{fig:dedup-res} shows the latency distributions for the compression
and archiving steps.
%
%Note that the \emph{pull layer} latency is the sum of archiving time,
%compression time, and searching time and does not include network transfer
%time.
%
%\LR{Always add a protected space or a \textbackslash, between a number and its unit.}
%\NZ{addressed}
%
We can see that around 55\% of the layers have a similar compression and archiving
time ranging from from 0.04\,s to 0.15\,s and both operations contribute equally
to pulling latency.
%60\% of compression and archiving time are less than 0.15 s.
%
%While compression has the highest run time 80\% of compression time is less than 2.82~s. 
%
\LR{Again, better to show the 90th percentile.}
\NZ{90\% of the compression time is less than 8\,s.}
After that, the times diverge and compression times increase faster with an
80\textsuperscript{ts} percentile of 2.82\,s.
%
Hence, compression time makes up the major portion of the pull latency and becomes a
bottleneck.
%
This is because compression times increase for larger layers and follow the distribution
of layer sizes (see Figure~\ref{fig:layer-size-cdf}).
%\LR{put reason here}
%\NZ{I think the layer size is the major reason}.
%
%We see that archiving time and compression contributes equally to pulling
%latency when their run time are lower than 0.15 s while compression time almost
%equals to pulling latency when the compression time is greater than 0.15 s. 

\LR{Should we move that to 4.3?}
%
\NZ{Can we use it as conclusion}
To reduce latency, we suggest that fast compression methods should be applied to reduce
compression times. As deduplication provides significant storage savings, faster compression
methods with a lower compression ratio are hence feasible.

%\alicomment{Separately mention pull layer requests}
%

%\paragraph{File-level deduplication run time}


\section{Discussion}

We propose additional optimizations that can help to speed up \sysname:

\begin{compactenumerate}
%
\item
%
As the majority of the pull time is caused by compression, we propose to cache
hot layers as precompressed tar files in the staging area.
%
%We observe that only a small proportion of images and layers are frequently
%requested and majority of images and layers are \textit{cold}.
%
%Figure~\ref{fig:pull-cnt} shows the total number of pulls from the time an
%image/layer has been stored in Docker Hub until May 30, 2017.
%
According to our statistics, only 10\% of all images were pulled
from Docker Hub more than 360 times from the time the image was first pushed to Docker Hub
until May 30, 2017. Moreover, we found that 90\% of pulls
went to only 0.25\% of images based on image pull counts.
%
This suggests the existence of both cold and hot images and layers.
%
%\VT{Nannan, can we instead compute that 90\% of pulls
%wen to 0.25\% of images?}
%
%
%translates to only 10\% of layers being pulled more than 660 times (at most).
%
%Note that we calculate the layer pull count shown in Figure~\ref{fig:pull-cnt}
%by aggregating the pull count of all images, which refer to this layer.
%
%Note that the image pull counts are crawled from Docker Hub website.
%
%Actual layer pull counts should be less because pulling an image does not
%necessarily pull all its containing layers if some layer have been previously
%downloaded and are already available locally.
%

\item
%
As deduplication provides significant storage savings, \sysname\ can use faster
but less effective local compression methods than gzip~\cite{lz4}.
%
%\VT{cite a few}

\item
%
%Deduplicating when workload is light As shown above, file-level deduplication
%comes with some performance overhead.
%
The registries often experience fluctuation in load with peaks and
troughs~\cite{dockerworkload}.
%
Thus, file-level deduplication can be triggered
when the load is low to prevent interference with 
client \texttt{pull} and \texttt{push} requests.
%
%To further improve the performance of \sysname\, we also suggest to use main
%memory for temporarily storing and processing \textit{small} layers.
%
%According to our findings (see~\S\ref{sec:dedup_ratio}), the majority of
%layers~(87.3\%) are smaller than 50\,MB and hence can be stored and processed
%in RAM to speed up deduplication. 
%
\end{compactenumerate}

%\begin{figure}
	\centering
	\includegraphics[width=0.4\textwidth]{graphs/pull-cnt.pdf}
	\caption{CDF of layer \& image pull count.
	}
	\label{fig:pull-cnt}
\end{figure}

%=======================================
%|             OLD VERSION              |
%=======================================

%\paragraph{Latency distribution for each operation}
%\subsubsection{When to start file-level dedup?} 

%\paragraph{Latency distribution for each operation}

%\paragraph{Small compression ratio and small layer size}
%
%\begin{figure}[!t]
	\centering
	\subfigure[CDF of compression ratio]{\label{fig_cdf_compression_ratio}
		\includegraphics[width=0.23\textwidth]{graphs/cdf_compression_ratio.pdf}
	}
	\subfigure[Histogram of comp. ratios]{\label{fig_his_compression_ratio}
		\includegraphics[width=0.223\textwidth]{graphs/his_compression_ratio.pdf}
	}
	\caption{Layer compression ratio distribution
	\vcomment{Different colors are used in figure (a) and (b) FLS/CLS}
	}
	\label{fig-compression-ratio}
\end{figure}

%
%\begin{figure}[!t]
	\centering
	\subfigure[CDF of layer sizes]{\label{fig_layer_size_cdf}
		\includegraphics[width=0.234\textwidth]{graphs/layer_size_mb.pdf}
	}
	\subfigure[Histogram of layer sizes]{\label{fig_hist_layer_size}
		\includegraphics[width=0.213\textwidth]{graphs/hist_layer_size.pdf}
	}
	\caption{Layer size distribution
%	\vcomment{Let's use CLS, ALS, and FLS abreviations\nancomment{addressed}}.
%	\vcomment{CLS size should go first}.
%	\vcomment{We need to use different types of lines (solid, dotted, etc.)
%		or markers (round, triangular)}.
%	\vcomment{In figure B it is not clear to which bar group corresponds
%		  to which layer size. I suggest to try to rotate the graph
%		  by 90 grads to fit all layer size labels.\nancomment{aligned label with bar}}
	}
	\label{fig-layer-size}
\end{figure}

%
%We found that most layers'compression ratio is really lower (?) while most of layers have a smaller size. 
%So how about we use archiving instead of compression if the network speed is higher (?GB/s)?

%\paragraph{Network transfer speed is high!}

%\subsubsection{File-level content addressable storage for cold layers}

%\begin{figure}
%	\centering
%	\includegraphics [width=0.45\textwidth]{plots/exp-total-stev-erase.eps}
%	\subfigure[]{\label{fig:per_layer_ratio_fcnt_cdf}
%		\includegraphics [width=0.23\textwidth]{graphs/}
%	}
%	\subfigure[Similar layer dedup]{\label{fig:per_layer_ratio_fcnt_pdf}
%		\includegraphics [width=0.22\textwidth]{graphs/graph_reconstruct_layers.pdf}
%	}
%	\caption{File-level content addressable storage model}
%	\label{fig:eval-stdev-erasure-cnt}
%\end{figure}

%\subsection{Hints for performance improvement and storage saving}

%\begin{table} 
%	\centering 
%	\scriptsize  
%	%\begin{minipage}{.5\linewidth}
%	\caption{Latency breakdown} \label{tbl:latency_breakdown} 
%	\begin{tabular}{|l|l|l|l|l|}%p{0.14\textwidth} 
%		\hline 
%		% after \\: \hline or \cline{col1-col2} \cline{col3-col4} ... 
%		% after \\: \hline or \cline{col1-col2} \cline{col3-col4} ... 
%		Operations/latency (S) & max & min & median & avg.\\
%		\hline
%		 gunzip decompression (RAM) & 257.16  & 0.04  & 0.15  & 0.39 \\
% 		\hline
% 		tar extraction (RAM) & 43.41  & 0.04  &  0.14  & 0.18 \\
%		\hline
%		Digest calculation (RAM) & 3455.01  & $<$0.00  & 0.05 & 10.65 \\
%		\hline
%		tar archiving (RAM)  & 53.44 & 0.04 & 0.14 & 0.19\\
%		\hline
%		gzip compression (RAM) & 496.04 & 0.04 & 0.15 & 2.10 \\
%%		\hline
%%		Total time (RAM) (with compression) & & & & \\
%%		\hline
%%		Total time (RAM) (without compression) & & & & \\
%		\hline
% 		\hline
% 		gunzip decompression (SSD) &   &   &    &  \\
% 		\hline
% 		tar extraction (SSD) &   &   &    &  \\
%		\hline
%		Digest calculation (SSD) &  &  & & \\
%		\hline
%		tar archiving (SSD) &  &  & & \\
%		\hline
%		gzip compression (SSD) & &  &  & \\
%%		\hline		 
%%		Total time (SSD) (with compression) & & & & \\
%%		\hline
%%		Total time (SSD) (without compression) & & & & \\
%		\hline
%		\hline
%		Network transfer & 20587.94 & $<$ 0.00 & $<$ 0.00 & 1.20 \\
%		\hline 	
%	\end{tabular} 
%\end{table}


%\begin{table} 
%	\centering 
%	\scriptsize  
%	%\begin{minipage}{.5\linewidth}
%	\caption{Summary of layer \& image characterization} \label{tbl:redundant_ratio} 
%	\begin{tabular}{|l|l|l|l|l|}%p{0.14\textwidth} 
%		\hline 
%		% after \\: \hline or \cline{col1-col2} \cline{col3-col4} ... 
%		% after \\: \hline or \cline{col1-col2} \cline{col3-col4} ... 
%		Metrics & max & min & median & avg.\\
%		\hline
%		Compressed layer size &   &   &   &  \\
%		\hline
%		Uncompressed layer size &   &   &    &  \\
%		\hline
%		Archival size &  &  & & \\
%		\hline
%		Compression ratio &   &   &    &  \\
%		\hline
%		Layer pull cnt. &  &  & & \\
%		\hline
%		File cnt. per layer &  &  & & \\
%		\hline
%		Dir. cnt. per layer &  &  & & \\
%		\hline
%		Layer depth &  &  & & \\
%		\hline
%		\hline
%		Compressed image size &  &  & & \\
%		\hline
%		Uncompressed image size & &  &  & \\
%		\hline
%		Archival image size & &  &  & \\
%		\hline
%		Compression ratio &   &   &    &  \\
%		\hline
%		Image pull cnt.  &  &  & & \\
%		\hline
%		Layer cnt. per image  &  &  & & \\
%		\hline
%		Shared layer cnt. per image  &  &  & & \\
%		\hline
%		File cnt. per layer &  &  & & \\
%		\hline
%		Dir. cnt. per layer &  &  & & \\
%		\hline	
%	\end{tabular} 
%\end{table} 

%\subsection{Constructing shared layers for redundant directories/files}
%
%\paragraph{Smaller number of layers are shared among different images}
%\begin{figure}[!t]
	\centering
	\subfigure[CDF of layer by layer count]{\label{fig_repeate_layer}
		\includegraphics[width=0.23\textwidth]{graphs/repeate_layer.pdf}
	}
	\subfigure[Histogram of images by layer count in images]{\label{fig_hist_repeate_layer}
		\includegraphics[width=0.223\textwidth]{graphs/hist_repeate_layer.pdf}
	}
	\caption{Compression rate distribution}
	\label{fig-repeat-layer-cnt}
\end{figure}
%
%\paragraph{Smaller pull latency than recompression model} the registry can prepare the reconstructed layers before users issue a pull request. But this model requires users to rebuild two layers.

%\subsubsection{Summary of Suggestions/trade-offs between dedup ratio and recompression overhead}
%
%\paragraph{1. using archiving instead of compression}
%\paragraph{2. using file-level dedup for cold images/layers}
%\paragraph{3. using file-level dedup economically}
%When to trigger file-level dedup?
%\paragraph{4. constructing shared layers for redundant dirs/files, for example,}
%%\subsection{Layer reconstruction model}
%%\subsubsection{Reconstruction overhead}
%%\subsubsection{Trade-offs between dedup ratio and reconstruction overhead}
%%\paragraph{Dedup ratio VS. Rebuild overhead}
%%\subsection{Evaluation results}

