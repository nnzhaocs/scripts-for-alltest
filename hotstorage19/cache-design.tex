
\subsection{User behavior based layer preconstruction cache}
\label{sec:cache-design}
In the following, we present our \sysname~\preconstructcachename design.
\subsubsection{User access patterns}
%\label{sec:design_cache_algori}

Docker client stores images as lists of layers and layers are shared among different repositories, which is similar to Docker registry.
When a user \emph{pulls} an image from a repository, 
it will first \texttt{pull} the manifest of the image~\cite{docker}~\cite{dockerworkload} and 
parse the manifest to get the layer digests,
then lookup each layer digest against a \emph{local layer digest index}.
After that it only \emph{pulls} the layers that has \emph{not been stored locally}.
Theoretically, users only pull layer once. 
However, some users might delete some local images and \emph{repull} layers for these images.
%Moreover, kubernetes allows users always \emph{repull} layers no matter these layers locally available or not~\cite{docker}.  
Here, a \emph{repull layer} means users pull this layer multiple times,
and a \emph{non-repull layer} means users only pull this layer once.

Figure~\ref{xxx} shows the CDF of layer repull count.
We see that majority of users don't \emph{repull} layers frequently.
For \texttt{Syd}, only 4\% of layers are repulled by the same clients.
\texttt{Dev} has the highest repull layer ratio of 36\% while 83\% of the repull layers are only repulled twice.
Majority of layers are repulled infrequently.
For example, only 3\% of layers from \texttt{Syd} are repulled more than twice.
Layer from \texttt{Prestage} and \texttt{Lon} have the highest repull frequency.
5\% of layers are pulled more than 6 times.
We also observe that few clients \emph{repull} layers continuously.
The highest layer repull count is 19,300 from \texttt{Lon}.
We think these clients probably deploy containers on a shared platform such as Cloud,
and run ephemeral jobs such as stateless microservices. 
Once the applications are finished, the container images are automatically deleted.

When different clients \texttt{pull} the same repository, 
they will fetch different amount of layers from the repository based on the availability of their local layer dataset.
Even the same clients \texttt{pull} the same repository at different times, 
they will fetch different amount of layers from the repository because their local layer dataset changes over time.
Therefore, a \texttt{pull manifest} requests doesn't usually result in pulling/repulling the layers in the repository. 
Here, we define \emph{repulling repository} as 
repulling the layer in the repository for the same client.
Figure~\ref{xxx} shows the CDF of the probability of repository repulling.
The probability of repository repulling is calculated 
as the number of \texttt{pull manifest} resulting in repository repulling divided by 
the total number of \texttt{pull manifest} requests issued by the same client for the same repository.
We see that majority of repositories aren't repulled.
The repull repository ratio ranges from 15\% for \texttt{Prestage} to 43\% for \texttt{Prestage}.
Majority of repull repositories have a low repulling probability.
Only 20\% of repositories from  \texttt{Prestage}, \texttt{Stage}, and 
\texttt{Syd} have a repulling probability higher than 0.5.
And only 20\% of repositories from the rest 4 workloads have a repulling probability higher than 0.33.
We also observe that few repositories' repulling probability are 1, meaning 
every time clients pull these repositories, they always repull the layers in these repositories. 
 
Figure~\ref{xxx} shows the client repulling probability.
Client repulling probability is calculated as the number of \emph{repull} layer requests divided by
the number of \texttt{pull} layer requests issued by the same client.
We see that majority of clients do repull layers but the probability is low.
60\% of clients from \texttt{Prestage}, \texttt{Dev}, \texttt{Lon}, and \texttt{Fra} have a repulling probability lower than 0.1.
55\% of clients from both \texttt{Dal} and \texttt{Stage} have a repulling probability lower than 0.1.
Less clients have repulling probability range between 0.1 to 0.7.
10\%-30\% of clients have a repulling probability ranged from 0.1-0.7 across 7 workloads.
We find few clients repull layers continuously.
2\%-12\% of clients have a repulling probability higher than 0.9 from workloads:
\texttt{Dal}, \texttt{Dev}, \texttt{Fra}, \texttt{Prestage},
\texttt{Stage}, \texttt{Syd}, and \texttt{Lon}.

\paragraph{User profiles}
client repull ratio
repo repull ratio
Based on the above 
%behavior 
pattern and hierarchy, we can record the users' repository and layer access history. 
Theoretically, once a user issues a \texttt{pull} manifest from a repository, all the layers that belong to this repository but have not been \emph{pulled} by this user should be prefetched into the cache.
%when a new use connects to registry, we can prefetch all the layers that have been accessed by this user into cache ideally.
In this case, the cache hit ratio will reach 1.
%However, because there is a limit to the cache size, %we won't be able to fit all the active users' layers into cache.
%not all active user's layers will fit into the cache.
%%But we can prefetch active users' popular layers in the cache based on the historical access information.
%To mitigate this, we can only prefetch the active user's top requested layers based on the historical access information.

As shown in algorithm~\ref{alg:prefetch}, \sysname maintains two maps: a RLMap for recording layer-repository
relationship, and 
a URLMap for recording 
users' repository and layer access history information. 
For example, if a user~\emph{U} \emph{pulls} a layer~\emph{L} from a repository~\emph{R},
\sysname will add an new entry (\emph{U,L}) in URLMap.
%Each node records the following history information: (\emph{Get\_cnt}, \emph{Put\_cnt}, \emph{last\_access\_time}). a child node layer~\emph{L} to parent node~\emph{R}
While if a user~\emph{U} \emph{pushes} a  layer~\emph{L} to a repository~\emph{R},
\sysname~will add an new entry (\emph{R,L}) in RLMap. 
Note that to identify which layers are locally available for a user, 
we extract \emph{user end host address} (\emph{r.client}) 
%as shown in Algorithm~\ref{alg:prefetch}
from each request and define the user end host address as user,
and keep track of all layers that have been downloaded by \emph{r.client}. 
%Note that for a layer node, \emph{Put\_cnt} $=$ 1 or $=$ 0.
%When a GET or PUT layer request is received, \sysname~will update the URLMap of the associating user, repository, and layer. 
When either a GET manifest request is received or 
%a GET layer request is miss,
%a requested layer is not cached or prefetched  (
a miss on a GET layer request happens,
\sysname will lookup RLMap and get the requested repository's containing layers,
and compare against the layers that are already \emph{pulled} by the user by looking up URLMap,
%select a certain number of \emph{popular} layers from the client's 
%previously accessed layers by lookup 
%URLMap
then prefetch the layers that have not been \emph{pulled}. 
We set a timer for each cached layer and evict it when its timer is $>U_{thresh}$.
To incorporate the algorithm with \sysname,
we prefetch \texttt{slices} in parallel from backend servers,  
%dedup storage system,
buffer them in the layer buffer first, then evict them into the file cache after they \emph{cool down}.

\subsubsection{Temporal trend}

Figure~\ref{xxx} shows the CDF of layer popularity.
We observe a heavy layer access skewness for \texttt{Fra}, \texttt{Syd}, \texttt{Dal}, \texttt{Stage}, 
and \texttt{Lon}.
We see that 80\%, 70\%, and 60\% of the \texttt{pull} layer requests access only 10\% of layers, 
for \texttt{Fra}, \texttt{Syd}, \texttt{Dal}, \texttt{Stage}, 
and \texttt{Lon}.
Figure~\ref{xxx} shows the CDF of repository popularity.
Compare to layer popularity, 
repository access skewness is heavier acorss 7 workloads.
~90\% of \texttt{pull} layer requests access only 10\% of repositories for 
\texttt{Dev}, \texttt{Fra}, \texttt{Prestage}, \texttt{Syd}, and \texttt{Stage}.
~75\% of \texttt{pull} layer requests access only 10\% of repositories for
both \texttt{Dal} and \texttt{Lon}.
Figure~\ref{xxx} shows the CDF of client popularity.
\texttt{Dal}, \texttt{Dev}, \texttt{Fra}, \texttt{Lon}, \texttt{Prestage}, and \texttt{Stage}
shows a heavey client access skewness.
10\% of clients send 95\% \texttt{pull} layer requests for \texttt{Lon}.
\texttt{Syd} shows a slight client skewness. 70\% of requests are sent by 36\% of clients.

Figure~\ref{xxx} shows the CDF of layer reuse time. 
Layer reuse time means the duration between two consecutive \texttt{pull} requests to the same layer.
We see that layer reuse time distribution varies among different workloads.
For \texttt{Fra}, \texttt{Syd}, and \texttt{Stage},
half of the layers' reuse time is shorter than 6 minutes.
While half of layers from \texttt{Dal} and \texttt{Lon} have a reuse time higher than 1 hour.
Half of layers from both \texttt{Prestage} and \texttt{Dev} are not accessed for over 100 hours.

Figure~\ref{xxx} shows the CDF of repository reuse time.
Repository reuse time means the duration between two consecutive \texttt{pull manifest} requests to the same repository.
We find that repository reuse time is much shorter than layer reuse time.
80\% of repositories are re-accessed within 2-12 minutes for the 7 workloads.

Figure~\ref{xxx} shows the CDF of client access intervals.
Client access intervals are much shorter than repository reuse time.
80\% of client are active for at least 1 - 3 minutes for the 7 workloads. 
   

\paragraph{User "Pull manifest" request as an indicator}
\begin{algorithm}
\scriptsize 
	\caption{Preconstruction}
	\label{alg:prefetch}
	\KwIn{\\
		%$\theta_{rp}$: Threshold for popular repulling clients. \\
		%{\tiny\texttt{/*when $L\_timer[layer] > L\_thresh$, layer is evicted or demoted to Flashcache */}}\\
		$\theta_{rpc}$: Threshold for repull layers to be preconstructed and cached. \\
		%$\theta_{R}$: Threshold for popular repulling layers. \\
		$RLmap$: Repository to layers map.\\
		$URLmap$: User to user profile map. \\
	}
	\While{true}{
		\emph{r} $\leftarrow$ \texttt{request received}\\
		\uIf{r = GET manifest}
		{
			newLayers $\gets$ \emph{RLmap[r.repo]} $-$ \emph{URLmap[r.client].layers} \\
			oldLayers $\gets$ \emph{RLmap[r.repo]} $\bigcap$ \emph{URLmap[r.client].layers} \\
			%\emph{layers} $\gets$ \emph{RLmap[r.repo]} $-$ \emph{URLmap[r.client]} \\
			%\emph{OnTimelayers}, \emph{NotOnTimelayers} $\gets$ \emph{OnTimeCalculation(layers)} \\
			%\emph{MEMcache} $\gets$ \emph{Prefetch(OnTimelayers)} \\
			%\emph{FLASHcache} $\gets$ \emph{Prefetch(NotOnTimelayers)} \\
			%\texttt{set} \emph{L\_timer[layer] for each layer in layers} \\
			\emph{cache} $\leftarrow$ \texttt{Preconstruct} newLayers \\
			\If{URLmap[r.client, r.repo, oldLayers].rp $>$ $\theta_{rpc}$} 
			{ 
				\emph{cache} $\leftarrow$ \texttt{Preconstruct} oldLayers \\
			}
		}
		\uElseIf{r = PUT layer }
		{
				%\texttt{update} \emph{URLmap[(r.client, r.repo, r.layer)]} \\
				\texttt{update} \emph{RLmap[r.repo, r.layer]} \\
				\emph{cache} $\leftarrow$ \texttt{cache} \emph{r.layer} \\
				%\texttt{set} \emph{L\_PUT\_timer[r.layer]} \\
				%{\tiny\texttt{/*when $L\_timer[layer] > L\_thresh$, layer is evicted/}}	 \\
			}
		\ElseIf{r = GET layer}
		{
				\uIf{r.layer in cache}
				{
					\emph{serve from cache} \\
					\emph{hit++} \\
				} 
			   %{ 
			   	    {\tiny\texttt{/* Otherwise, it's a repull layer miss   /}}\\
					\uElseIf{ files in r.layers are cached in file cache}
					{
						layer $\gets$ \texttt{Restoring from} \emph{file cache + Disks} 
					}
					\Else{
						layer $\gets$ \texttt{Restoring from} \emph{Disks} \\
					%\uIf{ $D_{rs}$ $>$ $\theta_{rs}$} 
					%{ 
						\emph{cache} $\gets$ layer
					%}		
				}
			%}
		
		}
	\texttt{update} \emph{URLmap[r.client, r.repo, r.layer]} \\
	\texttt{set} \emph{Timer[r.client, r.repo, layer]} \\
	}
\end{algorithm}

%
\begin{algorithm}
	\scriptsize 
	\caption{Eviction}
	\label{alg:eviction}
	\KwIn{\\
		$\theta_{C}$: Capacity threshold for layer cache to trigger replacement. \\
		%$T_{flash}$: Capacity threshold for FLASH cache to trigger eviction. \\
		%$\theta_{nrpe}$: Threshold for non-repull layer to be evicted.  \\
		$LFRUlayer$: LFRU of layers. \\
		%$URLmap$: User to user profile map. \\
		$Timer$: Timer for layers in layer cache.
	}
	
	\ForEach{layer in LFRUlayer} 
	{
		{\tiny $/*$\textit{Timeout eviction   /}} \\
		\If{Timeout == Timer[layer]}
			{
				\texttt{Evict} layer \\
			}
	}

	\While{freeSpace $<$ $\theta_{C}$}
	{
		{\tiny $/*$\textit{Replacement   /}} \\
		layer $\gets$ \emph{LFRUlayer.last\_item()}\\

		\texttt{Evict} layer \\
		\emph{freeSpace} $+=$ \texttt{sizeof} layer \\

  }
\end{algorithm}		
%		if layer.rrp < x:
%		remove the layers
%		
%		\If{layer exclusively belongs to last\_usr}{
%			%{\scriptsize $/*$\textit{If layer is not shared with other users, layer is deleted}}\\
%			%	last\_layer $\gets$ \emph{LayerLRU[last\_usr]}\\			
%			FLASHcache $\gets$ \emph{Demote(last\_layer)} \\
%			\emph{free\_MEM} $+=$ \emph{sizeof(last\_layer)} \\
%		}
%	}
%	
%}

%	\ForEach{layer in cache} 
%	{
%		{\tiny $/*$\textit{Non-repull layer eviction/}} \\
%		and URL[clients, repos, layer].rp <
%	}
%	for layer in cache:
%			if  and 
%				evict layer
%	for repo in timeout repo:
%	if repo.rrp < x:
%	remove the layer in the repo whoes repo all timeout and repo.rrp < x.
%	evictedLayer <-
%	
%	for user in timeout user:
%	if user.rrp < x:
%	remove the layers
	
	


%	\While{free\_FLASH $<$ $T_{flash}$}{
%	\emph{last\_repo} $\gets$ \emph{RepoLRU.last\_item()}\\
%	\For{last\_layer $\gets$ \emph{LayerLRU[last\_repo].last\_item()}}{
%		\If{layer exclusively belongs to last\_repo}{
%			%{\scriptsize $/*$\textit{If layer is not shared with other users, layer is deleted}}\\
%			%	last\_layer $\gets$ \emph{LayerLRU[last\_usr]}\\			
%			\emph{Discard(last\_layer)} \\
%			\emph{free\_FLASH} $+=$ \emph{sizeof(last\_layer)} \\
%		}
%	}
%}




%\begin{algorithm}
%    \tiny 
%	\caption{User access history based eviction}
%	\label{alg:prefetch}
%	%\SetAlgoLined
%	\KwIn{\\
%		$L_{thresh}$: Threshold for duration to keep a prefetched layer. \\
%		$RLMap$: Repository to layers map.\\
%		$URLmap$: User to layers map. \\
%	}
%	\While{true}{
%		\emph{r} $\leftarrow$ \texttt{request received}\\
%		\uIf{r = GET manifest}
%		{
%%			layerlst $\leftarrow$ URLmap[(r.client, r.repo)]
%			\emph{layers} $\gets$ \emph{RLMap[r.repo]} $-$ \emph{URLmap[r.client]} \\
%			\emph{OnTimelayers}, \emph{NotOnTimelayers} $\gets$ \emph{OnTimeCalculation(layers)} \\
%			\emph{MEMcache} $\gets$ \emph{Prefetch(OnTimelayers)} \\
%			\emph{FLASHcache} $\gets$ \emph{Prefetch(NotOnTimelayers)} \\
%			\texttt{set} \emph{L\_timer[layer] for each layer in layer} \\
%			{\tiny\texttt{/*when $L\_timer[layer] > L\_thresh$, layer is evicted/}}
%			}
%		\uElseIf{r = PUT layer }
%		{
%				\texttt{update} \emph{URLmap[(r.client, r.layer)]} \\
%				\texttt{update} \emph{RLMap[(r.repo, r.layer, put)]} \\
%				\emph{MEMcache} $\leftarrow$ \texttt{buffer} \emph{r.layer} \\
%				\texttt{set} \emph{L\_PUT\_timer[r.layer]} \\
%				{\tiny\texttt{/*when $L\_timer[layer] > L\_thresh$, layer is evicted/}}	 \\
%			}
%		\ElseIf{r = GET layer}
%		{
%				\eIf{r.layer in MEMcache or r.layer in FLASHcache}
%				{
%					\emph{serve from MEMcache or FLASHcache} \\
%					\texttt{update} \emph{URLmap[(r.client, r.layer)]} \\
%					\texttt{Reset} \emph{L\_timer[r.layer]}\\
%					\emph{hit++} 
%				}
%			   {
%					\emph{serve from backend storage system} \\
%					\texttt{update} \emph{URLmap[(r.client, r.layer, repulled)]} \\
%					\emph{RepulledLayers} $\gets$ \emph{RLMap[r.repo]} \\
%					\emph{FLASHcache} $\gets$ \emph{Prefetch(RepulledLayers)} \\
%					\texttt{set} \emph{L\_timer[layer] for each layer in layers} \\
%					{\tiny\texttt{/*when $L\_timer[layer] > L\_thresh$, layer is evicted/}}
%			}
%		}
%	}
%\end{algorithm}

%U_thresh
%
%while true do
%	r <- request received
%		if r = Get manifest then
%			layerlst <- Urlmap[(r.client, r.repo)]
%			layers <- choose_popular(layerlst) 
%			PrefetchedLayers <- Prefetch(layers) 
%			set L_timer[layer] for each layer in layers
%			/when L_timer[layer] > L_thresh, layer is evicted/
%		if r = Put layer then
%			update Urlmap[(r.client, r.repo, r.layer)]
%			Layerbuffer <- buffer r.layer
%			set L_timer[r.layer] for r.layer
%			/when L_timer[layer] > L_thresh, layer is evicted/		
%		if r = Get layer then
%			if r.layer in PrefetchedLayers then
%				serve from PrefetchedLayers[r.layer]
%				update Urlmap[(r.client, r.repo, r.layer)]
%				prefetch_hit++
%			else
%				server from backend storage system
%				update Urlmap[(r.client, r.repo, r.layer)]
%				layerlst <- Urlmap[(r.client, r.repo)]
%				layers <- choose_popular(layerlst) 
%				PrefetchedLayers <- Prefetch(layers) and 
%				set L_timer[layer] for each layer in layers
%				/when L_timer[layer] > L_thresh, layer is evicted/


\subsubsection{Layer preconstruction}



\subsubsection{Layer eviction}

\begin{algorithm}
	\scriptsize 
	\caption{Eviction}
	\label{alg:eviction}
	\KwIn{\\
		$\theta_{C}$: Capacity threshold for layer cache to trigger replacement. \\
		%$T_{flash}$: Capacity threshold for FLASH cache to trigger eviction. \\
		%$\theta_{nrpe}$: Threshold for non-repull layer to be evicted.  \\
		$LFRUlayer$: LFRU of layers. \\
		%$URLmap$: User to user profile map. \\
		$Timer$: Timer for layers in layer cache.
	}
	
	\ForEach{layer in LFRUlayer} 
	{
		{\tiny $/*$\textit{Timeout eviction   /}} \\
		\If{Timeout == Timer[layer]}
			{
				\texttt{Evict} layer \\
			}
	}

	\While{freeSpace $<$ $\theta_{C}$}
	{
		{\tiny $/*$\textit{Replacement   /}} \\
		layer $\gets$ \emph{LFRUlayer.last\_item()}\\

		\texttt{Evict} layer \\
		\emph{freeSpace} $+=$ \texttt{sizeof} layer \\

  }
\end{algorithm}		
%		if layer.rrp < x:
%		remove the layers
%		
%		\If{layer exclusively belongs to last\_usr}{
%			%{\scriptsize $/*$\textit{If layer is not shared with other users, layer is deleted}}\\
%			%	last\_layer $\gets$ \emph{LayerLRU[last\_usr]}\\			
%			FLASHcache $\gets$ \emph{Demote(last\_layer)} \\
%			\emph{free\_MEM} $+=$ \emph{sizeof(last\_layer)} \\
%		}
%	}
%	
%}

%	\ForEach{layer in cache} 
%	{
%		{\tiny $/*$\textit{Non-repull layer eviction/}} \\
%		and URL[clients, repos, layer].rp <
%	}
%	for layer in cache:
%			if  and 
%				evict layer
%	for repo in timeout repo:
%	if repo.rrp < x:
%	remove the layer in the repo whoes repo all timeout and repo.rrp < x.
%	evictedLayer <-
%	
%	for user in timeout user:
%	if user.rrp < x:
%	remove the layers
	
	


%	\While{free\_FLASH $<$ $T_{flash}$}{
%	\emph{last\_repo} $\gets$ \emph{RepoLRU.last\_item()}\\
%	\For{last\_layer $\gets$ \emph{LayerLRU[last\_repo].last\_item()}}{
%		\If{layer exclusively belongs to last\_repo}{
%			%{\scriptsize $/*$\textit{If layer is not shared with other users, layer is deleted}}\\
%			%	last\_layer $\gets$ \emph{LayerLRU[last\_usr]}\\			
%			\emph{Discard(last\_layer)} \\
%			\emph{free\_FLASH} $+=$ \emph{sizeof(last\_layer)} \\
%		}
%	}
%}




%\begin{algorithm}
%    \tiny 
%	\caption{User access history based eviction}
%	\label{alg:prefetch}
%	%\SetAlgoLined
%	\KwIn{\\
%		$L_{thresh}$: Threshold for duration to keep a prefetched layer. \\
%		$RLMap$: Repository to layers map.\\
%		$URLmap$: User to layers map. \\
%	}
%	\While{true}{
%		\emph{r} $\leftarrow$ \texttt{request received}\\
%		\uIf{r = GET manifest}
%		{
%%			layerlst $\leftarrow$ URLmap[(r.client, r.repo)]
%			\emph{layers} $\gets$ \emph{RLMap[r.repo]} $-$ \emph{URLmap[r.client]} \\
%			\emph{OnTimelayers}, \emph{NotOnTimelayers} $\gets$ \emph{OnTimeCalculation(layers)} \\
%			\emph{MEMcache} $\gets$ \emph{Prefetch(OnTimelayers)} \\
%			\emph{FLASHcache} $\gets$ \emph{Prefetch(NotOnTimelayers)} \\
%			\texttt{set} \emph{L\_timer[layer] for each layer in layer} \\
%			{\tiny\texttt{/*when $L\_timer[layer] > L\_thresh$, layer is evicted/}}
%			}
%		\uElseIf{r = PUT layer }
%		{
%				\texttt{update} \emph{URLmap[(r.client, r.layer)]} \\
%				\texttt{update} \emph{RLMap[(r.repo, r.layer, put)]} \\
%				\emph{MEMcache} $\leftarrow$ \texttt{buffer} \emph{r.layer} \\
%				\texttt{set} \emph{L\_PUT\_timer[r.layer]} \\
%				{\tiny\texttt{/*when $L\_timer[layer] > L\_thresh$, layer is evicted/}}	 \\
%			}
%		\ElseIf{r = GET layer}
%		{
%				\eIf{r.layer in MEMcache or r.layer in FLASHcache}
%				{
%					\emph{serve from MEMcache or FLASHcache} \\
%					\texttt{update} \emph{URLmap[(r.client, r.layer)]} \\
%					\texttt{Reset} \emph{L\_timer[r.layer]}\\
%					\emph{hit++} 
%				}
%			   {
%					\emph{serve from backend storage system} \\
%					\texttt{update} \emph{URLmap[(r.client, r.layer, repulled)]} \\
%					\emph{RepulledLayers} $\gets$ \emph{RLMap[r.repo]} \\
%					\emph{FLASHcache} $\gets$ \emph{Prefetch(RepulledLayers)} \\
%					\texttt{set} \emph{L\_timer[layer] for each layer in layers} \\
%					{\tiny\texttt{/*when $L\_timer[layer] > L\_thresh$, layer is evicted/}}
%			}
%		}
%	}
%\end{algorithm}

\paragraph{LRU of (user+repo) based cache eviction}

