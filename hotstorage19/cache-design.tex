%\paragraph{User active time-based cache replacement algorithm}
%%\label{sec:design_cache_algori}
%
%\begin{algorithm}
	%\scriptsize 
	\caption{User-based cache replacement algorithm.}
	\label{alg:cache}
	\SetAlgoLined
	\KwIn{
		$S_{thresh}$: Capacity threshold for layer buffer to trigger eviction.
	}
	\While{free\_buffer $<$ $S_{thresh}$}{
		$last\_usr \gets UsrLRU.last\_item()$\\
		\For{layer in reversed(LayerLRU.items())}{
			\If{layer exclusively belongs to last\_usr}{
				%{\scriptsize $/*$\textit{If layer is not shared with other users, layer is deleted}}\\
				\textbf{Evict} $LayerLRU[layer]$\\			
				$free\_buffer += sizeof(layer)$
		}
	}
	\textbf{Evict} $UsrLRU[last\_usr]$
	}
\end{algorithm}
%
%We leverage observations from our study of the user access patterns to guide our cache replacement. This is because
%a user's active time is more predictable as discussed in~\cref{sec:background}.
%%   In~\cref{sec:design}, the incoming \texttt{push} layer requests are first buffered in the layer buffer and later evicted to the file cache.
%%   If there is a \texttt{pull} layer request miss in both the layer buffer and the file cache, 
%%   \sysname~will fetch the layer from backend storage system and store it in the layer buffer.
%We service requests from the layer buffer, then file cache, and then from the backend storage system. 
%When a cache (layer buffer or file cache) is under pressure due to 
%shortage of free space,
%the layer buffer or the file cache will simply evict or delete some layers or files and reclaim space.
%Since our layer buffer and file cache both share the same cache replacement algorithm, we only present
%our user-based cache replacement algorithm for the layer buffer as shown in Algorithm~\ref{alg:cache}.
%
%Free space in the layer buffer is considered too low if ($free\_buffer$ $<$ $S_{thresh}$). 
%In such a sitation, \sysname~will free the buffer space used by inactive users. 
%\sysname~maintains two LRU lists: an LRU list of active users, and
%an LRU list of recently accessed layers, referred to as \emph{UsrLRU} and \emph{LayerLRU}, respectively.
%At first, \sysname~will select the least active user from \emph{UsrLRU}.
%Next, \sysname~will \arb{what is???: reversely iterate} \emph{UsrLRU} until it finds a layer that is exclusively owned by that least active user. 
%\sysname~then evicts this layer from the layer buffer. 
% Note that layers are shared among different active users.
% \sysname~will continue to evict until the free space in the layer buffer is above the specified capacity threshold.
% 
%%iteratively removes the least active users along with the least recently accessed layers 
%% that exclusively owned by these users from the two LRU lists and evict the layers from layer buffer.
%%, then   
\section{User behavior based preconstruct cache}
\label{sec:cache-design}
\paragraph{User-access-history-based prefetch algorithm.}
%\label{sec:design_cache_algori}
To improve our cache hit ratio for \texttt{pull} layer request, we propose a user-access-history-based prefetch algorithm. The algorithm
exploits the uniqueness of the registry's 
%image %structure
dataset hierarchy: repositories comprise a list of layers.
%or organization
%Users create repositories in the registry, \texttt{push} layers to their own repositories,
%and \texttt{pull} layers from their own repositories or other public repositories. 
When a user \emph{pulls} an image from a repository, it will first \texttt{pull} the manifest of the image~\cite{docker}~\cite{dockerworkload} and 
parse the manifest to get the layer digests,
then lookup each layer digest against a \emph{local layer digest index}.
After that it only \emph{pulls} the layers that has \emph{not been stored locally}.
%
Based on the above 
%behavior 
pattern and hierarchy, we can record the users' repository and layer access history. 
Theoretically, once a user issues a \texttt{pull} manifest from a repository, all the layers that belong to this repository but have not been \emph{pulled} by this user should be prefetched into the cache.
%when a new use connects to registry, we can prefetch all the layers that have been accessed by this user into cache ideally.
In this case, the cache hit ratio will reach 1.
%However, because there is a limit to the cache size, %we won't be able to fit all the active users' layers into cache.
%not all active user's layers will fit into the cache.
%%But we can prefetch active users' popular layers in the cache based on the historical access information.
%To mitigate this, we can only prefetch the active user's top requested layers based on the historical access information.

\begin{algorithm}
\scriptsize 
	\caption{User access history based prefetch}
	\label{alg:prefetch}
	%\SetAlgoLined
	\KwIn{\\
		$L_{thresh}$: Threshold for duration to keep a prefetched layer. \\
		{\tiny\texttt{/*when $L\_timer[layer] > L\_thresh$, layer is evicted or demoted to Flashcache */}}\\
		$RLMap$: Repository to layers map.\\
		$URLMap$: User to layers map. \\
	}
	\While{true}{
		\emph{r} $\leftarrow$ \texttt{request received}\\
		\uIf{r = GET manifest}
		{
%			layerlst $\leftarrow$ URLMap[(r.client, r.repo)]
			\emph{layers} $\gets$ \emph{RLMap[r.repo]} $-$ \emph{URLMap[r.client]} \\
			\emph{OnTimelayers}, \emph{NotOnTimelayers} $\gets$ \emph{OnTimeCalculation(layers)} \\
			\emph{MEMcache} $\gets$ \emph{Prefetch(OnTimelayers)} \\
			\emph{FLASHcache} $\gets$ \emph{Prefetch(NotOnTimelayers)} \\
			\texttt{set} \emph{L\_timer[layer] for each layer in layers} \\
			%{\tiny\texttt{/*when $L\_timer[layer] > L\_thresh$, layer is evicted/}}
			}
		\uElseIf{r = PUT layer }
		{
				\texttt{update} \emph{URLMap[(r.client, r.layer)]} \\
				\texttt{update} \emph{RLMap[(r.repo, r.layer, put)]} \\
				\emph{MEMcache} $\leftarrow$ \texttt{buffer} \emph{r.layer} \\
				\texttt{set} \emph{L\_PUT\_timer[r.layer]} \\
				%{\tiny\texttt{/*when $L\_timer[layer] > L\_thresh$, layer is evicted/}}	 \\
			}
		\ElseIf{r = GET layer}
		{
				\eIf{r.layer in MEMcache or r.layer in FLASHcache}
				{
					\emph{serve from MEMcache or FLASHcache} \\
					\texttt{update} \emph{URLMap[(r.client, r.layer)]} \\
					\texttt{Reset} \emph{L\_timer[r.layer]}\\
					\emph{hit++} \\
					{\tiny\texttt{/* if r.layer in FLASHcache, layer is promoted to MEMcache/}}	 \\
				}
			   {
					\emph{serve from backend storage system} \\
					\texttt{update} \emph{URLMap[(r.client, r.repo, repulled)]} \\
					\emph{RepulledLayers} $\gets$ \emph{RLMap[r.repo]} \\
					\emph{FLASHcache} $\gets$ \emph{Prefetch(RepulledLayers)} \\
					\texttt{set} \emph{L\_timer[layer] for each layer in RepulledLayers} \\
					%{\tiny\texttt{/*when $L\_timer[layer] > L\_thresh$, layer is evicted/}}
			}
		}
	}
\end{algorithm}


\begin{algorithm}
	\scriptsize 
	\caption{User access history based eviction.}
	\label{alg:eviction}
	\SetAlgoLined
	\KwIn{\\
		$T_{mem}$: Capacity threshold for MEM cache to trigger demotion. \\
		$T_{flash}$: Capacity threshold for FLASH cache to trigger eviction. \\
		$UsrLRU$: LRU of users.  \\
		$LayerLRU[Usr]$: LRU of layers that are accessed by user $Usr$. \\
		$RepoLRU$: LRU of repostiories. \\
		$LayerLRU[Repo]$: LRU of layers that are associated with repository $Repo$.
	}
	\While{free\_MEM $<$ $T_{mem}$}{
		\emph{last\_usr} $\gets$ \emph{UsrLRU.last\_item()}\\
		\For{last\_layer $\gets$ \emph{LayerLRU[last\_usr].last\_item()}}{
			\If{layer exclusively belongs to last\_usr}{
				%{\scriptsize $/*$\textit{If layer is not shared with other users, layer is deleted}}\\
			%	last\_layer $\gets$ \emph{LayerLRU[last\_usr]}\\			
				FLASHcache $\gets$ \emph{Demote(last\_layer)} \\
				\emph{free\_MEM} $+=$ \emph{sizeof(last\_layer)} \\
			}
		}
	}

	\While{free\_FLASH $<$ $T_{flash}$}{
	\emph{last\_repo} $\gets$ \emph{RepoLRU.last\_item()}\\
	\For{last\_layer $\gets$ \emph{LayerLRU[last\_repo].last\_item()}}{
		\If{layer exclusively belongs to last\_repo}{
			%{\scriptsize $/*$\textit{If layer is not shared with other users, layer is deleted}}\\
			%	last\_layer $\gets$ \emph{LayerLRU[last\_usr]}\\			
			\emph{Discard(last\_layer)} \\
			\emph{free\_FLASH} $+=$ \emph{sizeof(last\_layer)} \\
		}
	}
}

\end{algorithm}


%\begin{algorithm}
%    \tiny 
%	\caption{User access history based eviction}
%	\label{alg:prefetch}
%	%\SetAlgoLined
%	\KwIn{\\
%		$L_{thresh}$: Threshold for duration to keep a prefetched layer. \\
%		$RLMap$: Repository to layers map.\\
%		$URLMap$: User to layers map. \\
%	}
%	\While{true}{
%		\emph{r} $\leftarrow$ \texttt{request received}\\
%		\uIf{r = GET manifest}
%		{
%%			layerlst $\leftarrow$ URLMap[(r.client, r.repo)]
%			\emph{layers} $\gets$ \emph{RLMap[r.repo]} $-$ \emph{URLMap[r.client]} \\
%			\emph{OnTimelayers}, \emph{NotOnTimelayers} $\gets$ \emph{OnTimeCalculation(layers)} \\
%			\emph{MEMcache} $\gets$ \emph{Prefetch(OnTimelayers)} \\
%			\emph{FLASHcache} $\gets$ \emph{Prefetch(NotOnTimelayers)} \\
%			\texttt{set} \emph{L\_timer[layer] for each layer in layer} \\
%			{\tiny\texttt{/*when $L\_timer[layer] > L\_thresh$, layer is evicted/}}
%			}
%		\uElseIf{r = PUT layer }
%		{
%				\texttt{update} \emph{URLMap[(r.client, r.layer)]} \\
%				\texttt{update} \emph{RLMap[(r.repo, r.layer, put)]} \\
%				\emph{MEMcache} $\leftarrow$ \texttt{buffer} \emph{r.layer} \\
%				\texttt{set} \emph{L\_PUT\_timer[r.layer]} \\
%				{\tiny\texttt{/*when $L\_timer[layer] > L\_thresh$, layer is evicted/}}	 \\
%			}
%		\ElseIf{r = GET layer}
%		{
%				\eIf{r.layer in MEMcache or r.layer in FLASHcache}
%				{
%					\emph{serve from MEMcache or FLASHcache} \\
%					\texttt{update} \emph{URLMap[(r.client, r.layer)]} \\
%					\texttt{Reset} \emph{L\_timer[r.layer]}\\
%					\emph{hit++} 
%				}
%			   {
%					\emph{serve from backend storage system} \\
%					\texttt{update} \emph{URLMap[(r.client, r.layer, repulled)]} \\
%					\emph{RepulledLayers} $\gets$ \emph{RLMap[r.repo]} \\
%					\emph{FLASHcache} $\gets$ \emph{Prefetch(RepulledLayers)} \\
%					\texttt{set} \emph{L\_timer[layer] for each layer in layers} \\
%					{\tiny\texttt{/*when $L\_timer[layer] > L\_thresh$, layer is evicted/}}
%			}
%		}
%	}
%\end{algorithm}

%U_thresh
%
%while true do
%	r <- request received
%		if r = Get manifest then
%			layerlst <- UrlMap[(r.client, r.repo)]
%			layers <- choose_popular(layerlst) 
%			PrefetchedLayers <- Prefetch(layers) 
%			set L_timer[layer] for each layer in layers
%			/when L_timer[layer] > L_thresh, layer is evicted/
%		if r = Put layer then
%			update UrlMap[(r.client, r.repo, r.layer)]
%			Layerbuffer <- buffer r.layer
%			set L_timer[r.layer] for r.layer
%			/when L_timer[layer] > L_thresh, layer is evicted/		
%		if r = Get layer then
%			if r.layer in PrefetchedLayers then
%				serve from PrefetchedLayers[r.layer]
%				update UrlMap[(r.client, r.repo, r.layer)]
%				prefetch_hit++
%			else
%				server from backend storage system
%				update UrlMap[(r.client, r.repo, r.layer)]
%				layerlst <- UrlMap[(r.client, r.repo)]
%				layers <- choose_popular(layerlst) 
%				PrefetchedLayers <- Prefetch(layers) and 
%				set L_timer[layer] for each layer in layers
%				/when L_timer[layer] > L_thresh, layer is evicted/


As shown in algorithm~\ref{alg:prefetch}, \sysname maintains two maps: a RLMap for recording layer-repository
relationship, and 
a URLMap for recording 
users' repository and layer access history information. 
For example, if a user~\emph{U} \emph{pulls} a layer~\emph{L} from a repository~\emph{R},
\sysname will add an new entry (\emph{U,L}) in URLMap.
%Each node records the following history information: (\emph{Get\_cnt}, \emph{Put\_cnt}, \emph{last\_access\_time}). a child node layer~\emph{L} to parent node~\emph{R}
While if a user~\emph{U} \emph{pushes} a  layer~\emph{L} to a repository~\emph{R},
\sysname~will add an new entry (\emph{R,L}) in RLMap. 
Note that to identify which layers are locally available for a user, 
we extract \emph{user end host address} (\emph{r.client}) 
%as shown in Algorithm~\ref{alg:prefetch}
from each request and define the user end host address as user,
and keep track of all layers that have been downloaded by \emph{r.client}. 
%Note that for a layer node, \emph{Put\_cnt} $=$ 1 or $=$ 0.
%When a GET or PUT layer request is received, \sysname~will update the URLMap of the associating user, repository, and layer. 
When either a GET manifest request is received or 
%a GET layer request is miss,
%a requested layer is not cached or prefetched  (
a miss on a GET layer request happens,
\sysname will lookup RLMap and get the requested repository's containing layers,
and compare against the layers that are already \emph{pulled} by the user by looking up URLMap,
%select a certain number of \emph{popular} layers from the client's 
%previously accessed layers by lookup 
%URLMap
then prefetch the layers that have not been \emph{pulled}. 
We set a timer for each cached layer and evict it when its timer is $>U_{thresh}$.
To incorporate the algorithm with \sysname,
we prefetch \texttt{slices} in parallel from backend servers,  
%dedup storage system,
buffer them in the layer buffer first, then evict them into the file cache after they \emph{cool down}.
%    for the user later accesses.
%To determine if a layer is a popular layer and need to be prefetched into cache,
%we only consider the layers that was recently accessed with a time period 
%We consider popular layers to be the layers accessed during the most recent~\emph{T\_thresh} 
%seconds?
%time period. In other words, the layers that satisfy the condition:
%\emph{current\_time} $-$ \emph{last\_access\_time} $<$ \emph{T\_thresh}.
%Then we calculate the layer popularity as: 
%$\omega_{pull} \times Pull\_cnt + \omega_{push} \times Push\_cnt$.
%Since recently \emph{pushed} layer has a higher chance to get \emph{pulled} in the future,
%we give a higher weight ($\omega_{push} > \omega_{pull}$) for the layer if it has a \emph{Push\_cnt} equal to 1. 

 
%We assume that users are independent with each other.

