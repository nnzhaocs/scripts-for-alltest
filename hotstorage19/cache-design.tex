
\subsection{User behavior based layer preconstruction cache}
\label{sec:cache-design}

In the following, we present our \sysname~\preconstructcachename design.

%Each \texttt{pull layer} request has a precedent \texttt{pull manifest} request.
%Upon receiving a \texttt{pull manifest} request, 
%\sysname~sends the updated \sysname~manifest to client.
%After receiving a \sysname~manifest, the client parses the manifest
%and sends either a \texttt{pull layer} request if the layer hasn't been deduplicated,
%or a list of \texttt{pull slice} requests if a list of corresponding slices presents in the manifest.
%Those \texttt{pull slice} requests will then be
%forwarded to all the registry servers that store the requested
%layer's slices as shown in Figure~\ref{fig:sys-overview}. 

\subsubsection{User access pattern based preconstruction}

Docker client stores images as lists of layers and layers are shared among different repositories, which is similar to Docker registry.
When a client \texttt{pull}s an image from a repository, 
it will first \texttt{pull} the manifest of the image~\cite{docker}~\cite{dockerworkload} and 
parse the manifest to get the layer digests,
then lookup each layer digest against a \emph{local layer digest index}.
After that it only \emph{pulls} the layers that have \emph{not been stored locally}.
Theoretically, clients only pull layer once. 
However, some clients may delete some local images and \emph{repull} layers for these images.
%Moreover, kubernetes allows users always \emph{repull} layers no matter these layers locally available or not~\cite{docker}.  
Here, a \emph{repull layer} means users pull this layer multiple times,
and a \emph{non-repull layer} means users only pull this layer once.

\preconstructcachename~starts layer restoring when a \texttt{pull} manifest request is received, which is called layer preconstruction.
These preconstructed layers are temporally stored in the cache for later use.
In this case, layer restoring process and overhead can be avoided if requested layer is found in the cache.
Next, we analyze user access patterns to identify which layers in the repository will be pulled by users.

\paragraph{User access patterns}
Figure~\ref{xxx} shows the CDF of layer repull count.
We see that majority of users don't \emph{repull} layers frequently.
For \texttt{Syd}, only 4\% of layers are repulled by the same clients.
\texttt{Dev} has the highest repull layer ratio of 36\% while 83\% of the repull layers are only repulled twice.
Majority of repulled layers are repulled infrequently.
For example, only 3\% of layers from \texttt{Syd} are repulled more than twice.
Layer from \texttt{Prestage} and \texttt{Lon} have the highest repull frequency.
5\% of layers are pulled more than 6 times.
We also observe that few clients \emph{repull} layers continuously.
The highest layer repull count is 19,300 from \texttt{Lon}.
We think these clients probably deploy containers on a shared platform such as Cloud,
and run ephemeral jobs such as stateless microservices. 
Once the applications are finished, the container images are automatically deleted.

When different clients \texttt{pull} the same repository, 
they will fetch different amount of layers from the repository based on the availability of their local layer dataset.
Even the same clients \texttt{pull} the same repository at different times, 
they will fetch different amount of layers from the repository because their local layer dataset changes over time.
Therefore, a \texttt{pull manifest} requests doesn't usually result in pulling/repulling the layers in the repository. 
Here, we define \emph{repulling repository} as 
repulling the layer in the repository for the same client.
Figure~\ref{xxx} shows the CDF of the probability of repository repulling.
The probability of repository repulling is calculated 
as the number of \texttt{pull manifest} resulting in repository repulling divided by 
the total number of \texttt{pull manifest} requests issued by the same client for the same repository.
We see that majority of repositories aren't repulled.
The repull repository ratio ranges from 15\% for \texttt{Prestage} to 43\% for \texttt{Prestage}.
Majority of repull repositories have a low repulling probability.
Only 20\% of repositories from  \texttt{Prestage}, \texttt{Stage}, and 
\texttt{Syd} have a repulling probability higher than 0.5.
And only 20\% of repositories from the rest 4 workloads have a repulling probability higher than 0.33.
We also observe that few repositories' repulling probability are 1, meaning 
every time clients pull these repositories, they always repull the layers in these repositories. 
 
Figure~\ref{xxx} shows the client repulling probability.
Client repulling probability is calculated as the number of \emph{repull} layer requests divided by
the number of \texttt{pull} layer requests issued by the same client.
We see that majority of clients do repull layers but the probability is low.
60\% of clients from \texttt{Prestage}, \texttt{Dev}, \texttt{Lon}, and \texttt{Fra} have a repulling probability lower than 0.1.
55\% of clients from both \texttt{Dal} and \texttt{Stage} have a repulling probability lower than 0.1.
Less clients have repulling probability range between 0.1 to 0.7.
10\%-30\% of clients have a repulling probability ranged from 0.1-0.7 across 7 workloads.
We find few clients repull layers continuously.
2\%-12\% of clients have a repulling probability higher than 0.9 from workloads:
\texttt{Dal}, \texttt{Dev}, \texttt{Fra}, \texttt{Prestage},
\texttt{Stage}, \texttt{Syd}, and \texttt{Lon}.

\paragraph{User profiles}

\preconstructcachename~maintains a RLMap for recording repository-layer relationship.
If a user~\emph{u} \emph{pushes} a  layer~\emph{l} to a repository~\emph{r},
\preconstructcachename~will add an new entry (\emph{l}) in RLMap[\emph{r}]. 
Based on user access patterns,
\preconstructcachename~ also maintains URLMap for keeping track of user access patterns.
To identify a user, 
we extract \emph{user end host address} (\emph{r.client}) from each request (\emph{r}). 
If a user~\emph{u} \texttt{pull}s a layer~\emph{l} from a repository~\emph{r},
\preconstructcachename will update entry URLMap[\emph{u}].
%Each node records the following history information: (\emph{Get\_cnt}, \emph{Put\_cnt}, \emph{last\_access\_time}). a child node layer~\emph{L} to parent node~\emph{R}
Each URLMap entry maintains a user profile for each client as shown in Figure~\ref{xxx}.
User profile contains a list of repo profiles for each accessed repository,
and each repo profile contains a list of accessed layer profiles.
Note that layer profiles can be shared among different repo profiles for the same user.
User profile records the user repulling probability $u.rp$ for client $u$.
If $u.rp$ is greater than threshold $\theta_{crp}$, this user has a high probability of repulling layers.
Repo profile records repository repulling probability $u.r.rp$ for repository $r$.
If $u.r.rp$ is greater than threshold $\theta_{rrp}$, this repository is a popular repulling repository for this user.
Layer profile records layer repull count $u.l.R$ for layer $l$.
If $u.l.R$ is greater than threshold $\theta_{R}$, this layer is a popular repull layer for this user.

\paragraph{Preconstruction algorithm}
\begin{algorithm}
\scriptsize 
	\caption{User access history based prefetch}
	\label{alg:prefetch}
	%\SetAlgoLined
	\KwIn{\\
		$L_{thresh}$: Threshold for duration to keep a prefetched layer. \\
		{\tiny\texttt{/*when $L\_timer[layer] > L\_thresh$, layer is evicted or demoted to Flashcache */}}\\
		$RLMap$: Repository to layers map.\\
		$URLMap$: User to layers map. \\
	}
	\While{true}{
		\emph{r} $\leftarrow$ \texttt{request received}\\
		\uIf{r = GET manifest}
		{
%			layerlst $\leftarrow$ URLMap[(r.client, r.repo)]
			\emph{layers} $\gets$ \emph{RLMap[r.repo]} $-$ \emph{URLMap[r.client]} \\
			\emph{OnTimelayers}, \emph{NotOnTimelayers} $\gets$ \emph{OnTimeCalculation(layers)} \\
			\emph{MEMcache} $\gets$ \emph{Prefetch(OnTimelayers)} \\
			\emph{FLASHcache} $\gets$ \emph{Prefetch(NotOnTimelayers)} \\
			\texttt{set} \emph{L\_timer[layer] for each layer in layers} \\
			%{\tiny\texttt{/*when $L\_timer[layer] > L\_thresh$, layer is evicted/}}
			}
		\uElseIf{r = PUT layer }
		{
				\texttt{update} \emph{URLMap[(r.client, r.layer)]} \\
				\texttt{update} \emph{RLMap[(r.repo, r.layer, put)]} \\
				\emph{MEMcache} $\leftarrow$ \texttt{buffer} \emph{r.layer} \\
				\texttt{set} \emph{L\_PUT\_timer[r.layer]} \\
				%{\tiny\texttt{/*when $L\_timer[layer] > L\_thresh$, layer is evicted/}}	 \\
			}
		\ElseIf{r = GET layer}
		{
				\eIf{r.layer in MEMcache or r.layer in FLASHcache}
				{
					\emph{serve from MEMcache or FLASHcache} \\
					\texttt{update} \emph{URLMap[(r.client, r.layer)]} \\
					\texttt{Reset} \emph{L\_timer[r.layer]}\\
					\emph{hit++} \\
					{\tiny\texttt{/* if r.layer in FLASHcache, layer is promoted to MEMcache/}}	 \\
				}
			   {
					\emph{serve from backend storage system} \\
					\texttt{update} \emph{URLMap[(r.client, r.repo, repulled)]} \\
					\emph{RepulledLayers} $\gets$ \emph{RLMap[r.repo]} \\
					\emph{FLASHcache} $\gets$ \emph{Prefetch(RepulledLayers)} \\
					\texttt{set} \emph{L\_timer[layer] for each layer in RepulledLayers} \\
					%{\tiny\texttt{/*when $L\_timer[layer] > L\_thresh$, layer is evicted/}}
			}
		}
	}
\end{algorithm}


\begin{algorithm}
	\scriptsize 
	\caption{User access history based eviction.}
	\label{alg:eviction}
	\SetAlgoLined
	\KwIn{\\
		$T_{mem}$: Capacity threshold for MEM cache to trigger demotion. \\
		$T_{flash}$: Capacity threshold for FLASH cache to trigger eviction. \\
		$UsrLRU$: LRU of users.  \\
		$LayerLRU[Usr]$: LRU of layers that are accessed by user $Usr$. \\
		$RepoLRU$: LRU of repostiories. \\
		$LayerLRU[Repo]$: LRU of layers that are associated with repository $Repo$.
	}
	\While{free\_MEM $<$ $T_{mem}$}{
		\emph{last\_usr} $\gets$ \emph{UsrLRU.last\_item()}\\
		\For{last\_layer $\gets$ \emph{LayerLRU[last\_usr].last\_item()}}{
			\If{layer exclusively belongs to last\_usr}{
				%{\scriptsize $/*$\textit{If layer is not shared with other users, layer is deleted}}\\
			%	last\_layer $\gets$ \emph{LayerLRU[last\_usr]}\\			
				FLASHcache $\gets$ \emph{Demote(last\_layer)} \\
				\emph{free\_MEM} $+=$ \emph{sizeof(last\_layer)} \\
			}
		}
	}

	\While{free\_FLASH $<$ $T_{flash}$}{
	\emph{last\_repo} $\gets$ \emph{RepoLRU.last\_item()}\\
	\For{last\_layer $\gets$ \emph{LayerLRU[last\_repo].last\_item()}}{
		\If{layer exclusively belongs to last\_repo}{
			%{\scriptsize $/*$\textit{If layer is not shared with other users, layer is deleted}}\\
			%	last\_layer $\gets$ \emph{LayerLRU[last\_usr]}\\			
			\emph{Discard(last\_layer)} \\
			\emph{free\_FLASH} $+=$ \emph{sizeof(last\_layer)} \\
		}
	}
}

\end{algorithm}


%\begin{algorithm}
%    \tiny 
%	\caption{User access history based eviction}
%	\label{alg:prefetch}
%	%\SetAlgoLined
%	\KwIn{\\
%		$L_{thresh}$: Threshold for duration to keep a prefetched layer. \\
%		$RLMap$: Repository to layers map.\\
%		$URLMap$: User to layers map. \\
%	}
%	\While{true}{
%		\emph{r} $\leftarrow$ \texttt{request received}\\
%		\uIf{r = GET manifest}
%		{
%%			layerlst $\leftarrow$ URLMap[(r.client, r.repo)]
%			\emph{layers} $\gets$ \emph{RLMap[r.repo]} $-$ \emph{URLMap[r.client]} \\
%			\emph{OnTimelayers}, \emph{NotOnTimelayers} $\gets$ \emph{OnTimeCalculation(layers)} \\
%			\emph{MEMcache} $\gets$ \emph{Prefetch(OnTimelayers)} \\
%			\emph{FLASHcache} $\gets$ \emph{Prefetch(NotOnTimelayers)} \\
%			\texttt{set} \emph{L\_timer[layer] for each layer in layer} \\
%			{\tiny\texttt{/*when $L\_timer[layer] > L\_thresh$, layer is evicted/}}
%			}
%		\uElseIf{r = PUT layer }
%		{
%				\texttt{update} \emph{URLMap[(r.client, r.layer)]} \\
%				\texttt{update} \emph{RLMap[(r.repo, r.layer, put)]} \\
%				\emph{MEMcache} $\leftarrow$ \texttt{buffer} \emph{r.layer} \\
%				\texttt{set} \emph{L\_PUT\_timer[r.layer]} \\
%				{\tiny\texttt{/*when $L\_timer[layer] > L\_thresh$, layer is evicted/}}	 \\
%			}
%		\ElseIf{r = GET layer}
%		{
%				\eIf{r.layer in MEMcache or r.layer in FLASHcache}
%				{
%					\emph{serve from MEMcache or FLASHcache} \\
%					\texttt{update} \emph{URLMap[(r.client, r.layer)]} \\
%					\texttt{Reset} \emph{L\_timer[r.layer]}\\
%					\emph{hit++} 
%				}
%			   {
%					\emph{serve from backend storage system} \\
%					\texttt{update} \emph{URLMap[(r.client, r.layer, repulled)]} \\
%					\emph{RepulledLayers} $\gets$ \emph{RLMap[r.repo]} \\
%					\emph{FLASHcache} $\gets$ \emph{Prefetch(RepulledLayers)} \\
%					\texttt{set} \emph{L\_timer[layer] for each layer in layers} \\
%					{\tiny\texttt{/*when $L\_timer[layer] > L\_thresh$, layer is evicted/}}
%			}
%		}
%	}
%\end{algorithm}

%U_thresh
%
%while true do
%	r <- request received
%		if r = Get manifest then
%			layerlst <- UrlMap[(r.client, r.repo)]
%			layers <- choose_popular(layerlst) 
%			PrefetchedLayers <- Prefetch(layers) 
%			set L_timer[layer] for each layer in layers
%			/when L_timer[layer] > L_thresh, layer is evicted/
%		if r = Put layer then
%			update UrlMap[(r.client, r.repo, r.layer)]
%			Layerbuffer <- buffer r.layer
%			set L_timer[r.layer] for r.layer
%			/when L_timer[layer] > L_thresh, layer is evicted/		
%		if r = Get layer then
%			if r.layer in PrefetchedLayers then
%				serve from PrefetchedLayers[r.layer]
%				update UrlMap[(r.client, r.repo, r.layer)]
%				prefetch_hit++
%			else
%				server from backend storage system
%				update UrlMap[(r.client, r.repo, r.layer)]
%				layerlst <- UrlMap[(r.client, r.repo)]
%				layers <- choose_popular(layerlst) 
%				PrefetchedLayers <- Prefetch(layers) and 
%				set L_timer[layer] for each layer in layers
%				/when L_timer[layer] > L_thresh, layer is evicted/


As shown in algorithm~\ref{alg:prefetch}, 

%keep track of all layers that have been downloaded by \emph{r.client}. 
%Note that for a layer node, \emph{Put\_cnt} $=$ 1 or $=$ 0.
%When a GET or PUT layer request is received, \sysname~will update the URLMap of the associating user, repository, and layer. 
When either a GET manifest request is received or 
%a GET layer request is miss,
%a requested layer is not cached or prefetched  (
a miss on a GET layer request happens,
\sysname will lookup RLMap and get the requested repository's containing layers,
and compare against the layers that are already \emph{pulled} by the user by looking up URLMap,
%select a certain number of \emph{popular} layers from the client's 
%previously accessed layers by lookup 
%URLMap
then prefetch the layers that have not been \emph{pulled}. 
We set a timer for each cached layer and evict it when its timer is $>U_{thresh}$.
To incorporate the algorithm with \sysname,
we prefetch \texttt{slices} in parallel from backend servers,  
%dedup storage system,
buffer them in the layer buffer first, then evict them into the file cache after they \emph{cool down}.

\subsubsection{User access pattern  based cache replacement}

\begin{algorithm}
	\scriptsize 
	\caption{User access history based eviction.}
	\label{alg:eviction}
	\SetAlgoLined
	\KwIn{\\
		$T_{mem}$: Capacity threshold for MEM cache to trigger demotion. \\
		$T_{flash}$: Capacity threshold for FLASH cache to trigger eviction. \\
		$UsrLRU$: LRU of users.  \\
		$LayerLRU[Usr]$: LRU of layers that are accessed by user $Usr$. \\
		$RepoLRU$: LRU of repostiories. \\
		$LayerLRU[Repo]$: LRU of layers that are associated with repository $Repo$.
	}
	\While{free\_MEM $<$ $T_{mem}$}{
		\emph{last\_usr} $\gets$ \emph{UsrLRU.last\_item()}\\
		\For{last\_layer $\gets$ \emph{LayerLRU[last\_usr].last\_item()}}{
			\If{layer exclusively belongs to last\_usr}{
				%{\scriptsize $/*$\textit{If layer is not shared with other users, layer is deleted}}\\
			%	last\_layer $\gets$ \emph{LayerLRU[last\_usr]}\\			
				FLASHcache $\gets$ \emph{Demote(last\_layer)} \\
				\emph{free\_MEM} $+=$ \emph{sizeof(last\_layer)} \\
			}
		}
	}

	\While{free\_FLASH $<$ $T_{flash}$}{
	\emph{last\_repo} $\gets$ \emph{RepoLRU.last\_item()}\\
	\For{last\_layer $\gets$ \emph{LayerLRU[last\_repo].last\_item()}}{
		\If{layer exclusively belongs to last\_repo}{
			%{\scriptsize $/*$\textit{If layer is not shared with other users, layer is deleted}}\\
			%	last\_layer $\gets$ \emph{LayerLRU[last\_usr]}\\			
			\emph{Discard(last\_layer)} \\
			\emph{free\_FLASH} $+=$ \emph{sizeof(last\_layer)} \\
		}
	}
}

\end{algorithm}


%\begin{algorithm}
%    \tiny 
%	\caption{User access history based eviction}
%	\label{alg:prefetch}
%	%\SetAlgoLined
%	\KwIn{\\
%		$L_{thresh}$: Threshold for duration to keep a prefetched layer. \\
%		$RLMap$: Repository to layers map.\\
%		$URLMap$: User to layers map. \\
%	}
%	\While{true}{
%		\emph{r} $\leftarrow$ \texttt{request received}\\
%		\uIf{r = GET manifest}
%		{
%%			layerlst $\leftarrow$ URLMap[(r.client, r.repo)]
%			\emph{layers} $\gets$ \emph{RLMap[r.repo]} $-$ \emph{URLMap[r.client]} \\
%			\emph{OnTimelayers}, \emph{NotOnTimelayers} $\gets$ \emph{OnTimeCalculation(layers)} \\
%			\emph{MEMcache} $\gets$ \emph{Prefetch(OnTimelayers)} \\
%			\emph{FLASHcache} $\gets$ \emph{Prefetch(NotOnTimelayers)} \\
%			\texttt{set} \emph{L\_timer[layer] for each layer in layer} \\
%			{\tiny\texttt{/*when $L\_timer[layer] > L\_thresh$, layer is evicted/}}
%			}
%		\uElseIf{r = PUT layer }
%		{
%				\texttt{update} \emph{URLMap[(r.client, r.layer)]} \\
%				\texttt{update} \emph{RLMap[(r.repo, r.layer, put)]} \\
%				\emph{MEMcache} $\leftarrow$ \texttt{buffer} \emph{r.layer} \\
%				\texttt{set} \emph{L\_PUT\_timer[r.layer]} \\
%				{\tiny\texttt{/*when $L\_timer[layer] > L\_thresh$, layer is evicted/}}	 \\
%			}
%		\ElseIf{r = GET layer}
%		{
%				\eIf{r.layer in MEMcache or r.layer in FLASHcache}
%				{
%					\emph{serve from MEMcache or FLASHcache} \\
%					\texttt{update} \emph{URLMap[(r.client, r.layer)]} \\
%					\texttt{Reset} \emph{L\_timer[r.layer]}\\
%					\emph{hit++} 
%				}
%			   {
%					\emph{serve from backend storage system} \\
%					\texttt{update} \emph{URLMap[(r.client, r.layer, repulled)]} \\
%					\emph{RepulledLayers} $\gets$ \emph{RLMap[r.repo]} \\
%					\emph{FLASHcache} $\gets$ \emph{Prefetch(RepulledLayers)} \\
%					\texttt{set} \emph{L\_timer[layer] for each layer in layers} \\
%					{\tiny\texttt{/*when $L\_timer[layer] > L\_thresh$, layer is evicted/}}
%			}
%		}
%	}
%\end{algorithm}
\paragraph{Temporal trend}

Figure~\ref{xxx} shows the CDF of layer popularity.
We observe a heavy layer access skewness for \texttt{Fra}, \texttt{Syd}, \texttt{Dal}, \texttt{Stage}, 
and \texttt{Lon}.
We see that 80\%, 70\%, and 60\% of the \texttt{pull} layer requests access only 10\% of layers, 
for \texttt{Fra}, \texttt{Syd}, \texttt{Dal}, \texttt{Stage}, 
and \texttt{Lon}.
Figure~\ref{xxx} shows the CDF of repository popularity.
Compare to layer popularity, 
repository access skewness is heavier acorss 7 workloads.
~90\% of \texttt{pull} layer requests access only 10\% of repositories for 
\texttt{Dev}, \texttt{Fra}, \texttt{Prestage}, \texttt{Syd}, and \texttt{Stage}.
~75\% of \texttt{pull} layer requests access only 10\% of repositories for
both \texttt{Dal} and \texttt{Lon}.
Figure~\ref{xxx} shows the CDF of client popularity.
\texttt{Dal}, \texttt{Dev}, \texttt{Fra}, \texttt{Lon}, \texttt{Prestage}, and \texttt{Stage}
shows a heavey client access skewness.
10\% of clients send 95\% \texttt{pull} layer requests for \texttt{Lon}.
\texttt{Syd} shows a slight client skewness. 70\% of requests are sent by 36\% of clients.

Figure~\ref{xxx} shows the CDF of layer reuse time. 
Layer reuse time means the duration between two consecutive \texttt{pull} requests to the same layer.
We see that layer reuse time distribution varies among different workloads.
For \texttt{Fra}, \texttt{Syd}, and \texttt{Stage},
half of the layers' reuse time is shorter than 6 minutes.
While half of layers from \texttt{Dal} and \texttt{Lon} have a reuse time higher than 1 hour.
Half of layers from both \texttt{Prestage} and \texttt{Dev} are not accessed for over 100 hours.

Figure~\ref{xxx} shows the CDF of repository reuse time.
Repository reuse time means the duration between two consecutive \texttt{pull manifest} requests to the same repository.
We find that repository reuse time is much shorter than layer reuse time.
80\% of repositories are re-accessed within 2-12 minutes for the 7 workloads.

Figure~\ref{xxx} shows the CDF of client access intervals.
Client access intervals are much shorter than repository reuse time.
80\% of client are active for at least 1 - 3 minutes for the 7 workloads. 




