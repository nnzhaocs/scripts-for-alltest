\section{Background}
\label{sec:background}

\paragraph{Docker image storage.} Docker~\cite{docker} is a platform to
develop, deploy, and run applications inside \emph{containers}.
%
Users interact with Docker using the Docker client, which in turn sends
commands to the Docker host.
%
The Docker host runs a daemon process that implements the core logic of Docker
and is responsible for \emph{running} containers from locally available images.
%
A Docker image consists of an ordered series of \emph{layers}.
%
Each Docker layer contains a subset of the files in the image and often
represents a specific component/dependency of the image, \eg a shared library.
%
Layers can be shared between two or more images if the images depend on the
same layer.
%
Image layers are read-only.
%
When users start a container, Docker creates a new \emph{writable layer} on top
of the underlying read-only layers.
%
% (Figure~\ref{fig-docker-architecture}).
%
Any changes made to files in the image will be reflected inside the writable
layer via a copy-on-write~(COW) mechanism.
%
\VT{I propose to remove/reduce this paragraph. We can focus on registry only, I
think.}




Docker registry~\cite{docker-hub} is a platform for storing and distributing
container images.
%
It stores images in \emph{repositories}, each containing different versions
(\emph{tags}) of the same image, identified as \texttt{<repo-name:tag>}.
%
For each image, the Docker registry stores a \emph{manifest}.
%
The manifest is a JSON file, which contains the runtime configuration for a
container image (\eg target platform and environment variables) and the list of
layers which make up the image.
%
Layers are identified via a digest that is computed as a hash (SHA-256) over
the uncompressed content of the layer and stored as compressed archival files.
%
If a user tries to launch a container from an image that is not available
locally, Docker host \texttt{pulls} the required image from the Docker
registry.
%
Additionally, the daemon supports \texttt{building} new images and
\texttt{pushing} them to the registry.
%
By using layer-level content addressable store, Image layers are stored as
compressed archival files and image manifests as JSON files.


%\VT{Somewhere we need to describe the common setup: nginx, independent registry servers
%for balancy, object storage beneath.}
%
%
The current Docker registry software server is a single-node application.
%
To serve many requests concurrently, organizations typically deploy a load
balancer like NGINX in front of several independent registry
instances~\cite{dockerworkload, anwar-cloud19}.
%
All the instances delegate storage and retrieval of images to drivers that interact with
either a local file system or a remote cloud storage such as Amazon
S3~\cite{s3}, Microsoft Azure~\cite{azure}, OpenStack Swift~\cite{swift}, and
Aliyun OSS~\cite{aliyun}.
%
For example, Google Container Registry~\cite{GoogleContainerRegistry} uses
Google cloud as its backend image store. 
%
Users \texttt{push} and \texttt{pull}
Docker images to and from their repositories that are stored on cloud storage.
%



%
Upon a \texttt{push} layer request, Docker registry fowrards the layer to the
backend storage which stores multiple replicas of the layer.
%
The subsequent \texttt{pull} layer requests can be served by any of the three
layer replicas.  Service providers often use geographically distributed
registries for faster access, e.g., IBM's Container Registry setup spans five
regions~\cite{dockerworkload}. 
%
\VT{Should we avoid pull and push for layers? Strictly speaking these cations
only apply to images.}




\Ali{I am not sure how to weave in the following paragraph with the previous ones.}
%
\VT{I don't think we need this paragraph...}
%
\paragraph{Deduplication} Most existing cloud storage providers employ data
deduplication techniques to eliminate redundant data, same data stored more
than once.
%
Deduplication techniques significantly reduce storage needs and therefore
reduce storage costs and improve storage efficiency.
%
Data deduplication works by storing duplicate data chunks only once, keeping
only the unique data chunks. 
%
Current cloud providers deploy a cross-user client-side fixed-size-chunk-level
data deduplication that delivers the highest deduplication
gain~\cite{pooranian2018rare}.
%
These approaches maximize the benefit of deduplication: The cross-user data
deduplication treats cloud storage as a pool shared by all the cloud users,
because the potential for data deduplication is the highest as the probability
for redundancies and duplicates is higher the more inclusive the shared pool.
%
The fixed-size-chunk-level specifies that a fixed-size chunk is the unit for
checking for duplicates on cloud storage.
%
Google cloud and AWS employ StorReduce, a deduplication software that performs
in-line data deduplication transparently and resides between the client's
application and the hosting cloud storage.
%
StorReduce provide 80-97\% storage and bandwidth reduction to the cloud
providers~\cite{StorReduce_google}.
