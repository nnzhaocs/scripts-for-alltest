\section{Docker Registries}
\label{sec:background}

%\paragraph{Docker registry.}

%A Docker image consists of an ordered series of \emph{layers}.
%
%Each Docker layer contains a subset of the files in the image and often
%represents a specific component/dependency of the image, \eg a shared library.
%
%Layers can be shared between two or more images if the images depend on the
%same layer.

%In this section, we explain how the existing Docker registry works and how users
%can interact with it.
%
The main purpose of a registry is to store container images and make them available
to other users.
%
A registry allows Docker clients to \emph{push} images to and \emph{pull}
images from it.
%
A number of public Docker registry deployments
exist~\cite{docker-hub,amazon-ecr,jfrog-artifactory,azure-cr,google-cr} and
enterprises often resort to running private registries for improved security
and privacy.
%
The Docker community specifies the REST API for clients to communicate with the
registries~\cite{docker-registry-api} and provides a reference registry server implementation~\cite{docker-registry-software} which serves as
a basis for commercial deployments.





Docker registries group images into \emph{repositories}, each containing
versions (\emph{tags}) of the same image, identified as
\texttt{<repo-name:tag>}.
%
For each tagged image in a repository, the Docker registry stores a \emph{manifest}.
%
The manifest is a JSON file, which contains the runtime configuration for a
container image (\eg environment variables) and the list of layers that make
up the image.
%
A layer is stored as a compressed archival file and identified using a digest that is computed as a hash (SHA-256) over
the uncompressed content of the layer.
%
When pulling an image, a Docker client first downloads the manifest and then the
layers referenced in it (unless the layers are already present on the client).
%
When pushing an image, a Docker client first uploads the layers (if not already present in
the registry) and then uploads the manifest.





The current Docker registry software is a single-node application.
%
To scale the registry and serve many requests concurrently, organizations typically deploy a load
balancer like NGINX in front of several independent registry
instances~\cite{dockerworkload, littley2019bolt}.
%
All the instances delegate storage and retrieval of images to drivers that interact with
either a locally mounted shared file system or a remote object storage~\cite{s3,azure,swift,aliyun}.
%
%For example, the Google Container Registry~\cite{GoogleContainerRegistry} uses
%Google cloud as its backend image store.
%
Upon a \texttt{PUT} layer request from a Docker client, the Docker registry receives the layer and forwards it to the
backend storage driver.
%
Typically, multiple replicas of the layer are stored for redundancy.
%
Subsequent \texttt{GET} layer requests can be served by any registry server,
which will get the layer from the backend storage using the configured driver.

% that hold a replica of that layer.
%
%Service providers often use geographically distributed registries for faster
%access, e.g., IBM's Container Registry setup spans five
%regions~\cite{dockerworkload}. 


%\paragraph{Deduplication} Most existing cloud storage providers employ data
%deduplication techniques to eliminate redundant data, same data stored more
%than once.
%
%Deduplication techniques significantly reduce storage needs and therefore
%reduce storage costs and improve storage efficiency.
%
%Data deduplication works by storing duplicate data chunks only once, keeping
%only the unique data chunks. 
%
%Current cloud providers deploy a cross-user client-side fixed-size-chunk-level
%data deduplication that delivers the highest deduplication
%gain~\cite{pooranian2018rare}.
%
%These approaches maximize the benefit of deduplication: The cross-user data
%deduplication treats cloud storage as a pool shared by all the cloud users,
%because the potential for data deduplication is the highest as the probability
%for redundancies and duplicates is higher the more inclusive the shared pool.
%
%The fixed-size-chunk-level specifies that a fixed-size chunk is the unit for
%checking for duplicates on cloud storage.
%
%Google cloud and AWS employ StorReduce, a deduplication software that performs
%in-line data deduplication transparently and resides between the client's
%application and the hosting cloud storage.
%
%StorReduce provide 80-97\% storage and bandwidth reduction to the cloud
%providers~\cite{StorReduce_google}.
