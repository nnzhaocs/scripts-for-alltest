\section{Introduction}
\label{sec:intro}

%
% Introduction structure:
%
% 1. Containerization background, benefits, and popularity.
%
% 2. Background on container images and registry storage;
%    large number of images => high storage requirements for the registries.
%
% 3. Significant amount of unharvested duplication across images.
%    Traditional dedup cannot be applied directly - need to decompress
%    Even with decompression, things do not work well: perfomance overhead
%
% 4. Sift's idea
%
% 5. Evaluation: how and what we got
%

%%%
% 1. Background, benefits, and popularity of containerization;
%%%

\emph{Container runtimes} such as Docker~\cite{docker} 
%and \mbox{CRI-O}~\cite{cri-o}
leverage Linux kernel namespaces~\cite{namespaces} and cgroups~\cite{cgroups}
to efficiently virtualize computational resources.
%
Unlike the hardware virtualization-based Virtual Machines
(VMs)~\cite{adams2006comparison}, containers residing on the same host share
the Operating System (OS) kernel, which yields virtual environments with
lower overheads and faster starts~\cite{7819678}.
%
Further, container runtimes introduce portable, self-contained \emph{container
images} and are surrounded by a rich ecosystem of technologies that automate
and accelerate application development, deployment, and
management~\cite{cncfprojects}.

Given the benefits of containers, it is not surprising that they have seen a
remarkable adoption in modern cloud environments.
%
The rapid spread of containers is further fueled by the growing popularity of
microservices~\cite{wolff2016microservices},
%design methodology
for which containers act as an imperative enabler.
%
By now, all major cloud platforms endorse containers as a core deployment
technology~\cite{googlecengine,azurec,ibmkube,awscont} and adoption
is increasing.
%
According to \cite{20percentdocker}, in 2018, about 21\% of Datadog's customers' monitored hosts run Docker and this continues to grow by about 5\% annually.

%%%
% 2. Background on container images and registry storage;
%    large number of images => high storage requirements for the registries;
%%%

Images are at the core of containerized applications.
%
An application's container image includes the executable of the application
along with a complete set of its dependencies---other executables, libraries, and
configuration and data files.
%
Images are structured in \emph{layers}. When building an image with Docker, each
executed command, such as \texttt{apt-get install}, creates a new layer on
top of the previous one~\cite{dockerfile}.
%
The layer contains the files that the command has modified or added during its
execution.
%
%A layer can be shared between multiple images.
%
Docker leverages union file systems~\cite{docker-driver-eval} to
efficiently merge layers into a single file system tree when starting a container.
%
Identical layers across different images can be shared by containers.

To store and distribute container images Docker relies on image
\emph{registries} (e.g., Docker Hub~\cite{docker-hub}).
%
Docker clients can push images to or pull them from the registries as needed.
%
On the registry side, each layer is stored as a compressed tarball and
identified by a content-based address.
%
The Docker registry supports various storage backends for saving and retrieving
layers. For example, a typical large-scale setup stores each layer as an object in an
object store~\cite{s3,swift}.

As the container market continues to expand, Docker registries manage a
growing number of images and corresponding layers.
%
Some conservative estimates show that in spring 2019, Docker Hub alone
stored at least 2 million \emph{public} images totaling roughly 1~PB in
size~\cite{skourtis2019carving,dedupanalysis}. 
%
We believe that this is just the tip of the iceberg and the number of
\emph{private} images is significantly higher.
%
Other popular public
registries~\cite{amazon-ecr,jfrog-artifactory,azure-cr,google-cr}, as well as
on-prem registry deployments in large organizations, experience a similar
surge of container images.
%
As a result,
organizations spend an increasing amount of their storage and networking
infrastructure on operating image registries to provide adequate performance while keeping up with the demand.
%
%
%%%
% 3. Significant amount of unharvested duplication across images.
%%%

Deduplication is a popular method to reduce capacity demands of storage systems.
%
As Docker images have to be self-contained by definition, they are prone to high
redundancy as different images often rely on the same dependencies and hence,
images are likely to benefit from deduplication.
%
Indeed, a recent analysis of the
% $47$~TB ($167$~TB uncompressed)
Docker Hub image dataset showed that about $97$\% of files across layers are
duplicates~\cite{dedupanalysis}.
%
%Such high redundancy is not surprising: after all every Docker image has to be
%self-sufficient by definition and is typically built by developers without much
%coordination with others.
%
Registry storage backends exacerbate the redundancy further due to the
replication they perform to improve image durability and
availability~\cite{Bonvin:2010:SFS:1807128.1807162}.
%
For instance, assuming 3-way replication, the number of duplicate files in the
Docker Hub dataset reaches 99\%.
%
And yet, existing registry deployments do not exploit this high data redundancy
across layers to make the service more efficient.

Layer sharing is one approach to reduce this redundancy.
%
However, it is insufficient as layers are coarse and rarely identical due the the
fact that they are typically built by independent developers without much coordination
with others.
%
While existing deduplication methods can be applied~\cite{dedup1,dedup2,dedup3}, they
are problematic because \LR{add reasons against existing file- or block-level deduplication
strategies}
\NZ{fixed}
the image dataset is a set of \textbf{compressed} layer tarballs
and the deduplication ratio for compressed tarballs is low \Subil{Do we have a citation for this? Or did we run file and block level deduplication on compressed tarballs and found it have poor deduplication results?}.

%%%
% 4. Sift's idea
%     Even with decompression, things do not work well: perfomance overhead.
%%%

While applying decompression and then file-level deduplication to the Docker registry 
can yield promising storage
reduction benefits, it is not straightforward.
%
The main challenge is retaining acceptable performance such that client
interactions with the registry (in particular pushing and pulling of images)
are not affected.
%
A na\"{i}ve application of deduplication to the registry---decompressing
layers and storing individual files in content-addressable storage---introduces
prohibitive overheads due to high layer reconstruction 
(denoted as layer restoring) cost.
%
These slowdowns during image pulls are especially
harmful because they contribute directly to the start times of containers.
%
Our experiments showed that on average, na\"{i}ve deduplication increases layer
download latencies by 2.5$\times$ compared to the registry without deduplication and layer reconstruction.

In this paper we propose \sysname, the first Docker registry that natively
supports deduplication.
%
\sysname's design is tailored to
% Docker registry environments and workloads to
increase storage efficiency via deduplication of layers while
reducing the corresponding layer restoring overhead. \DIM{I don't understand the preceeding sentence.}
\NZ{can you read again?} \Subil{reworded. is this better?}
It employs four key techniques to reduce
the impact of utilizing layer deduplication on performance: 
%and keep request response time low:
%
%It exploits two kinds of redundancies: 1)~redundant data across layers;
%and 2)~redundancy caused by replication in the storage backend.
%
\begin{compactenumerate}
\item It exploits redundancy caused by replication in the storage backends and keeps a fixed
number of layer replicas as-is, without decompressing and deduplicating them.
%
Accesses to these replicas do not experience layer restoring overhead while
the remaining layer replicas are decompressed
 \DIM{Is this supposed to say "compressed"?}
\NZ{it is decompressed. otherwise how we do dedup} 
and deduplicated for storage reduction.
\Subil{I understand Dimitris' confusion. Maybe this is a better rewording: It maintains a level of redundancy by keeping a configurable number of layer replicas on certain assigned servers in the registry cluster, while also storing unique files (and maintaining replicas for redundancy purposes) from across all the layers in the remaining servers.}
%
\item It deduplicates less frequently accessed layers more
aggressively than popular layers to speed up accesses to popular
layers while still achieving high storage reduction.
%
\item It monitors and exploits user access patterns to preconstruct
layers before layer download requests arrive to avoid adding reconstruction
latency to layer pull requests.
%
\item During layer deduplication, it groups files in \emph{slices} and evenly
distributes them across the cluster to parallelize and speed up layer reconstruction.
%
%Depending on the user and environment requirements for performance, storage
%savings, and durability users can configure...
\end{compactenumerate}

%%%
% 4. Sift's evaluation
%%%

\VT{will populate as eval is ready}
\NZ{fixed}

We implement \sysname by modifying the existing registry's file system driver and backend as a 
layer deduplication driver and backend (detailed in \S\ref{sec:impl}).
We deploy \sysname
on a 14-node cluster and evaluated with real-world workloads and real layers.
With the \emph{highest performance mode},
\sysname outperforms the state-of-the-art by reducing the layer downloading latency up to 14\%.
With the \emph{highest deduplication mode},
\sysname saves up to 50\% of storage space compare to the registry without layer deduplication.
The remaining deduplication modes make different trade offs in performance and data reduction (detailed in \S\ref{sec:Evaluation}).
%
%The evaluation results show that uses a cache that takes into consideration the
