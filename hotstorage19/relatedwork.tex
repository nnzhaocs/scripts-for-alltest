\section{Related Work}
\label{sec:related}

%?: ~\cite{7158965}.
%?: ~\cite{dockerbook}.
%?: ~\cite{5655241} - dedup of containers?



%Due to its increasing popularity, Docker has recently received increased
%attention from the research community.

A number of studies investigated various dimensions of Docker storage
performance~\cite{dockerssd,shifter,slacker,exoclones,docker-driver-eval,
improve-cow-container-drivers,cntr,slacker}.
%
%Harter \etal~\cite{} studied 57 images from Docker Hub for a variety of metrics
%but not for data redundancy.
%The authors used the results from their study to derive a benchmark to evaluate
%the push, pull, and run performance of Docker graph drivers based on the studied
%images. Compared to Slacker, our analysis focuses on the entire Docker Hub dataset.
%
%Cito \etal~\cite{analysisdockergithub} studied 70,000
%Dockerfiles and focused on the image build process not its contents.
%of Docker ecosystem with a focus on prevalent quality issues, and the evolution of Docker 
%files based on a data set of 70,000 Docker files.
%However, their study did not focus on actual image data.
%
%Shu \etal~\cite{dockervulnerabile} studied the security vulnerabilities in Docker Hub
%for 356,218 images.
%
Anwar \etal~\cite{dockerworkload} performed a detailed analysis
of an IBM Docker registry workload but not the dataset.
%
% and found there is a strong need for more
%automated and systematic methods of applying security updates to Docker images. While
%the amount of images is similar compared to our study, Shu \etal focused on a subset
%of 100,000 repositories and different image tags in these repositories.
%
%Dockerfinder~\cite{dockerfinder} is a microservice-based prototype that allows searching
%for images based on  multiple attributes, e.g., image name, image size, or supported software 
%distributions. It also crawls images from remote Docker registry but the authors do
%not provide a detailed description of their crawling mechanism.
%
%Bhimani~\cite{dockerssd} \etal characterized the performance of persistent storage options
%for I/O intensive containerized applications with NVMe SSDs.
%
Data deduplication is a well
explored and widely applied technique~\cite{2009-sparse_indexing_inline_dedup_using_sampling-fast,
2001-low_bandwidth_network_fs-sosp,
2012-idedup-fast,
tarasov2014dmdedup,
2008-avoid_disk_bottleneck_data_domain_dedupfs-fast}.
%
A number of studies which focus on
real-world datasets~\cite{2009-dedup_effectiveness_on_vm_disk_images-systor,
2012-data_reduction_in_primary_storage-systor,
2012-hpc_practical_dedup_study-sc,
2013-charact_increment_changes_data_protect-atc,
msst16dedup-study,
2012-charact_backup_workloads-fast,
2013-charact_dedup_effic_big_data-iiswc}
 can be complementary to our approach.
%but to the best of our knowledge, we are the first to analyze a large-scale Docker
%registry dataset for its deduplication potential.
%and propose to apply deduplication to it.


%Unlike our study, their analysis
%is focused on the execution of containers rather than on their storage at the registry side.

%\textbf{Analysis on file systems}:
%File system contents have been widely studied in different operating system environments.
%Douceur and Bolosky collected and analyzed over ten-thousand Windows file
%systems~\cite{largefscontent}. They found that the size of file and directory are fairly
%consistent across file systems, but file lifetimes and file-name extensions vary based on
%the job function of the users. 
%Agrawal, Bolosky, Douceur, and Lorch collected and analyzed a five-year file-system metadata
%over 60,000 Windows file systems, and presented the temporal trends relating to file size,
%file type, and directory size~\cite{fiveyearfsmetadata}.
%Sienknecht, Friedrich, Martinka, Friedenbach collected file system data from UNIX systems
%based on a dataset of 46 systems, 267 file systems, 151,000 directories and 2,300,000 files
%and found small files dominate in count~\cite{distributedatainfs}.
%%Traeger, Zadok, Joukov, and Wright surveyed 415 file system and storage benchmarks from
%106 recent papers~\cite{xxx}.  
