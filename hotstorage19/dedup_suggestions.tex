\section{Discussion}

We propose additional optimizations that can help to speed up \sysname:

\begin{compactenumerate}
%
\item
%
As the majority of the pull time is caused by compression, we propose to cache
hot layers as precompressed tar files in the staging area.
%
%We observe that only a small proportion of images and layers are frequently
%requested and majority of images and layers are \textit{cold}.
%
%Figure~\ref{fig:pull-cnt} shows the total number of pulls from the time an
%image/layer has been stored in Docker Hub until May 30, 2017.
%
According to our statistics, only 10\% of all images were pulled
from Docker Hub more than 360 times from the time the image was first pushed to Docker Hub
until May 30, 2017. Moreover, we found that 90\% of pulls
went to only 0.25\% of images based on image pull counts.
%
This suggests the existence of both cold and hot images and layers.
%
%\VT{Nannan, can we instead compute that 90\% of pulls
%wen to 0.25\% of images?}
%
%
%translates to only 10\% of layers being pulled more than 660 times (at most).
%
%Note that we calculate the layer pull count shown in Figure~\ref{fig:pull-cnt}
%by aggregating the pull count of all images, which refer to this layer.
%
%Note that the image pull counts are crawled from Docker Hub website.
%
%Actual layer pull counts should be less because pulling an image does not
%necessarily pull all its containing layers if some layer have been previously
%downloaded and are already available locally.
%

\item
%
As deduplication provides significant storage savings, \sysname\ can use faster
but less effective local compression methods than gzip~\cite{lz4}.
%
%\VT{cite a few}

\item
%
%Deduplicating when workload is light As shown above, file-level deduplication
%comes with some performance overhead.
%
The registries often experience fluctuation in load with peaks and
troughs~\cite{dockerworkload}.
%
Thus, file-level deduplication can be triggered
when the load is low to prevent interference with 
client \texttt{pull} and \texttt{push} requests.
%
%To further improve the performance of \sysname\, we also suggest to use main
%memory for temporarily storing and processing \textit{small} layers.
%
%According to our findings (see~\S\ref{sec:dedup_ratio}), the majority of
%layers~(87.3\%) are smaller than 50\,MB and hence can be stored and processed
%in RAM to speed up deduplication. 
%
\end{compactenumerate}

%\begin{figure}
	\centering
	\includegraphics[width=0.4\textwidth]{graphs/pull-cnt.pdf}
	\caption{CDF of layer \& image pull count.
	}
	\label{fig:pull-cnt}
\end{figure}

%=======================================
%|             OLD VERSION              |
%=======================================

%\paragraph{Latency distribution for each operation}
%\subsubsection{When to start file-level dedup?} 

%\paragraph{Latency distribution for each operation}

%\paragraph{Small compression ratio and small layer size}
%
%\begin{figure}[!t]
	\centering
	\subfigure[CDF of compression ratio]{\label{fig_cdf_compression_ratio}
		\includegraphics[width=0.23\textwidth]{graphs/cdf_compression_ratio.pdf}
	}
	\subfigure[Histogram of comp. ratios]{\label{fig_his_compression_ratio}
		\includegraphics[width=0.223\textwidth]{graphs/his_compression_ratio.pdf}
	}
	\caption{Layer compression ratio distribution
	\vcomment{Different colors are used in figure (a) and (b) FLS/CLS}
	}
	\label{fig-compression-ratio}
\end{figure}

%
%\begin{figure}[!t]
	\centering
	\subfigure[CDF of layer sizes]{\label{fig_layer_size_cdf}
		\includegraphics[width=0.234\textwidth]{graphs/layer_size_mb.pdf}
	}
	\subfigure[Histogram of layer sizes]{\label{fig_hist_layer_size}
		\includegraphics[width=0.213\textwidth]{graphs/hist_layer_size.pdf}
	}
	\caption{Layer size distribution
%	\vcomment{Let's use CLS, ALS, and FLS abreviations\nancomment{addressed}}.
%	\vcomment{CLS size should go first}.
%	\vcomment{We need to use different types of lines (solid, dotted, etc.)
%		or markers (round, triangular)}.
%	\vcomment{In figure B it is not clear to which bar group corresponds
%		  to which layer size. I suggest to try to rotate the graph
%		  by 90 grads to fit all layer size labels.\nancomment{aligned label with bar}}
	}
	\label{fig-layer-size}
\end{figure}

%
%We found that most layers'compression ratio is really lower (?) while most of layers have a smaller size. 
%So how about we use archiving instead of compression if the network speed is higher (?GB/s)?

%\paragraph{Network transfer speed is high!}

%\subsubsection{File-level content addressable storage for cold layers}

%\begin{figure}
%	\centering
%	\includegraphics [width=0.45\textwidth]{plots/exp-total-stev-erase.eps}
%	\subfigure[]{\label{fig:per_layer_ratio_fcnt_cdf}
%		\includegraphics [width=0.23\textwidth]{graphs/}
%	}
%	\subfigure[Similar layer dedup]{\label{fig:per_layer_ratio_fcnt_pdf}
%		\includegraphics [width=0.22\textwidth]{graphs/graph_reconstruct_layers.pdf}
%	}
%	\caption{File-level content addressable storage model}
%	\label{fig:eval-stdev-erasure-cnt}
%\end{figure}

%\subsection{Hints for performance improvement and storage saving}

%\begin{table} 
%	\centering 
%	\scriptsize  
%	%\begin{minipage}{.5\linewidth}
%	\caption{Latency breakdown} \label{tbl:latency_breakdown} 
%	\begin{tabular}{|l|l|l|l|l|}%p{0.14\textwidth} 
%		\hline 
%		% after \\: \hline or \cline{col1-col2} \cline{col3-col4} ... 
%		% after \\: \hline or \cline{col1-col2} \cline{col3-col4} ... 
%		Operations/latency (S) & max & min & median & avg.\\
%		\hline
%		 gunzip decompression (RAM) & 257.16  & 0.04  & 0.15  & 0.39 \\
% 		\hline
% 		tar extraction (RAM) & 43.41  & 0.04  &  0.14  & 0.18 \\
%		\hline
%		Digest calculation (RAM) & 3455.01  & $<$0.00  & 0.05 & 10.65 \\
%		\hline
%		tar archiving (RAM)  & 53.44 & 0.04 & 0.14 & 0.19\\
%		\hline
%		gzip compression (RAM) & 496.04 & 0.04 & 0.15 & 2.10 \\
%%		\hline
%%		Total time (RAM) (with compression) & & & & \\
%%		\hline
%%		Total time (RAM) (without compression) & & & & \\
%		\hline
% 		\hline
% 		gunzip decompression (SSD) &   &   &    &  \\
% 		\hline
% 		tar extraction (SSD) &   &   &    &  \\
%		\hline
%		Digest calculation (SSD) &  &  & & \\
%		\hline
%		tar archiving (SSD) &  &  & & \\
%		\hline
%		gzip compression (SSD) & &  &  & \\
%%		\hline		 
%%		Total time (SSD) (with compression) & & & & \\
%%		\hline
%%		Total time (SSD) (without compression) & & & & \\
%		\hline
%		\hline
%		Network transfer & 20587.94 & $<$ 0.00 & $<$ 0.00 & 1.20 \\
%		\hline 	
%	\end{tabular} 
%\end{table}


%\begin{table} 
%	\centering 
%	\scriptsize  
%	%\begin{minipage}{.5\linewidth}
%	\caption{Summary of layer \& image characterization} \label{tbl:redundant_ratio} 
%	\begin{tabular}{|l|l|l|l|l|}%p{0.14\textwidth} 
%		\hline 
%		% after \\: \hline or \cline{col1-col2} \cline{col3-col4} ... 
%		% after \\: \hline or \cline{col1-col2} \cline{col3-col4} ... 
%		Metrics & max & min & median & avg.\\
%		\hline
%		Compressed layer size &   &   &   &  \\
%		\hline
%		Uncompressed layer size &   &   &    &  \\
%		\hline
%		Archival size &  &  & & \\
%		\hline
%		Compression ratio &   &   &    &  \\
%		\hline
%		Layer pull cnt. &  &  & & \\
%		\hline
%		File cnt. per layer &  &  & & \\
%		\hline
%		Dir. cnt. per layer &  &  & & \\
%		\hline
%		Layer depth &  &  & & \\
%		\hline
%		\hline
%		Compressed image size &  &  & & \\
%		\hline
%		Uncompressed image size & &  &  & \\
%		\hline
%		Archival image size & &  &  & \\
%		\hline
%		Compression ratio &   &   &    &  \\
%		\hline
%		Image pull cnt.  &  &  & & \\
%		\hline
%		Layer cnt. per image  &  &  & & \\
%		\hline
%		Shared layer cnt. per image  &  &  & & \\
%		\hline
%		File cnt. per layer &  &  & & \\
%		\hline
%		Dir. cnt. per layer &  &  & & \\
%		\hline	
%	\end{tabular} 
%\end{table} 

%\subsection{Constructing shared layers for redundant directories/files}
%
%\paragraph{Smaller number of layers are shared among different images}
%\begin{figure}[!t]
	\centering
	\subfigure[CDF of layer by layer count]{\label{fig_repeate_layer}
		\includegraphics[width=0.23\textwidth]{graphs/repeate_layer.pdf}
	}
	\subfigure[Histogram of images by layer count in images]{\label{fig_hist_repeate_layer}
		\includegraphics[width=0.223\textwidth]{graphs/hist_repeate_layer.pdf}
	}
	\caption{Compression rate distribution}
	\label{fig-repeat-layer-cnt}
\end{figure}
%
%\paragraph{Smaller pull latency than recompression model} the registry can prepare the reconstructed layers before users issue a pull request. But this model requires users to rebuild two layers.

%\subsubsection{Summary of Suggestions/trade-offs between dedup ratio and recompression overhead}
%
%\paragraph{1. using archiving instead of compression}
%\paragraph{2. using file-level dedup for cold images/layers}
%\paragraph{3. using file-level dedup economically}
%When to trigger file-level dedup?
%\paragraph{4. constructing shared layers for redundant dirs/files, for example,}
%%\subsection{Layer reconstruction model}
%%\subsubsection{Reconstruction overhead}
%%\subsubsection{Trade-offs between dedup ratio and reconstruction overhead}
%%\paragraph{Dedup ratio VS. Rebuild overhead}
%%\subsection{Evaluation results}
