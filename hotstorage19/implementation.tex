\section{Implementation}
\label{sec:impl}

\paragraph{Metadata database implementation}
We use Redis~\cite{xxx} as metadata database to save slice and layer recipes, file index, layer index, ULmap, and RLmap.
To guarantee the durability of metadata, 
we configure Redis to use an \emph{append-only file (AOF)} to log all changes.
Thus, when Redis server stops, we can restart Redis and re-play the AOF to rebuild the dataset to prevent data loss.
Moreover, we configure Redis to save RDB (Redis Database File) snapshots every few minutes to further improve metadata reliability. 
To improve metadata availability, 
we configure Redis to use three replicas.
In this case, Redis will not become a performance bottleneck.

\paragraph{Distributed lock for distributed deduplication}
We use Redis as a distributed lock to make sure that no file duplicate is stored in registry cluster.
For file index, layer index, slice and layer recipes, each key can be set to hold its value only if the key does not exist in Redis database (i.e, SETNX~\cite{xxx}).
When key already holds a value, a file duplicate or layer duplicate is identified and will be removed from registry cluster.
 Besides, 
 each registry instance maintains a  synchronization map for layer restoring to make sure that only a unique layer is restoring at a time.
 While different layers can be constructed in parallel. 
 In this case, redundant layer restoring can be avoid to save I/O and network bandwidth, and computation resources.
 Slice restoring is the same.
 
 \textbf{GET-DEDUP LOCK} for fast layer dedup.
 layer stage area delay 1-2 seconds.
 \textbf{GET-RESTORING LOCK} for fast layer restoring.
 sync.map and delay 5 seconds, and buffer
 Stage area to layer cache

\paragraph{Layer deduplication implementation}
During layer deduplication, only \emph{regular} file duplicates are removed.
\sysname does not deduplicate \emph{irregular} files.
After removing regular file duplicates, the remaining unique files are saved in a local file store.
File store is implemented by modifying Docker registry's file system driver~\cite{xxx}, 
which is local storage driver 
that uses a directory tree in the local file system.
The default root directory in the local file system for registry storage system is `/var/lib/registry' (denoted as $rootdir$).
Each layer is stored in directory `$rootdir$/docker/registry/v2/blobs/sha256/$\langle$first two hex bytes of layer digest$\rangle$/$\langle$hex layer digest$\rangle$/' (denoted as $layerdir$) as a file named `data'.
We store the unique files for the layer in the same directory with layer as `$layerdir$/uniquefiles/$\langle$hex file digest$\rangle$'.

\paragraph{Layer restoring implementation}
To speedup layer restoring performance,
we utilize a parallel gzip compression/decompression library (pgzip)~\cite{xxx} to do layer or slice compression and decompression.
Different from single-threaded standard gzip compression library~\cite{xxx},
pgzip splits compression stream into blocks so that data can be compressed in parallel to speedup compression.
pgzip is compatible with standard gzip library.

\paragraph{File cache and layer diskcache implementation}

File cache is implemented on bigcache~\cite{xxx}, which can provide fast and concurrent cache access.
Layer diskcache is implemented by using diskv~\cite{xxx}, which is a disk-backed key-value store.
However, both bigcache and diskv do not use any cache algorithm. 
Note that bigcache always evicts the oldest entries from cache without consider the entries' popularity.
We build two ARC lists for bigcache and diskv respectively by using a golang cache library~\cite{xxx}, 
which provides expirable cache with different cache algorithm, such as LFU, LRU and ARC~\cite{xxx}.
Note that we select ARC with expiration as our cache algorithm.
We combine ARC cache algorithm with bigcache as our file cache
and ARC cache algorithm with diskv as our layer diskcache.

\paragraph{Client implementation}
%For simply implementation,
Client is comprised of a trace replayer and 
a proxy emulator.
The trace replayer is implemented by modifying IBM cloud registry trace replayer~\cite{xxx}.
Instead of generating a random layer~\cite{xxx},
our trace replayer first matches IBM cloud registry trace with 
the real layers from our dataset downloaded from Docker Hub~\cite{xxx, xxx}.
Specially, we extract layer digest from layer request recorded in IBM traces~\cite{xxx}.
Then, 
we randomly match each layer digest to a layer in our dataset. 
Consequently, each layer request from IBM traces \texttt{pulls} or \texttt{pushs} a real layer.
For manifest requests recorded in IBM traces, we generate a random file to emulate a manifest file.
This is because sift do not deduplicate manifest file.
The trace replayer uses Docker registry client emulator Python-dxf~\cite{xxx} to 
\texttt{pull} or \texttt{push} layers or manifests to registry.
%In this case, each layer request 
%will \texttt{pull} or \texttt{push} a real layer.

In addition, client uses a proxy emulator to decide the destination registry server for each request.
The proxy emulator uses
consistent hashing algorithm (CHT)~\cite{xxx} to distribute layers and manifests.
 Proxy emulator maintains a CHT ring of registry servers
 and calculates a destination registry server for each \texttt{push} layer or manifest request
 by hashing its digest.% as a key.
%Registry clusters are known to.
%Proxy emulator uses layer and manifest digests as keys.
For \texttt{pull} manifest requests, 
proxy emulator also uses CHT to calculate the destination server.
While for \texttt{pull} layer requests, 
proxy emulator will first consult the metadata database to get the layer recipe.
If the recipe  presents, 
the emulator will send the request to the restoring master.
Otherwise, the emulator will calculate the destination server by using CHT,
and send the request to the destination server.
%While if the layer is deduplicated, client \texttt{pull} layer request will be forwarded to its restoring master
%and the master server will serve the client request.

\paragraph{Redundancy implementation}

To maintain reliability and fault-tolerance, 
traditional registry maintains a certain level of redundancy for each layer 
by delegating replication to backend
distributed storage systems, such as S3 and Swift~\cite{xxx}.
To implement the same level of redundancy for layers,
Sift storage system stores $R$ replicas for each layer on $R$ registry servers 
by using
chain replication~\cite{xxx}.
In this case, each layer has a primary replica and $R-1$ backup replicas.
During layer deduplication, 
only primary replica is dedupcated.
After that,
the remaining unique files
will be $R$-way replicated.
Then the layer will be
partitioned to $N*R$ servers
denoted as $N*R$ slices instead of $N$ slices(see Section~\cite{xxx}).
Here, each slice for the layer 
has a primary replica and $R-1$ replicas.
Consequently, each unique file stored in cluster
has $R$ replicas.
For layer restoring,
each layer has a primary restoring master and $R-1$ backup restoring master. 
After layer deduplication, 
the primary layer and its two backups are discard.
 

\paragraph{Dealing with big layers}
Due to our testbed cluster scale constrain,
we set a layer size limitation (i.e., 250 MB) to reduce their restoring overhead.
If a layer that exceeds the size limitation is received,
\sysname will first divide the big layer into sub-layers and update its associated manifests.
After that,
the sub-layers are deduplicated and restored as normal layers.
After a client receives sub-layers,   
it will decompress all the sub-layers together into a whole big layer.
%We implemented our preconstruction cache and \filecachename file cache on Redis cluster
%deployed by using Docker swarm, denoted as cache service.
%
%We decoupled our \sysname~implementation into cache implementation and backend dedup storage implementation. 
%We first modify the Docker registry source code -- local file system driver part by implementing decompression/compression, file-level deduplication, and 
%unique file distribution modules to the local file system driver so that the driver decompresses the layers and removing the duplicate files once the registry receives a layer tarfile, and fetching and compress the files into layer slices once the registry receives a get layer request. 
%We also implemented a simple round-robin based distributed key-value object store by modifying the local file system driver. 
%Once the file-level deduplication is done, the driver will re-distributed the newly added unique files to different destination servers based on round-robin. 
%Besides, we uses Zookeeper as the cluster coordinator, and MongoDB to store dedup metadata.
%
%Our \sysname~cache is also implemented by modifying Docker registry source code -- cache part. 
%The original Docker registry only caches local layer digests to identify if a requested layer is stored locally or not. 
%We first deploy our cache cluster by using Redis cluster. 
%We add our \sysname~cache into Docker registry source code as a new cache module which makes prefetch decisions when a GET/PUT request is received and talks to Redis cache. The metadata tables used by cache is also stored in MongoDB. For cache demotion/eviction and monitoring cache space utilization/backend dedup storage system performance, we elect a master registry node to do that by using Zookeeper as the cluster coordinator. 
%
%Therefore, by modifying Docker registry source code, we can configure Docker registry to provide two different kinds of functionalities. It can be either used as a distributed backend layer dedup storage system with the same Docker registry APIs or used as a distributed user-oriented cache with the same Docker registry APIs.

%\paragraph{Gzip compression}