\section{Implementation}
\label{sec:impl}
We decoupled our \sysname~implementation into cache implementation and backend dedup storage implementation. 
We first modify the Docker registry source code -- local file system driver part by implementing decompression/compression, file-level deduplication, and 
unique file distribution modules to the local file system driver so that the driver decompresses the layers and removing the duplicate files once the registry receives a layer tarfile, and fetching and compress the files into layer slices once the registry receives a get layer request. 
We also implemented a simple round-robin based distributed key-value object store by modifying the local file system driver. 
Once the file-level deduplication is done, the driver will re-distributed the newly added unique files to different destination servers based on round-robin. 
Besides, we uses Zookeeper as the cluster coordinator, and MongoDB to store dedup metadata.

Our \sysname~cache is also implemented by modifying Docker registry source code -- cache part. 
The original Docker registry only caches local layer digests to identify if a requested layer is stored locally or not. 
We first deploy our cache cluster by using Redis cluster. 
We add our \sysname~cache into Docker registry source code as a new cache module which makes prefetch decisions when a GET/PUT request is received and talks to Redis cache. The metadata tables used by cache is also stored in MongoDB. For cache demotion/eviction and monitoring cache space utilization/backend dedup storage system performance, we elect a master registry node to do that by using Zookeeper as the cluster coordinator. 

Therefore, by modifying Docker registry source code, we can configure Docker registry to provide two different kinds of functionalities. It can be either used as a distributed backend layer dedup storage system with the same Docker registry APIs or used as a distributed user-oriented cache with the same Docker registry APIs.

\paragraph{Gzip compression}