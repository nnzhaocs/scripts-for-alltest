\vspace{-4pt}
\subsection{Overview}
\label{sec:design}
\vspace{-4pt}

\begin{figure}[t]
	\centering
		%\begin{minipage}{0.225\textwidth}
			\centering
			\includegraphics[width=0.4\textwidth]{graphs/sys-architecture}
%\vspace{-4pt}
			\caption{Architecture of \sysname.}
			%\label{fig:ref_count}
		%\end{minipage}
%	\begin{minipage}{0.225\textwidth}
%		\centering
%		\includegraphics[width=1\textwidth]{graphs/slimmer-cache.png}
%		\caption{CDF of compress. and uncompress. layer size.}
%		\vspace{-3pt}
		\label{fig:sys-overview}
%\vspace{-4pt}
%	\end{minipage}
\end{figure}


Figure~\ref{fig:sys-overview} shows the architecture of \sysname.
 \sysname is comprised of a distributed registry cluster and a distributed metadata database.
 Layers are stored on registry cluster and 
Metadata, such as Docker image manifests, are stored on distributed NoSQL databases for 
reliability, consistency, and fast accesses.
Each \sysname~registry consists of a user behavior based layer preconstruct cache (\preconstructcachename),
and a layer restoring latency aware deduplication (\dedupname) system.

In the following, we describe how Docker clients interact with \sysname.
\paragraph{Push}
As shown in Figure~\ref{fig:sys-overview}, client \textit{A} creates a new hello-world image
\texttt{hello-world:new}
from the official image which only contains a single layer \textit{L1}
by 
committing
the modifications over \textit{L1} as a new layer \textit{L2}.  
When client \textit{A} pushes \texttt{hello-world:new} to the registry,
it only pushes the new layer \textit{L2} to the registry since registry already stores \textit{L1}.
When \sysname~receives \textit{L2}, 
it first cache \textit{L2} in \preconstructcachename~for later accesses,
and at the same time, 
\sysname~will also submit \textit{L2} to the backend storage system as shown in Figure~\ref{fig:sys-overview}.
\preconstructcachename~ uses write through policies. 
Since there is no modification to the layer, 
there is no data consistency issue between the cache and the backend storage system.
Next, client \textit{A} pushes a new manifest \textit{M1:0} to the registry and finish image pushing.
As shown, \textit{L2} is added to the manifest \textit{M1:0} for image \texttt{hello-world:new}.


\dedupname~process runs periodically to deduplicate compressed layer tarballs (detailed in~\cref{sec:dedup-desgin})
into unique files to save storage space.
As shown in Figure~\ref{fig:sys-overview}, cold layer \textit{L2} is selected to be deduplicated.
\dedupname~process decompresses \texttt{L2} and removes the duplicated files from \emph{uncompressed} \texttt{L2}.
After that, \dedupname~process evenly distributes the unique files to the registry servers.
In this case, each server stores a \textbf{deduplicated slice} of \texttt{L2}, from which a layer \textbf{slice} of \texttt{L2}
can be constructed.
We define all the per-server files belonging to a layer as a {\em deduplicated slice}. 
A server stores deduplicated slices for many layers, 
and a layer is composed of \emph{slices} 
that can be restored from the deduplicated slices stored on multiple servers, 
which allows restoring a layer in parallel. 
To do that, 
\dedupname~process uses copy-on-write to update the old manifest \textit{M1:0} by adding slices' digests into it 
and generates new manifest \textit{M1:1} as shown in Figure~\ref{fig:sys-overview}.
Slice digest is calculated by hashing slice content~\cite{xxx}.

\paragraph{Pull}

As shown in Figure~\ref{fig:sys-overview},
when client \textit{C} pulls an official image \texttt{hello-world} from registry,
\sysname~first check if the requested layer \textit{L1} presents in~\preconstructcachename.
If so, the \texttt{pull} layer request will be served by cache.
Otherwise, \dedupname~process starts parallel slice restoring process.
For example, when client \textit{B} pulls image \texttt{hello-world:new} from registry,
\sysname~sends the latest manifest \textit{M1:1} to client.
After receiving the \textit{M1:1}, client \textit{B} first parses \textit{M1:1} and get a list of slice digests for \textit{L2}.
Instead of sending ``\texttt{pull layer L2}'' request, client will send multiple ``\texttt{pull slice of L2}'' requests to registry.
As shown, client \textit{B} sends ``\texttt{pull slice S1}'', ``\texttt{pull slice S2}'', and 
``\texttt{pull slice S3}'' to the registry 
since \textit{L2} is comprised of \textit{S1} , \textit{S2}, and \textit{S3}.
These requests are forwarded to the servers that stores corresponding deduplicated slice of \textit{L2}.
%On \sysname~side, \textit{L2} is stored as deduplicated slices distributed among different servers.
The servers will start to restore slices from local deduplicated slices and send individual slices back to client 
in parallel as shown in Figure~\ref{fig:sys-overview}. 

\paragraph{Docker client modifications}
 
To interact with \sysname, Docker client is modified to 
parse \sysname~manifest with additional attributes -- slice digests.
If slice digests present in a layer object in the manifest JSON file, 
client will replace ``\texttt{pull layer} request with multiple ``\texttt{pull slice}'' requests.
  
%When a user requests a
%layer that is not present in the layer buffer, the request is forwarded to the
%file cache (detailed in~\cref{sec:design_operations}). 
%If a layer is also not found in the
%file cache, the request is forwarded to the backend dedup storage system.
%Note that after layer deduplication, unique files are
%scattered across multiple servers. 
%We define all the per-server files belonging to a layer as a {\em slice}. 
%A server stores slices for many layers, and a layer is composed of slices stored on multiple servers.
%To avoid the network latency caused by fetching slices from different servers and
%assembling them into a whole compressed layer, we split a \texttt{pull} request 
%into several~\texttt{pull slice}~requests. Those requests will then be
%forwarded to all the backend servers that store the requested
%layer's slices. 
%After a~\texttt{pull slice}~request is received, each backend server compresses the slice 
%and directly sends it back to the user.
%We modify the Docker client
%interface such that when it receives all the compressed slices, it can
%decompress them into a single layer. 
%Furthermore, compressing slices in parallel considerably lowers the layer compression latency,
%since compression time depends on the size of the
%uncompressed data.
%to cache layers and cache unique files after decompression and deduplication,
%respectively.  consists of a \emph{layer buffer} and a \emph{file cache}.  The
%layer buffer stores all the newly pushed layers in memory.  Although accessing
%memory is very fast, the size of main memory is limited. 
%All the slices for a layer are fetched in parallel for performance improvement.






%\sysname~seamlessly integrates 
%%the management of 
%caching and deduplication on the
%backend storage system (\emph{backend dedup storage}) with Docker registries.
%%
%We address a set of unique challenges to enable this integration.
%%
%First, for caching layers, \texttt{pull} layer requests are difficult to
%predict because layers are accessed infrequently.
%In~\cref{sec:background},
%%\arb{???}, 
%we have observed that about half of the layers are not
%accessed again for at least $1.3$~hours. Which means that if we
%cache a layer, we may need to wait a long time before we observe a hit on that layer.  %(as discussed in~\cref{sec:background}).  
%This is mainly 
%because when a user pulls an image from the registry, the Docker daemon on the
%requesting host will only pull the layers that are not locally stored.
%%\Ali{I do not understand the following sentence.}
%%Moreover, we have to consider that a user might deploy an applications on
%%multiple machines, so it's not easy to predict when a user will access which layers. 
%%%\Ali{The above statement is incorrect. You have to distinguish between GET layer requests
%%that are issued after a (PUSH layer + GET manifest) request and a normal GET layer request.
%%FAST paper only talk about case 1. Whereas you are generalizing that any GET layer request
%%should have a precedent GET layer request which is wrong. We can make a case
%%that not all GET layers requests have a precedent PUSH layer request but we can
%%not say that it takes a few days, weeks, or even months for a user to make a pull
%%layer request after a push layer request.}
%%\NZ{I mean the first case, push beyond your trace collection time.}
%%
%
%Second, we can not deduplicate compressed layers. For deduplication, each layer
%needs to be uncompressed, and only then can undergo file-level deduplication. Similarly,
%to restore a layer, we need to fetch files from multiple servers, and only then compress
%them in to a tar file to serve a \texttt{pull} layer request. 
%%\arb{that can service the ??? request}\NZ{addressed}. 
%This whole process can incur a 
%considerable performance overhead on \texttt{pull} layer requests.
%Deduplication also slows down
%\texttt{push} layer requests because of its high demand for CPU, memory, I/O, and network resources.
%%\Ali{Explain how push layer requests are not effected?}\NZ{fixed}
%
%%\subsection{Design}
%To address these issues, we propose a new registry design. The key feature of our design is a user-access-history-based prefetch algorithm that helps mitigate the performance degradation due to the 
%backend dedup storage system (Figure~\ref{fig:sys-overview}). Based
%on layer access pattern we observed in~\cref{sec:background} and user access history information,
%\sysname precisely prefetch the layers that may be pulled shortly.
%%has not been pulled in the requested repository
%%and the prefetched 
%%In this case, we can   
%%a user's active time is predictable. 
%%Thus, we leverage users' behavior, \ie
%%when a user is most likely to be active, to drive layer evictions from the cache.
%
%
%%\begin{figure}[t]
	\centering
		%\begin{minipage}{0.225\textwidth}
			\centering
			\includegraphics[width=0.4\textwidth]{graphs/fig-sift-manifest}
%\vspace{-4pt}
			\caption{\sysname~manifest.}
			%\label{fig:ref_count}
		%\end{minipage}
%	\begin{minipage}{0.225\textwidth}
%		\centering
%		\includegraphics[width=1\textwidth]{graphs/slimmer-cache.png}
%		\caption{CDF of compress. and uncompress. layer size.}
%		\vspace{-3pt}
		\label{fig:sys-overview}
%\vspace{-4pt}
%	\end{minipage}
\end{figure}
%
%Considering that layer sizes are typically about several MB~\cite{dockerworkload}, 
%a small main memory cache will be unable to accommodate
%all prefetched layers for all active users. 
%To address this issue, we 
%create separate caches for layers and \emph{unique} files, called {\em layer buffer} and {\em file cache}, respectively. 
%%Both caches comprise both
%%main memory and flash memory.
%%Layer buffer
%%\arb{are main memory for one type and flash for the other type, or both for both types. I assumed both types of memory are used, and there are two caches. check previous sentence for correctness.}\NZ{addressed}
%Note that, layers are  compressed tarballs and buffered in layer buffer, and 
%%sent by users
% \emph{unique} files are uncompressed files from which duplicates have been removed and stored on flash-based storage. 
%%We call compressed layer cache and \emph{deduped} files cache,
%%\emph{layer buffer} and \emph{file cache}, respectively.
%For 
%cache evictions, we first evict inactive users' layers from the layer buffer.
%Next, we \emph{dedup} the evicted layers, then store the \emph{unique} files
%into the file cache (detailed in~\cref{sec:design_operations}). 
%%the following operations: decompressing each evicted layer and comparing its
%%containing files with the files that are already stored in the file cache,
%%eliminating duplicate files, that is, only storing the unique files on flash
%%storage.



 
