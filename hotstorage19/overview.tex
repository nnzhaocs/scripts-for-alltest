\vspace{-4pt}
\subsection{Overview}
\label{sec:design}
\vspace{-4pt}

\begin{figure}[t]
	\centering
		%\begin{minipage}{0.225\textwidth}
			\centering
			\includegraphics[width=0.4\textwidth]{graphs/sys-architecture}
%\vspace{-4pt}
			\caption{Architecture of \sysname.}
			%\label{fig:ref_count}
		%\end{minipage}
%	\begin{minipage}{0.225\textwidth}
%		\centering
%		\includegraphics[width=1\textwidth]{graphs/slimmer-cache.png}
%		\caption{CDF of compress. and uncompress. layer size.}
%		\vspace{-3pt}
		\label{fig:sys-overview}
%\vspace{-4pt}
%	\end{minipage}
\end{figure}

%\input{fig-diskcache}

Figure~\ref{fig:sys-overview} shows the architecture of \sysname.
 \sysname comprises a cluster of registry servers and a distributed metadata database. 
Registry servers store and serve both mainfest and layer requests sent by clients.
As shown in Figure~\ref{fig:architecture}, 
each registry server local storage system is
divided into three parts: layer stage area, layer diskcache, and file store.

\emph{Layer stage area} temporally stores recently \texttt{pushed} layers by clients shown in Figure~\ref{fig:cache}.
After a layer is \texttt{deduplicated}, 
the remaining unique files are stored in a content 
addressable \emph{file store} with a small \emph{file cache}. 
Then, the layer is deleted from layer stage area.
To improve layer restoring performance, 
layer is evenly divided into several \textbf{slices} and distributed to different registry servers so that
the layer can be restored in parallel.
  
\emph{Layer diskcache} is a on-disk layer cache which caches popular layers for later layer accesses.
As shown in Figure~\ref{fig:cache}, 
if a \texttt{pull} layer request hits in layer stage area,
the layer will be moved to \emph{layer diskcache}.
%
Upon a \texttt{pull} layer request miss, \sysname will rebuild the layer from file store
based on its associated \textbf{layer recipe}. 
After that, the restored layer is stored in layer diskcache as shown in Figure~\ref{fig:cache}.
\sysname also preconstruct layers and save them into layer diskcache to improve cache hit ratio. 
Layer diskcache is a write-through cache.
So when layers are evicted from diskcache, they are simply discard.

Deduplication related metadata such as layer recipe, \textbf{slice recipe}, and \textbf{file index} is kept in a \emph{distributed database} for 
reliability, consistency, and fast accesses as shown in Figure~\ref{fig:architecture}. 
Slice recipe and layer recipe are used to restore slices and rebuild layer.
File index records \emph{file fingerprints} uniquely mapped to the associated physical files. %psychical file based on
Besides, \sysname keeps track of user accesses and repository status, 
denoted as \textbf{ULmap} and \textbf{RLmap}, 
and save them into distributed metadata database.

In the following, we now describe how \sysname performs layer deduplication and layer restoring.

%\sysname uses a fast and reliable distributed database to 
%%, and unique files after layer deduplication
%The metadata such as layer recipes and slice recipes which
%Each server in the \sysname~registry cluster uses part of its storage for storing assembled (\ie preconstructed) layers. 
%We name such storage: user behavior based layer preconstruct cache (\preconstructcachename).
%Another part, a layer restoring performance--aware deduplication system (\dedupname system), is for storing deduplicated files. 
%Such files' metadata (\eg Docker image manifests) is kept in a metadata store; a distributed NoSQL database for 
%reliability, consistency, and fast accesses.
%
%

%\paragraph{Push}
%
%%As shown in Figure~\ref{fig:sys-overview}, 
%Consider a Docker client, client \textit{A}, who creates a new hello-world image
%\texttt{hello-world:new}
%from the official image which only contains a single layer \textit{L1}. 
%Pushing a new version of hello-world image corresponds to performing a PUT of layer \textit{L2} to the registry and a PUT of manifest to the metadata store as shown in Figure~\ref{fig:sys-overview}. 
%%first PUT
%Because the registry already stores \textit{L1}, 
%%only the modifications to the hello-world image that are commited as a new layer \textit{L2}, 
%only the compressed \textit{L2} tarball is PUT to the registry reflecting the modifications to the hello-world image.
%When \sysname~receives \textit{L2}, 
%it will cache \textit{L2} in the \preconstructcachename~for later accesses,
%and at the same time submit \textit{L2} to the backend storage system (Figure~\ref{fig:sys-overview}).
%The \preconstructcachename~uses write through policies. 
%Since layers are immutable, no data consistency issue exists between the \preconstructcachename~and the backend storage system.
%% Second PUT (manifest)
%The addition of layer \textit{L2} to the hello-world image is reflected in the manifest \textit{M1:0} which is PUT to the registry.
%Putting the new manifest \textit{M1:0} to the registry for the new image \texttt{hello-world:new} conlcudes the process of pushing it.
%
%
%
%The \dedupname~process runs periodically to deduplicate compressed layer tarballs (detailed in~\cref{sec:dedup-desgin})
%into unique files to save storage space.
%As shown in Figure~\ref{fig:sys-overview}, cold layer \textit{L2} is selected to be deduplicated.
%The \dedupname~process decompresses \textit{L2}, removes duplicate files from the \emph{uncompressed} \textit{L2}, and  evenly distributes the remaining unique files across the registry servers.
%This way, each server stores a 
%%\textbf{deduplicated slice}
%unique \textbf{slice}
% of the deduplicated \textit{L2}, from which a layer comprising a \textbf{slice} of \textit{L2}
%can be constructed and cached in the \preconstructcachename. We define such layer as a {\em slice layer}.
%We define all the per-server files belonging to a layer as a {\em deduplicated slice}. 
%Each server stores deduplicated slices belonging to many layers. 
%A layer is represented as a set of deduplicated slices that are distributed across multiple servers.
%The layer can be restored 
%%to a compressed layer tarball 
%from those distributed slices. 
%Such distribution allows restoring a layer in parallel.
%%To do that, 
%%\dedupname~process uses copy-on-write to update the old manifest \textit{M1:0} by adding slices' digests into it 
%%and generates new manifest \textit{M1:1} as shown in Figure~\ref{fig:sys-overview}.
%To do that, the \dedupname~process generates a new manifest \textit{M1:1} that holds, in addition to the contents of the old manifest \textit{M1:0}, the digests to the unique slices %comprising
%that make up the layer as shown in Figure~\ref{fig:sys-overview}.
%The slice digest is calculated by hashing the slice content~\cite{xxx}.
%
%\paragraph{Pull}
%
%As shown in Figure~\ref{fig:sys-overview},
%when client \textit{C} pulls an official image \texttt{hello-world} from the registry,
%\sysname~first checks if the requested layer \textit{L1} is present in~\preconstructcachename.
%If so, the \texttt{GET} layer request will be served by the cache.
%Otherwise, the \dedupname~system starts a parallel layer restoring process by serving slice layers from their hosting servers.
%For example, when client \textit{B} pulls the \texttt{hello-world:new} image from the registry,
%\sysname~sends the latest manifest (\textit{M1:1}) to client \textit{B}.
%After receiving \textit{M1:1}, client \textit{B} 
%%first 
%parses \textit{M1:1} to get a list of slice digests for \textit{L2}.
%Then, instead of sending a ``\texttt{GET layer L2}'' request, client \textit{B} will send multiple ``\texttt{GET slice of L2}'' requests to the registry.
%As shown in Figure~\ref{fig:sys-overview}, client \textit{B} sends ``\texttt{GET slice S1}'', ``\texttt{GET slice S2}'', and 
%``\texttt{GET slice S3}'' to the registry 
%since \textit{L2} is comprised of the slices \textit{S1}, \textit{S2}, and \textit{S3}.
%These requests are forwarded to the servers that store the corresponding deduplicated slice of \textit{L2}.
%%On \sysname~side, \textit{L2} is stored as deduplicated slices distributed across different servers.
%The servers will start to restore slices, \ie compose slice layers, from local deduplicated slices and send the compressed slices to the client 
%in parallel %as shown in Figure~\ref{fig:sys-overview}. 
%When client \textit{B} receives all the slices, it decompresses them together into an uncompressed layer.
%
%\paragraph{Docker client modifications}
% 
%To interact with \sysname, the Docker client is modified to 
%parse \sysname~manifests that include additional attributes, %-- 
%\ie slice digests.
%Also, if slice digests are present in a layer object in the manifest JSON file, 
%the client will replace ``\texttt{GET layer}'' request with multiple ``\texttt{GET slice}'' requests.
%Moreover, when the client receives the slices from registry, 
%the client decompresses these slices together into an uncompressed layer.  





%When a user requests a
%layer that is not present in the layer buffer, the request is forwarded to the
%file cache (detailed in~\cref{sec:design_operations}). 
%If a layer is also not found in the
%file cache, the request is forwarded to the backend dedup storage system.
%Note that after layer deduplication, unique files are
%scattered across multiple servers. 
%We define all the per-server files belonging to a layer as a {\em slice}. 
%A server stores slices for many layers, and a layer is composed of slices stored on multiple servers.
%To avoid the network latency caused by fetching slices from different servers and
%assembling them into a whole compressed layer, we split a \texttt{pull} request 
%into several~\texttt{pull slice}~requests. Those requests will then be
%forwarded to all the backend servers that store the requested
%layer's slices. 
%After a~\texttt{pull slice}~request is received, each backend server compresses the slice 
%and directly sends it back to the user.
%We modify the Docker client
%interface such that when it receives all the compressed slices, it can
%decompress them into a single layer. 
%Furthermore, compressing slices in parallel considerably lowers the layer compression latency,
%since compression time depends on the size of the
%uncompressed data.
%to cache layers and cache unique files after decompression and deduplication,
%respectively.  consists of a \emph{layer buffer} and a \emph{file cache}.  The
%layer buffer stores all the newly pushed layers in memory.  Although accessing
%memory is very fast, the size of main memory is limited. 
%All the slices for a layer are fetched in parallel for performance improvement.






%\sysname~seamlessly integrates 
%%the management of 
%caching and deduplication on the
%backend storage system (\emph{backend dedup storage}) with Docker registries.
%%
%We address a set of unique challenges to enable this integration.
%%
%First, for caching layers, \texttt{pull} layer requests are difficult to
%predict because layers are accessed infrequently.
%In~\cref{sec:background},
%%\arb{???}, 
%we have observed that about half of the layers are not
%accessed again for at least $1.3$~hours. Which means that if we
%cache a layer, we may need to wait a long time before we observe a hit on that layer.  %(as discussed in~\cref{sec:background}).  
%This is mainly 
%because when a user pulls an image from the registry, the Docker daemon on the
%requesting host will only pull the layers that are not locally stored.
%%\Ali{I do not understand the following sentence.}
%%Moreover, we have to consider that a user might deploy an applications on
%%multiple machines, so it's not easy to predict when a user will access which layers. 
%%%\Ali{The above statement is incorrect. You have to distinguish between GET layer requests
%%that are issued after a (PUSH layer + GET manifest) request and a normal GET layer request.
%%FAST paper only talk about case 1. Whereas you are generalizing that any GET layer request
%%should have a precedent GET layer request which is wrong. We can make a case
%%that not all GET layers requests have a precedent PUSH layer request but we can
%%not say that it takes a few days, weeks, or even months for a user to make a pull
%%layer request after a push layer request.}
%%\NZ{I mean the first case, push beyond your trace collection time.}
%%
%
%Second, we can not deduplicate compressed layers. For deduplication, each layer
%needs to be uncompressed, and only then can undergo file-level deduplication. Similarly,
%to restore a layer, we need to fetch files from multiple servers, and only then compress
%them in to a tar file to serve a \texttt{pull} layer request. 
%%\arb{that can service the ??? request}\NZ{addressed}. 
%This whole process can incur a 
%considerable performance overhead on \texttt{pull} layer requests.
%Deduplication also slows down
%\texttt{push} layer requests because of its high demand for CPU, memory, I/O, and network resources.
%%\Ali{Explain how push layer requests are not effected?}\NZ{fixed}
%
%%\subsection{Design}
%To address these issues, we propose a new registry design. The key feature of our design is a user-access-history-based prefetch algorithm that helps mitigate the performance degradation due to the 
%backend dedup storage system (Figure~\ref{fig:sys-overview}). Based
%on layer access pattern we observed in~\cref{sec:background} and user access history information,
%\sysname precisely prefetch the layers that may be pulled shortly.
%%has not been pulled in the requested repository
%%and the prefetched 
%%In this case, we can   
%%a user's active time is predictable. 
%%Thus, we leverage users' behavior, \ie
%%when a user is most likely to be active, to drive layer evictions from the cache.
%
%
%%\begin{figure}[t]
	\centering
		%\begin{minipage}{0.225\textwidth}
			\centering
			\includegraphics[width=0.4\textwidth]{graphs/fig-sift-manifest}
%\vspace{-4pt}
			\caption{\sysname~manifest.}
			%\label{fig:ref_count}
		%\end{minipage}
%	\begin{minipage}{0.225\textwidth}
%		\centering
%		\includegraphics[width=1\textwidth]{graphs/slimmer-cache.png}
%		\caption{CDF of compress. and uncompress. layer size.}
%		\vspace{-3pt}
		\label{fig:sys-overview}
%\vspace{-4pt}
%	\end{minipage}
\end{figure}
%
%Considering that layer sizes are typically about several MB~\cite{dockerworkload}, 
%a small main memory cache will be unable to accommodate
%all prefetched layers for all active users. 
%To address this issue, we 
%create separate caches for layers and \emph{unique} files, called {\em layer buffer} and {\em file cache}, respectively. 
%%Both caches comprise both
%%main memory and flash memory.
%%Layer buffer
%%\arb{are main memory for one type and flash for the other type, or both for both types. I assumed both types of memory are used, and there are two caches. check previous sentence for correctness.}\NZ{addressed}
%Note that, layers are  compressed tarballs and buffered in layer buffer, and 
%%sent by users
% \emph{unique} files are uncompressed files from which duplicates have been removed and stored on flash-based storage. 
%%We call compressed layer cache and \emph{deduped} files cache,
%%\emph{layer buffer} and \emph{file cache}, respectively.
%For 
%cache evictions, we first evict inactive users' layers from the layer buffer.
%Next, we \emph{dedup} the evicted layers, then store the \emph{unique} files
%into the file cache (detailed in~\cref{sec:design_operations}). 
%%the following operations: decompressing each evicted layer and comparing its
%%containing files with the files that are already stored in the file cache,
%%eliminating duplicate files, that is, only storing the unique files on flash
%%storage.



 
