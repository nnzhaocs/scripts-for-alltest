
\begin{table}[h!]
	%\centering
	%\tiny 
	\scriptsize
	\caption{Workload parameters.}
	\begin{tabular}{| p{0.38cm} | p{0.5cm} | p{0.5cm} | p{0.5cm} | p{0.5cm}| p{0.5cm}| p{0.5cm}| p{1.25cm} |} 
		\hline
	Trace    &   \#GET L & \#GET M & \#PUT L & \#PUT M & \#Uniq L & \#Uniq M & Duration (Hr)/Speedup \\ 
		\hline\hline
		Dal    &  2867  & 2000   & 124  & 9     & 1278 & 88  & 6/12    \\ 
		\hline
		Fra     &  1602  & 3278   & 111  & 9     & 420 & 43  & 24 /48   \\
		\hline
		Lon    &  924    & 3972   & 98  & 6      & 698 & 88  & 6/12    \\
		\hline 
		Syd      &  1310   & 3653   & 35 & 2     & 154 & 18  & 30/60    \\  
		\hline
	\end{tabular}

\label{tab:eval-overall}
\end{table}


\begin{table}[h!]
	%\centering
	\scriptsize 
	\caption{Testing Dataset.}
	\begin{tabular}{| c | c | c | c | c | } 
		\hline
		Trace  (GB)  &   Dataset  & Data transferred  & GET L size  & PUT L size  \\ 
		\hline\hline
		Dal   & 20  & 37 & 35 & 2    \\ 
		\hline
		Fra     & 6  & 16 & 14 & 2   \\
		\hline
		Lon    &10   &  14 &12  & 2      \\
		\hline 
		Syd      &  2  &  9 & 8 & 1      \\  
		\hline
	\end{tabular}
	
	\label{tab:eval-dataset}
\end{table}


\begin{figure}[t]
	\centering
	\includegraphics[width=0.45\textwidth]{graphs/get-layer-latency.pdf}
	\caption{GET layer latency across different workloads from different schemes.}
	%	\vspace{-3pt}
	\label{fig:getlayerlatency}
	
\end{figure}

