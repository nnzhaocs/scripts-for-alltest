\section{\sysname~Design}
\label{sec:Sift}

%Our analysis in Section~\ref{sec:redundant_files} suggests that adding the
%support of file-level deduplication to Docker registry can significantly reduce
%its storage capacity requirements, especially in large-scale deployments.
%
%In this section, we first describe a high-level design of \emph{\sysname}---
%a Docker registry that supports file-level deduplication.
%We then proceed with a simulation-based evaluation of the expected performance
%implications.
%Integrated Caching and Deduplication
%Cache-assisted Inline deduplication system


%In this section, we first present the challenges brought by the unique characteristics of the Docker ecosystem.
%Next, we discuss our design goals and finally present the design of~\sysname. 
%Then we present the design of~\sysname. 
%O are to improve Docker registry performance and to decrease the backend storage requirements such as ???.
In this section, we present an overview of \sysname.
Then, we present how \sysname performs layer deduplication and layer restoring. 
Finally, we present our user-behavior based layer preconstruct diskcache. % (\preconstructcachename).
%a latency aware deduplication system (\dedupname system) and
\subsection{Overview}
\label{sec:overview}

\begin{figure}[t]
	\centering
		%\begin{minipage}{0.225\textwidth}
			\centering
			\includegraphics[width=0.4\textwidth]{graphs/sys-architecture}
%\vspace{-4pt}
			\caption{Architecture of \sysname.}
			%\label{fig:ref_count}
		%\end{minipage}
%	\begin{minipage}{0.225\textwidth}
%		\centering
%		\includegraphics[width=1\textwidth]{graphs/slimmer-cache.png}
%		\caption{CDF of compress. and uncompress. layer size.}
%		\vspace{-3pt}
		\label{fig:sys-overview}
%\vspace{-4pt}
%	\end{minipage}
\end{figure}


Figure~\ref{fig:sift} shows the architecture of \sysname.
%
\sysname consists of the two main components: %1)~\emph{frontend registries}, 
1)~a cluster of \emph{storage servers}, each
exposing the registry RESTful API and 2)~a distributed
\emph{metadata database}. 
%\NZ{arg1}
%
When uploading or downloading layers, 
Docker clients communicate with any Sift
server using the registry RESTful API.
%
%As in the original registry, 
%\sysname registries are independent from each
%other.
%
Each server in the cluster contains a registry RESTful API and a backend storage system.
%uses a \sysname driver that processes requests by
%communicatin with the metadata database and the backend storage servers.
%
Backend storage systems store
%manifests and
layers and perform deduplication, keeping the deduplication metadata in the
database.
%
%The metadata database keeps information necessary to deduplicate layers.
%
%To reduce deduplication overhead, \sysname tries to minimize the time it takes
%to restore a layer during a \texttt{pull} request.
%
\sysname uses three main techniques to reduce deduplication and restoring overhead:
1)~replica deduplication modes; 2)~parallel layer reconstruction; and
3)~predictive layer prefetching/preconstruction.




\paragraph{Replica deduplication modes}
%
For higher fault tolerance and availability existing registry setups replicate each layer.
%
\sysname also performs layer replication, but in addition allows to deduplicate
files inside the layers.

A \emph{basic deduplication mode} $n$ (B mode \emph{n}) defines that \sysname should only
keep $n$ layer replicas intact and deduplicate the remaining $R-n$ layer
replicas, where $R$ is the layer replication level. This is in comparison to R-way replication in existing registry setups that hold $R$ layer replicas.
%
At one extreme, B mode $R$ means that no replica should be deduplicated and
hence, provides the best performance but no data reduction.
%
At the other end, B mode $0$ deduplicates all layer replicas, i.e. it provides
the highest deduplication ratio but adds restoration overhead for every
\texttt{GET} request.
%
The remaining B modes in between allow to trade off performance and data
reduction.


For heavily skewed workloads, \sysname also provides a \emph{selective
deduplication mode} (S mode).
%
The S mode utilizes the skewness in layer popularity, observed
in~\cite{dockerworkload}, to decide, how many replicas should be deduplicated
for each layer.
%
As there are hot layers that receive the majority of \texttt{GET} requests,
S mode sets the number of intact layer replicas proportional to its
popularity.
%
This means that hot layers have more intact layer replicas and
hence, can be served faster.
%
On the other hand, cold layers have fewer replicas to improve storage utilization.

Figure~\ref{fig:sift-original} shows an example for B mode $1$ with $R=3$.
\sysname first creates 3 layer replicas across our servers. It maintains a single layer replica as the \emph{primary layer replica} on
server A for newly pushed layers.
%
Deduplication is then carried out in one of the other servers i.e. one layer replica is decompressed and any duplicate
files are discarded while unique files are kept.  The unique files are replicated
and saved on different servers B and C. Once deduplication is complete, the remaining two layer replicas are removed. 
%
Any subsequent \texttt{GET} layer requests are sent to server A since it
stores a complete layer replica.
%
If server A crashes, server B or C is needed to rebuild the layer and serve the
\texttt{GET} request.

To support the different deduplication modes, \sysname divides storage servers
into two groups (Figure~\ref{fig:sift}): a \emph{primary cluster}
consisting of \emph{P-servers} and a \emph{deduplication cluster} consisting of
\emph{D-servers}.
%
P-servers are responsible for storing full layer replicas and manifest replicas
while D-servers deduplicate layer replicas at the file level, store the unique
files, and replicate them.
%
The split ensures that layer replicas and their containing file
replicas are stored on different servers to maintain fault tolerance.


P- and D-servers form a 2-tier storage hierarchy. In the default case,
the primary cluster serves all incoming \texttt{GET} requests.
%
If a request can't be served from the primary cluster (e.g., due to a
node failure, or \sysname operating in B mode 0 or S modes),
it will be forwarded to the deduplication cluster and the requested
layer will be restored.
%
%\LR{Here we also need to talk about dealing with overloaded (not just crashed)
%P-servers.}
%
%\NZ{We use the superfetch cache on registry side instead of on backend storage
%system to save some workload overhead to backend storage system.}





\paragraph{Parallel layer reconstruction}
%
\sysname speeds up layer reconstruction through parallelism.
%
As shown in Figure~\ref{fig:sift}, each D-server's local storage is divided
into three parts: layer stage area, preconstruction cache, and file store.
%
The layer stage area temporarily stores newly added layer replicas.
%
After deduplicating a replica, the resulting unique files are stored in a
content addressable file store and replicated to the peer servers to provide
redundancy. Once all file replicas have been stored, the layer replica is deleted
from the layer stage area.% and cached in the preconstruction cache.
%
%The preconstruction cache is used to hold preconstructed
%layers~(see~\S\ref{sec:cache-design}).

\sysname distributes the layer's unique files onto several servers
(see~\S\ref{sec:dedup-desgin}).
%
All files on a single server belonging to the same layer are called a
\emph{slice}.
%
A slice has a corresponding \emph{slice recipe}, which defines the files that
are part of this slice, and a \emph{layer recipe} defines the slices needed to
reconstruct the layer.
%
This information is stored in \sysname{}'s metadata database.
%for reliability, consistency, and fast accesses. 
%
This allows D-servers to rebuild layer slices in parallel and thereby, improve
reconstruction performance.
%
%To avoid storing duplicate layers and keep track of already stored unique
%files,
\sysname maintains layer and file fingerprint indices in the
metadata database.




\paragraph{Predictive cache prefetching}
%
To improve the layer access latency, \sysname employs a cache layer in both the
primary and the deduplication cluster.
%
Each P-server has an in-memory \emph{user-behavior based superfetch cache} to
reduce disk I/Os.
%
When a \texttt{GET} manifest request is received from a user, \sysname
predicts which layers in the image will actually need to be \texttt{pulled} and
\emph{prefetches} them in the cache.
%
To do that, \sysname maintains two maps: \emph{RLmap} and \emph{ULmap}.
%
RLmap stores the mapping between images and layers while ULmap keeps track of a
user's access history, i.e. which layers the user has pulled and how many
times~(see details in~\S\ref{sec:cache-design}).

Additionally, to avoid layer restoring overhead,
each D-server maintains an on-disk layer cache (called \emph{user-behavior based preconstruct cache}).
%
As with the superfetch cache algorithm,
when a \texttt{GET} manifest request is received,
\sysname predicts which layers in the image will be \texttt{pulled},
preconstructs the layers, and loads them in the preconstruct cache.
%
%As shown in Figure~\ref{fig:intervals}, the time between a \texttt{GET} manifest request and its
%subsequent \texttt{GET} layer request is more than 1\,s on average \todo{Give more specifics,
%i.e. 50th percentile for different workloads}. 
%\NZ{The median duration ranges from 0.8 s to 6.4 s across all the workloads.}
%As we will show in~\S\ref{sec:eval-dedup}, this is
%sufficient to preconstruct the necessary layers and load them into the cache.


\subsection{Deduplication modes}
\label{sec:dedup-mode}

To satisfy different space saving and performance requirements,
\sysname provides two kinds of deduplication modes:
Basic deduplication modes and selective deduplication modes, denoted as \emph{B-modes} and \emph{S-modes}.
B-mode mantains a certain amount of layer replicas for each layer and deduplicates the rest layer replicas.
While s-mode mantains a number of layer replicas for different layer proportionate to their popularity.

\subsubsection{Basic deduplication modes}
Assume that registry storage system uses R-way replication.
Consider that layer dataset deduplication analysis~\cite{xxx} shows a file-level deduplication ratio of 2$\times$.
Thus, the average layer deduplication ratio for the layer dataset~\cite{xxx} is 2$\times$.

B-mode \emph{r} is defined as keeping \emph{r} layer replicas and deduplicating the rest of layer replicas, where $r \leq R$. 
The space saving is calculated by following equation~\ref{eq:b-savings}.
\begin{equation}\label{eq:b-savings}
Space savings = \frac{R-(r+\frac{1}{2}(R-r))}{R} = \frac{R-r}{2R}
\end{equation}

Layer pulling performance largely depends on registry server load.
Assume that the total number of P-servers and D-servers are \emph{P} and \emph{D} especially.
The total \texttt{pull} layer request load is \emph{L}.
Consequently, the average server load is $\frac{L}{P}$ and $\frac{L}{P+D}$ for \sysname and original registry respectively.
Note that original registry is \sysname's B-mode \emph{R}, which represents the best pulling layer performance with no space savings.
The performance degradation can be calculated as:
%$$
 %$\emph{f}_\frac{L}{P} \times \theta_{r}$. 
 
 \begin{equation}\label{eq:c-pull}
Performance degradation =  \Delta(\frac{L}{P} - \frac{L}{(P+D)})\times \delta_{R-r}
 \end{equation}
 where $\Delta(\frac{L}{P} - \frac{L}{(P+D)})$ denotes the performance degradation caused by server load increases.
 $\delta_{R-r}$ means the performance degradation caused by load variance among different servers impacted by replication level.

\subsubsection{Selective deduplication modes}
In S-mode, the number of kept layer replicas \emph{r} is proportional to their popularity ($r \leq R$).
Layer popularity is calculated as $\frac{\emph{l}_{pcnt}}{L}$, 
where $\emph{l}_{pcnt}$ is the total number of pulling request count to layer \emph{l}.  
Consequently, the number of kept layer replicas $r_{l}$ for layer \emph{l} is
%\emph{r} ($r \leq R$) is calculated as:
$r_{l} = R \times \Phi (\frac{\emph{l}_{pcnt}}{L})$, 
where $\Phi (\frac{\emph{l}_{pcnt}}{L})$ denotes a replication factor.

The space savings for S-mode is calculated as following equation~\ref{eq:s-savings}.
\begin{equation}\label{eq:s-savings}
Space savings = \frac{R-(\sum\limits_{l=1}^n(r_{l}+\frac{1}{2}(R-r_{l})))}{R} 
\end{equation}

The performance degradation is:
  \begin{equation}\label{eq:s-pull}
Performance degradation =  \Delta(\frac{L}{P} - \frac{L}{(P+D)})\times \prod\limits_{l=1}^n\delta_{R-r_{l}}
 \end{equation}
where $\prod\limits_{l=1}^n\delta_{R-r_{l}}$ means 
the performance degradation caused by load variance among different servers impacted by different replication levels.
 

\subsection{Deduplicating layers}
\label{sec:dedup-desgin}

Next, we explain in detail, how \sysname stores and deduplicates layers.
%
%\subsubsection{Layer replication, deduplication, and partitioning}
%
As the traditional Docker registry, \sysname maintains a \emph{layer index}.
After receiving a \texttt{PUT} layer request,
\sysname first checks the \textbf{layer fingerprint} in the \emph{layer index} to ensure 
an identical layer is not already stored.
The layer fingerprint is calculated by hashing the layer content 
and is also used as the layer Id.
After that, \sysname replicates the layer $r$ times across the P-servers
and submits the remaining $R-r$ layer replicas to the D-servers. Those replicas are
temporarily stored in the layer stage areas of the D-servers. \sysname uses
two consistent hashing rings to for each cluster to pick the target servers for
a layer. Once the replicas
have been stored successfully, \sysname notifies the client of the request completion.
%
%\LR{How are the target D- and P-servers selected? Through consistent hashing?} 
%\NZ{In our current implement, we distribute layers to D- and P-servers
%by using two CHT rings: a ring of D-server and a ring of P-servers.}

% and acknowledges back to the client.
%
%
%After the layer is deduplicated, it will be removed from stage area.
%\dedupname system~initiates layer deduplication process only if 
%layer deduplication will achieve significant space savings and 
%the process won't impact foreground requests. 
%Sepcially, layer deduplication process is triggered when
%the layer dataset $S$ is greater than a predefined threshold $\theta_{s}$ and 
%the registry traffic $RPS$ ( i.e., requests per second) is lower than $\theta_{RPS}$. Thus, layer deduplication process runs periodically.
%The process always stars with the cold layers that haven't been
%access for a long time.
%
% After D-servers receive the layer replicas,
% replicas will be temporally saved into layer stage areas as shown in Figure~\ref{fig:sift}.
%
\sysname then initiates the deduplication process, which runs \emph{off-line} to not affect \texttt{push}
latencies. Deduplication is only performed once by a single D-server for a single layer replica
and the resulting unique files are then replicated to achieve the desired replica count.
%which collaborates with the metadata database
%to remove redundant files from layers. 
%Note that only one layer replica is selected from the submitted $R-r$ replicas on D-servers to do layer deduplication.
After decompressing the layer, the deduplication process consists of three main steps: 
%layer decompression, 
1)~file-level deduplication;
2)~unique file replication; and
3)~layer partitioning. 
The first two steps are necessary for removing duplicate files from compressed layer tarballs
and reliably store the remaining unique files.
%
%\LR{Need to make this more clear. Are we just deduplicating one layer but then copy the
%resulting unique files to $R-r$ D-servers?}
The last step is needed
to evenly distribute unique files for a layer across D-servers to balance the I/O and computational load
during layer reconstruction while maintaining fault-tolerance.  

%After the layer replica is \emph{deduplicated}, all the backup layer replicas stored on D-servers in the layer stage area will be discarded.  

%The last two step -- file replication and layer partitioning is 
%to evenly distribute the I/O and computation load of layer restoring across multiple registry servers 
%while also maintaining a level of redundancy.  
%Note that only one layer replica is selected among $R-r$ layer replicas to do layer deduplication. % subil: NEED CLARIFICATION ON WHAT THIS MEANS
%After that, all the layer replicas will be discarded.  
%>>>>>>> ab86d0a4767bfcfcbb58baf9bd7b2e81e54d7e51
%This makes the process of layer restoring achieve the maximum parallelism for each layer to accelerate the restoring process. 
%(detailed in~\cref{subsubsec:slice-restoring}) 
%After layer deduplication, unique files are evenly distributed across multiple registry servers. 

\input{fig-dedup-partition}

%\begin{figure}[t]
	\centering
%		\begin{minipage}{0.23\textwidth}
%			\centering
%			\includegraphics[width=1\textwidth]{graphs/evaluation_hitratios.png}
%			\caption{Hit ratio.}
%			\label{fig:hitratio}
%		\end{minipage}
	%	\begin{minipage}{0.25\textwidth}
			\centering
			\includegraphics[width=0.4\textwidth]{graphs/sift-metadata.pdf}
			\caption{Metadata for deduplication.}
			%\vspace{-3pt}
			\label{fig:sift-metadata}
	%	\end{minipage}
\end{figure}
\paragraph{File-level deduplication}
%\LR{Is the following just how the .tar format is structured or are we doing anything
%additional to the headers and the content? If not, we should just mention that this is
%how tar is creating archives.}
%
%As shown in Figure~\ref{fig:dedup-partition},
Following the staging area, 
the decompressed layer tar archive is loaded to the layer deduplication process.
%In this process, 
%the layer tarball is first decompressed into a \textbf{layer tar archive}~\cite{xxx}.
%Algorithm~\ref{alg:dedup-partition} details layer deduplication and partitioning. 
%After layer decompression, 
Each file entry in the archive is represented as a \emph{file header} and the associated
\emph{file content}~\cite{xxx}.
The file header contains metadata such as file name, path, size, mode, owner information, etc~\cite{xxx}.
%
%and is needed to correctly rebuild the \emph{file header} for 
%the associated file entry in the restoring layer archive.
%
\sysname records every file header in the layer tar archive
to be able to correctly restore the complete layer archive.
%
%##### need to record all files in the tar archive, not just unique files' header but also the deduplicated
%files'headers. otherwise, how do you restore a complete tar achieve?????

After that, \sysname computes the file Id for each file by hashing the file content and 
checks, whether the file Id is already present in the file index.
If the file Id is already stored, the file content will be discarded. 
Otherwise, it will be saved physically in the D-server's file store and 
the file Id is recorded in the file index. As shown in Figure~\ref{fig:dedup-partition},
the file index maps different file Ids to the physical file replicas stored
on different D-servers.
%represented as a file location and a file size.
%File location is denoted as 
%Besides, each file Id in the file index is also mapped to its \emph{host address} with the associated \textbf{backup level}.
%Backup level denotes whether the host stores a primary backup file replica (i.e., \texttt{P}) or the $n^{th}$ backup replica (ie., \texttt{B}\emph{n}) detailed as follows.
%detailed as following.

%\LR{Here, we need to explain in more detail, what exactly a slice is (just a collection of headers?),
%how a header is linked to its file content, and how files/headers are linked to layers (through layer
%recipe? We should also talk a little bit about our metadata overhead here.}

\paragraph{Unique file replication}

%Note that, layer deduplication process only deduplicates regular files., 

After discarding duplicate files,
\sysname replicates and distributes newly added unique file replicas across D-servers.
%Hence, the layer partitions on D-servers can be used to restore a \emph{slice} of layer in parallel.
%To speedup layer restoring process,
%\sysname distributes file replicas and 
%in a way such that
%each server is able to rebuild an $\sim$equal-sized slice of the layer from its local file store.
%
%As mentioned earlier, \sysname records every file header from a layer tar archive.
During file replication, \sysname distributes the corresponding
file headers to the associated servers 
along with a content pointer pointing to the corresponding physical file.
The headers from a single D-server which refer to files from the same layer
form a \emph{slice recipe} (see Figure~\ref{fig:replication-partition}). 
%After unique file replication and layer partitioning,
After file replication, \sysname adds the new slice recipes to the metadata database.
%A slice recipe represents a layer slice and is used to construct a
%\emph{partial layer archive}.

\sysname also creates a \emph{layer recipe} for the uploaded layer and stores it in the metadata database
as shown in Figure~\ref{fig:replication-partition}.
The layer recipe records all the D-servers that store layer slices and which can act
as \emph{restoring workers}. When a layer needs to be reconstructed,
one worker is selected as the \emph{restoring master}, responsible 
for gathering all slices and rebuilding the layer~(see~\S\ref{sec:restore-desgin}).

Figure~\ref{fig:replication-partition} demonstrates an example for unique file replication.
The example assumes B-mode 1 with 3-way replication, i.e. each unique file has two replicas distributed
on two different D-servers.
The files $f1$, $f2$, and $f3$ are already stored in \sysname,
and $f1'$, $f2'$, and $f3'$ are their corresponding backup replicas.
Layer $L1$ is being pushed and contains files $f1$, $f2$, $f3$, $f4$, $f5$, and $f6$. 
%is decompressed and \emph{file-level deduplicated}.
$f1$, $f2$, and $f3$ are \emph{shared files} by $L1$ and other layers, and hence,
are discarded during file-level deduplication.
%\emph{f1, f2,} and \emph{f3} are shared by \emph{L1} and other layers, denoted as \textbf{shared files}.
The unique files $f4$, $f5$ and $f6$ are added to the system and
replicated to D-servers $A$, $B$, and $C$. We describe how the target servers are chosen in detail
in the next paragraph.

After replication, server $B$ contains $f2$, $f5$, $f1'$, and $f4'$. Together
$f2$ and $f5$ form the \emph{primary slice} of $L1$, denoted as $L1::A::P$.
This slice ID contains the layer ID the slices belongs to ($L1$), the node, which
stores the slice ($A$) and the backup level ($P$ for primary).
%\todo{explain this notation}.
%\NZ{This slice Id contains the layer Id it belongs, the host server, and the backup level of this slice.
%This slice Id notation largely simplifies layer recipe and facilitate layer restoring.
%The restoring master only needs to fetch a small layer recipe, get the restoring workers,
% and decide which backup level slices are needed for layer restoring.
% After that, the master sends out simple layer restoring instructions to the workers.
% The layer restoring instruction only contains the layer Id and the backup level.
%The workers will generate a slice Id by using the layer Id and backup level
%along with its own address, which holds the slice stored locally.}
%
%shared file replica \emph{f2} and 
%unique file replica \emph{f5} stored on \emph{B} comprise a slice of \emph{L1},
%denoted as a \textbf{primary slice} of \emph{L1} (ie., \emph{L1::A::P}).
The two backup file replicas 
$f1'$ and $f4'$ on $B$ form the \emph{backup slice} $L1::B::B$.
During layer restoring, $L1$ can be restored by using any combination of primary and backup slices
to achieve maximum parallelism.
%
%\LR{Why are \emph{f4'} and \emph{f8'} only a backup slice?}
%\LR{Instead of putting the fX in emph environments, we should use math mode and put numbers in the index, e.g.
%$f_{4}'$.}

%=======
%After discarding duplicate files
%, the deduplication process distributes the layer's remaining files across different registry servers
%so that each server is able to build an $\sim$equal-sized % subil: is the \sim necessary?
% slice of the layer from its local file store when needed.
% We define the collection of files that belong to a layer as a \emph{slice} of that layer stored on the registry server. 
%A server stores slices for many layers, and a layer is composed of slices stored on multiple servers, which allows restoring a layer in parallel. 
%Finally, \sysname will add slice recipes and layer recipe to metadata database as shown in Figure~\ref{fig:dedup-partition}.
%
%Figure~\ref{fig:replication-partition} shows an example of file replication and layer partition.
%Each file has two replicas distributed on different servers,
%\emph{r1, r2, r3,} and \emph{r4} are four primary file replicas that were already stored in the system.
%and \emph{r1', r2', r3',} and \emph{r4'} are their associated backup file replicas. 
%A layer \emph{L} that contains files \emph{r1, r2, r3, r4, r5, r6, r7,} and \emph{r8} is pushed.
%Files \emph{r1, r2, r3,} and \emph{r4} are deduplicated
%while new files
%\emph{r5, r6, r7,} and \emph{r8} will be added to the system, 
%As shown, \emph{r5, r6, r7,} and \emph{r8} are replicated and distributed to registry \emph{A, B, C,} and \emph{D} 
%as primary file replicas and backup file replicas.
%Consequently,
%primary file replicas \emph{r1} and \emph{r5} stored on registry \emph{A} comprise one of layer \emph{L'}s slices, 
%denoted as \textbf{primary slice}:
%\emph{layerid:A}.
%The two backup file replicas \emph{r4'} and \emph{r8'} stored on registry \emph{A} comprise a \textbf{backup slice} for the primary slice stored 
%on registry \emph{D} (i.e., \emph{r4} and \emph{r8}).
%>>>>>>> ab86d0a4767bfcfcbb58baf9bd7b2e81e54d7e51

%primary file replicas
\begin{figure}[t]
	\centering
	\centering
	\includegraphics[width=0.5\textwidth]{graphs/replication.pdf}
	\caption{Layer replication and partitioning. \LR{The text is very small}}
	\label{fig:replication-partition}
\end{figure}
%\begin{algorithm}
\scriptsize 
	\caption{Layer deduplication and partition}
	\label{alg:dedup-partition}
	\KwIn{\\	
			$FileIndex$: File index. \\
	}

	\KwOut{\\
			$LayerRecipe$: Layer recipe for layer $L$.\\
			$SliceRecipes$: Slice recipes for layer $L$' slices.\\
	}

	archive $\leftarrow$ \texttt{Decompress} layer $L$ \\

	\ForEach{header, content in archive} 
	{ 
		Id $\leftarrow$ \texttt{Hash} content \\
		%destHeader $\leftarrow$ header \\
		\eIf{ Id in FileIndex}{
			host  $\leftarrow$ $FileIndex$[Id].host \\
			src $\leftarrow$ $FileIndex$[Id].header \\
			{\tiny\texttt{/* SliceRecipe is identified by `layerid::host'}}
			SliceRecipe[$L::host$] $\leftarrow$  \texttt{Add} (header, src) \\
			{\tiny\texttt{/* Skip redundant file.  }}\\
		}{
			{\tiny\texttt{/* create file with content in file store.  }}\\
			file $\leftarrow$ \texttt{Create and write} content \\
			src $\leftarrow$ \texttt{Stat} file \\
			entries $\leftarrow$ (header, src)  \\
		}
	}
	
	\Do{entries}{
		{\tiny\texttt{/* get file entry with maximum file size. }}\\
		maxFile $\leftarrow$ \texttt{Max} entries \\
		{\tiny\texttt{/* get smallest slice. }}\\
		minSlice $\leftarrow$ \texttt{Min} SliceRecipes \\
		{\tiny\texttt{/* assign biggest file to smallest slice. }}\\
		minSlice $\leftarrow$ \texttt{Add} maxFile \\
		{\tiny\texttt{/* Set maxFile to FileIndex}}\\
		{\tiny\texttt{/* Remove maxFile from entries. }}\\
	}
	{\tiny\texttt{/* Send slices.}}\\
	{\tiny\texttt{/* Set SliceRecipes.}}\\
	{\tiny\texttt{/* Include all slices' hosts to workers.}}\\
	{\tiny\texttt{/* Select biggest slice' host as master.}}\\
	{\tiny\texttt{/* Set LayerRecipe.}}\\
	
	
		
\end{algorithm}




%<<<<<<< HEAD
%Then, the process sends slices to the corresponding registries.
%Slice is identified by a simple combination of the layer id and its host address 
%with the associated backup level,
%denoted as $layerid::host::backuplevel$. % as shown in Algorithm~\ref{alg:dedup-partition}.
%As shown in Figure~\ref{fig:replication-partition},
%the primary slice stored on \emph{A} is denoted as \emph{L::}\emph{A::}\texttt{P}
%while the rest backup slice is denoted as \emph{L::}\emph{A::}\texttt{B2}.

%=======
%Algorithm~\ref{alg:dedup-partition} details layer deduplication and partitioning algorithm.
%After layer decompression, 
%each file entry in the layer archive % subil: what is the layer archive referring to?
% is represented as a \textbf{file header} and a \textbf{file content}.
%The file header contains the file name, path, size, mode, owner information, etc. % subil: should elaborate on structure of the header, how is it constructed?
%The file header is needed to rebuild the \emph{file header} % subil: how are the two file headers different?
%for 
%the associated file entry in the layer archive.
%
%%During layer deduplication process,
%\sysname records each file entry's file header and 
%calculates the file id by hashing the file content.
%As mentioned earlier, the file index maps a \emph{file id} to its associated physical file that is stored in the file store.
%To address a physical file,
%each file id in the file index holds its 
%%denoted by a combination of its 
%\emph{host address} and \textbf{file status} of the physical file which mainly contains file name, path, and size.
%As shown in Algorithm~\ref{alg:dedup-partition}, 
%if a file's id already exists in file index (i.e. it's present in a registry server), 
%this file will be added to its host's corresponding slice for that layer. 
%Otherwise,
%\sysname stores the file content as a physical file in the file store
%and retrieves the file status of the physical file.
%The file index is updated accordingly.
%
%The slice recipe is identified by a simple combination of the layer id and its host registry address,
%denoted as $layerid::host$ as shown in Algorithm~\ref{alg:dedup-partition}.
%A slice recipe represents a layer partition and
%is used to construct a partial layer,
%>>>>>>> ab86d0a4767bfcfcbb58baf9bd7b2e81e54d7e51
%which
%records each file entry' $header$ in the layer archive partition and 
%its corresponding physical file's status . 
%In this case, 
%each file is represented as a $header$ in the layer archive and
%its corresponding physical file's status $src$. 
%
%During layer deduplication process,
%After unpacking the layer archive, 
%
%Note that destHeader can be used to rebuild the \textbf{file header} for 
%associated file entry in the layer archive.
%
 %its host' slice
%to get identical files' server addresses.

%\LR{The following needs to be reformulated to be more clear.}
%
\paragraph{Layer partitioning}
To improve reconstruction times, it is important that different layer slices
are equally sized and evenly distributed across D-servers.
The example in Figure~\ref{fig:replication-partition}
shows such a perfect scenario, in which the files of layer $L1$ are evenly partitioned across all D-servers.
%since its shared files and unique files are evenly distributed to D-servers. 
However, in general, it is not straightforward to evenly distribute shared files for different
layers as their placement is fixed by the first layer, which was pushed to the registry that
contained these files.
%\LR{Provide more details. Why is that so? Is this assuming consistent hashing?}
%\NZ{Let's consider an empty storage system, the first layer can be easily evenly partitioned
% to servers because no shared files are stored.
%After the first layer is partitioned and stored,
%consider that the second layer is pushed,
%the second layer has some same files with the first layer and
%these shared files are already distributed to servers.
%These shared files are distributed in a way such that the first layer can be evenly partitioned.
%But they might not be evenly distributed for the second layer because
%when we distribute files for first layer we dont know the subsequent layers and
%we dont know which files will be shared with the subsequent layers.
%So the shared files' layout will affect the subsequent layers' partition.}
%
To evenly distribute and partition a layer's unique and shared files
%, shared and unique files,
%
%To distribute the newly added unique files' primary replicas,
\sysname employs a greedy packing algorithm.
%, which is constrained by  with shared file constraints.
%distribute the newly added unique files in a way such that
%each D-server stores relatively equal-sized slice for the layer.
%\LR{What exactly is meant by `shared file constraints'. Can we explain a bit more?}
%\NZ{It means before partition the layer, some of its containing files (shared files)
%are already allocated to the servers.
%We dont want to migrate those files to make all its containing files evenly distributed.
%So I called it a constrain.
%It means when we partition layers, we maintain its containing shared files that are already allocated
%and stored on servers as it is. 
%Through only distribute the newly added unique files,
%we achieve a relatively even partition for a layer.}

We first consider the simpler case in which each file only has
a single replica. During layer partitioning, \sysname does not migrate shared files that already
have been stored on D-servers to reduce I/O overhead.
%
%\LR{The following is unclear, why does \sysname partition shared files for $L1$? Aren't they already there?}
%\NZ{I think it's allocating not partitioning.}
%
%\sysname first partitions $L1$ shared files to their host servers.
%\sysname first determines the location of $L1$'s shared files in the cluster.
%For example, in Figure~\ref{fig:replication-partition}, shared files $f1$, $f2$, and $f3$ are located
%on D-servers $A$, $B$ and $C$ respectively. It then computes the total size of the shared files
%for each affected D-servers.
\sysname first computes the total size of the layer's existing shared files for each D-server (this could
potentially be 0 if a D-server does not store any shared files for the layer).
Next, \sysname starts to assign unique files to D-servers following a greedy packing strategy.
It assigns the biggest unique file to the smallest partition until all the unique files are assigned.
%In the example, the shared files are evenly partitioned to all three D-servers,
%and hence, the three newly added unique files are also evenly assigned to the three servers.
%Otherwise, the biggest unique files will be assigned to the smallest partitions.
%\LR{As a general note, it seems like we generate quite a bit of metadata to keep track
%of the slices etc. Are we planning to evaluate how much metadata we generate?}
%\LR{Here, we need to talk about how we are planning to address the problem of metadata management as
%we are manually assigning files to servers and hence, have to store the location of each file in an index.
%How do we scale? What if we are dealing with billions of files?}
%\NZ{We just add a server address for each file Id in the file index.
%As shown in implementation section,
%we store the unique files as: ‘rootdir/docker/registry/v2/uniquefiles/sha256/
% <first two hex bytes of file digest> /<hex file diges>t’ so that given a file digest, 
% we can build the file path to save metadata information.}
Finally, \sysname updates the file index with the newly added unique files.

%\LR{The following is unclear and needs a better explanation. Are the new primary replicas
%assigned to servers which do not yet store shared files for the new layer? What if all
%servers already store at least one shared file for the layer? How are the backup replica
%servers picked?
%}
%\NZ{The shared files also have primary replica and backup replicas.
%Simply, we distributed the primary replica of newly added unique files
%according to the distribution of primary replicas of shared files.
%And we distributed the secondary replicas of newly added unique files
%according to the distribution of secondary replicas of shared files to maintain a relatively even partition.
%Moreover, we also make sure that the secondary replicas of a newly added unique file
%is not stored on the same server with its primary replica.
%So when we distribute the secondary replicas, we need to consider the 
%secondary replicas of shared files and the previously allocated primary replicas.
%}
%
In the case where a file has more than one replica, \sysname performs the
above described partitioning \emph{per replica}. That means that it first assigns
the primary replicas of the new unique files to D-servers according
to the location of the primary replicas of the existing shared files. It then assigns the secondary
replicas according to the location of the existing secondary replicas and so on.
Additionally, \sysname also ensures that two replicas of the same file are never
placed on the same node.

%Next, we consider the case that each file has multiple replicas. The file
%index contains the location of the primary and backup replicas for each shared file.
%%For each shared file, its file Id points to multiple file replicas in the file index as
%%shown in Figure~\ref{fig:dedup-partition}.
%When adding new unique files, \sysname first distribute the files
%according to the shared files' primary file replicas' partitions. After that,
%\sysname assigns the replicas of unique files onto different servers with their primary
%replicas.

%After layer partition,.
%All the primary slices' hosts are denoted as layer restoring \textbf{primary workers}. 
%Next, \sysname stores slice recipes in metadata database after successfully
%sending slices to their corresponding workers.
%\sysname selects the biggest slice' host as layer restoring \textbf{primary master}.
%Layer recipe records the primary workers and master information for layer restoring.
%After that, \sysname~distributes primary slices' corresponding backup slices.
%Their hosts are denoted as \textbf{backup workers}.
%Finally, \sysname stores layer recipe in metadata database.
%Layer recipe 

%Here, we use the metadata database as a distributed lock for file index.
%File index sets a file id to hold  only if the file id does not exists.

%it uses a weighted round robin algorithm to distribute newly added 
%unique files to the registry cluster. These servers that already contains this layer's identical files
%are assigned a lower weight. 
%This is to ensue that different servers maintain same amount of files that are needed to restoring equal sized 
%slices for this layer.
%Deduplication process also 
%updates the \emph{file index} with the newly added unique files' fingerprints and host addresses.   
%Then, it calculates the slice fingerprints and creates a slice recipe for each slice.
%Slice recipe contains partial of the layer tarball's directory tree structure, file fingerprints, file name along with file path in the tarball, and file metadata information 
%(such as permissions and creation date), which are needed for restoring a slice for the layer. 

%After that
%After removing the duplicate files,
%Then, it
%distributes the \emph{deduplicated slices} evenly to the registry cluster by using round-robin, and,
%In the end, deduplication process creates a layer recipe which contains its slices' fingerprints and 
%based on layer recipe, it updates image manifests.
%The layer's tarball and file duplicates are removed after deduplication.

%<<<<<<< HEAD
%\subsection{Layer restoring}
%\label{sec:restore-design}
%%\paragraph{Parallel slice restoring}
%%\label{subsubsec:slice-restoring}
%
%

\begin{figure}[t]
	\centering
	\centering
	\includegraphics[width=\columnwidth]{graphs/sift-layer-construct-new.pdf}
	\caption{Parallel streaming layer construction.}
	\label{fig:construct}
\end{figure}

\begin{figure}[t]
	\centering
	\centering
	\includegraphics[width=\columnwidth]{graphs/sift-cache-bigger.pdf}
	\caption{Tiered storage architecture.}
	\label{fig:tieredstore}
\end{figure} 
%
%When a~\texttt{pull} layer request is received and its associated 
%P-servers fail, 
%\sysname will first search layer diskcache on D-servers.
%If not found,
%\sysname will rebuild the layer from file store according to its layer recipe. 
%%the \dedupname~system 
%%first prepares a directory structure for the slice, based on the slice recipe.
%%Then, it copies the files into the directory tree.
%%Next, it compresses the slice's directory tree into a slice tarball,
%%and directly sends it back to the client.
%
%\paragraph{Parallel layer construction}
%
%When a~\texttt{pull} layer request is received, 
%\sysname will update \emph{ULmap}. 
%ULmap records user access status,
%which maps a \textbf{user id} to its accessed layers with its corresponding access count,
%where user id is defined as client request address.
%
%Upon a \texttt{pull} layer request fails on P-servers,
%\sysname initiates a layer restoring process on D-servers for it.
%The process has two parts: slice constructor and layer constructor.
%Figure~\ref{fig:restoring} shows an example of parallel layer restoring when a \texttt{pull} layer request failure happens.
%First, 
%layer constructor fetches the layer recipe from metadata database.
%As shown in $L1$'s layer recipe, 
%the primary restoring workers contains registry $A$, $B$, and $C$.
%Since registry $A$ is the primary restoring master,
%\emph{A} sends ``\texttt{Get primary slice}" requests to its peer workers: $B$ and $C$.
%After a \texttt{Get primary slice} request is received, 
%$B$ and $C$ start primary slice construction and return a primary slice back to $A$ respectively.
%Meanwhile $A$ instructs local slice constructor to rebuild a primary slice for $L1$.
%Slice constructor first gets its associate slice recipe (``$L1::P:A$") from metadata database 
%keyed by a combination of  layer id, slice type (i.e., primary $P$ or backup $B \#$) and registry address.
%Then, based on slice recipe,
%slice constructor fetches files pointed by $Srcs$
%and builds a slice archive.
%After receiving all slices, layer constructor concatenates them into a compressed layer
%and sends back the client. 
%
%\paragraph{Streaming layer construction}
%Figure~\ref{fig:construct} details the stateless streaming layer construction process.
%First, slice constructor loads file in parallel from file store based on the slice recipe.
%Each file is written to an archive buffer asynchronously.
%Before writing file content to the archive buffer, slice constructor first builds and writes its
%associated 
%file header into the archive buffer according to the slice recipe. 
%After archiving all the files in the slice,
%the archive buffer will be divided into few chunks and compressed in parallel,
%then concatenated into a single compressed slice stream.
%Through network transfer, multiple slice streams will be concatenated into a single layer.
%No intermediate file will be created or stored on disk. % subil: whoa!
%\sysname uses a small file inmemory cache to reduce file I/Os. 
%File cache uses Adaptive Replacement(ARC)~\cite{xxx}.
%
%%headers according to $Dests$,
%%after that it writes file contents into the archive
% 
%%Slice restoring process has four suboperations: 
%%slice recipe lookup,
%%slice file copying,
%%slice compression, and
%%slice network transfer. 
%%To measure the overhead for each suboperation, 
%%we implemented layer deduplication and parallel slice
%%restoring on a 4-node registry cluster. 
%%We first warmup the cluster by pushing 200 layers to the cluster
%%and initiating layer deduplication process.
%%The layers were randomly selected from our layer dataset detailed in xxx limited to 50MB.
%%After finishing layer deduplication,
%%we sent 400 \texttt{pull slice} requests to the cluster with 10 \texttt{pull slice} requests issued at a time.
%%Figure~\ref{fig:slice-restoring-breakdown} shows the CDFs of the latencies for each suboperation.
%%We see that across the four suboperations,
%%the duration for slice compress is the shortest.
%%Slice compression only took less than 0.001 s because a slice is a smaller unit. 
%%The next shortest suboperation is network transfer since we pulled layer slice through Ethernet.
%%90\% of slice recipe lookups took less than 0.1 s while 
%%the highest slice recipe lookup duration almost reaches 0.8 s,
%%which is caused by high concurrent lookup requests 
%%%(note that we use redis to store metadata \NZ{use mongodb instead}).
%%The most time consuming suboperation is slice file copying, which involves 
%%copying all 
%%the files that belong to the slice to their destination directory based on the slice recipe.
%%Note that we implemented a thread pool on each registry server to read files in parallel
%%and write data in RAMdisk to reduce disk IOs.
%%40\%of slice file copying duration is greater than 1 s and 
%%10\% of slice file copying duration is higher than 10 s.
%%This is because bigger slices contains more files and requires more disk IOs.
%%The overhead of slice copying can be largely mitigated for a large-scale registry cluster
%%since the size of slice roughly equals to $S_{l}/N$, where $S_{l}$ denotes the layer size and $N$ is size of registry cluster.
%%However, it could be a bottleneck for slice restoring on a small-scale registry cluster.
%%%and slice file copying duration depends on slice size.
%%
%%%\begin{algorithm}
\scriptsize 
	\caption{File cache assisted slice restoring}
	\label{alg:prefetch}
	\KwIn{\\
		$\theta_{rsfc}$: Slice restoring latency threshold. \\
		$s$: Slice to be restored. \\
	}

	\SetKwFunction{Fsub}{Restore}
	\SetKwProg{Fn}{Function}{:}{}

	\Fn{\Fsub{s}}{
		%{\tiny\texttt{/* Otherwise, it's a repull layer miss   /}}\\
		\eIf{files in s are cached in file cache}
		{
			slice $\gets$ \texttt{RestoreSlice} \emph{s} \texttt{From} \emph{file cache + disk} 
		}
		{
			slice, $D_{rs}$ $\gets$ \texttt{RestoreSlice} \emph{s} \texttt{From} \emph{disk} \\
			\If{ $D_{rs} > \theta_{rsfc}$} 
			{ 
				\emph{file cache} $\gets$ \texttt{Cache} \texttt{Subsetof} \emph{s.files}
			}		
		}
	}

\end{algorithm}



%%
%%To reduce slice file copying overhead,
%%\sysname~\filecachename~temporally cache a subset of unique files for bigger and popular slices that have a high slice restoring latency, ie., $D_{rs} > \theta_{rsfc}$, 
%%where $D_{rs}$ is the slice restoring latency and $\theta_{rsfc}$ is the restoring latency threshold for 
%%caching
%%a subset of files from the slice to help improve its restoring performance as shown in Algorithm~\ref{alg:file-cache}.
%%Upon a \texttt{pull slice} request for those slices, 
%%\dedupname~ system fetches a subset of its containing files from \filecachename~and
%%the remaining files from disk for slice restoring.
%%
%%To identify which slices have a high slice restoring latency,
%%\dedupname~system monitors slice restoring performance and 
%%maintains a restoring performance  profile for each slice that has been restored,
%%% as shown in Figure~\ref{fig:xxx},
%%which contains the latency breakdown of slice restoring
%%% (,and a decompression latency updated by layer decompression process) 
%%and its containing files' sizes.
%%All the slice restoring performance profiles are also stored in distributed  databases,
%% and addressed by slice digests. 
%%To estimate the restoring latency for a slice $i$ that hasn't been restored, 
%%\dedupname system~first lookups the slice restoring performance profiles by slice size,
%% then selects a slice $x$ that is most similar in size to $i$,
%% and estimates $i$'s restoring latency as: 
%% $D_{rs}(i) \approx D_{rs}(x) + \Phi_{rs}(\Delta_{S})$,
%% where $\Delta_{S}$ is the size different between two slices.
%% $\Phi_{rs}(\Delta_{S})$ denotes a slice restoring latency function of slice size variation.
%%  $\Phi_{rs}(\Delta_{S})$ is generated by using linear regression~\cite{xxx}.
%% %$\varepsilon_{rs}$ is the standard error of restoring latency estimation for the layers similar in size.
%%If the estimated slice restoring $D_{rs}(i) > \theta_{rs}$,
%%then, \dedupname~lookups the slice restoring performance profiles by slice size,
%%selects a slice $y$ that has a acceptable restoring latency and
%%most similar in size to $i$.
%%Next, \dedupname system~caches a subset of files $F$ for slice $i$, so that
%%$D_{rs}(i) - \Phi_{rs}(\Sigma_{S}(F)) \approx D_{rs}(y)$,
%%where $\Sigma_{S}(F)$ is the sum size of files in $F$.
%%
%%Note that \filecachename~size is limited so that \filecachename~only caches 
%%subsets of files for big slices that belongs to popular layers.
%%%that will be accessed later. 
%%\cref{sec:cache-design} will describe how to determine popular layers based on user access patterns.
%%Note that the slices for the same layer have similar sizes, restoring latencies, and popularity 
%%because of unique file
%%distribution. 
%%Thus, once a layer is determined as popular layer, 
%%\dedupname~will cache similar amount of files for its slices.
%%Note that all the files in file cache are unique and can be shared for restoring different slices.
%%
%%For on-premise or private registry cluster, the network transfer speed is usually faster than remote cloud.
%%Thus, slice compression is less important for medium to small size slices, 
%%especially for the slices that have a high decompression latency, 
%%i.e., $D_{stt} < \theta_{stt}$ and $D_{dc} > \theta_{dc}$, where $D_{stt}$ and $D_{dc}$ denote slice transfer duration
%%and decompression duration respectively; 
%%$\theta_{stt}$ and $\theta_{dc}$ denote thresholds for them respectively.
%%Consequently, \dedupname system~only archives these slices without compressing them and directly sends
%%these archival files back to the clients to eliminate clients' decompression latency.
%
%
%
%=======
%>>>>>>> bbd2a10a262cf8b3e024fa09e3b0f5ca8afb21c3


\subsection{Restoring layers}
\label{sec:restore-desgin}
%\paragraph{Parallel slice restoring}
%\label{subsubsec:slice-restoring}



\begin{figure}[t]
	\centering
	\centering
	\includegraphics[width=\columnwidth]{graphs/sift-layer-construct-new.pdf}
	\caption{Parallel streaming layer construction.}
	\label{fig:construct}
\end{figure}

\begin{figure}[t]
	\centering
	\centering
	\includegraphics[width=\columnwidth]{graphs/sift-cache-bigger.pdf}
	\caption{Tiered storage architecture.}
	\label{fig:tieredstore}
\end{figure} 

%When a~\texttt{pull} layer request is received and its associated 
%P-servers fail, D-servers will rebuild a layer for the request.
%\sysname will first search layer preconstruct cache on D-servers.
%If not found,
%\sysname will rebuild the layer according to the layer recipe. 
%the \dedupname~system 
%first prepares a directory structure for the slice, based on the slice recipe.
%Then, it copies the files into the directory tree.
%Next, it compresses the slice's directory tree into a slice tarball,
%and directly sends it back to the client.

%\LR{Are we only considering the case when a P-server fails or
%are we also considering, if a P-server gets overloaded?}

To restore a layer, each D-server in \sysname uses  a slice and a
layer constructor. Considering Figure~\ref{fig:replication-partition} as
an illustration, the restoring process works as follows:

First, the layer constructor fetches the layer recipe from the metadata database.
%According to $L1$'s layer recipe, 
%(Figure~\ref{fig:replication-partition})
%Figure~\ref{fig:construct}
According to $L1$'s layer recipe,  the restoring workers are D-servers $A$, $B$, and $C$. The node with the largest slice is picked as the restoring master ($A$ in the example).
Since $A$ is the restoring master
%\LR{Explain how the master is picked},
%\NZ{We select the server which stores the biggest slice as master to reduce restoring overhead.}
it sends \texttt{GET slice} requests for the primary slices to $B$ and $C$.
%Note that the master first tries to rebuild the layer from primary slices.
If a primary slice is missing, the master locates its corresponding backup
slice and send a \texttt{GET slice} request to the corresponding D-server.

%\paragraph{Parallel slice construction}

%Upon a \texttt{pull} layer request fails on P-servers,
%\sysname initiates a layer restoring process on D-servers for it.
%which involves two modules: slice constructor and layer constructor. 
After a \texttt{GET slice} request has been received, 
$B$'s and $C$'s slice constructors start rebuilding their primary slices and send them to $A$ as shown in Figure~\ref{fig:construct}.
Meanwhile $A$ instructs its local slice constructor to restore its primary slice for $L1$.
%
To construct a layer slice, a slice constructor first gets the associate slice recipe
from the metadata database. The recipe is keyed by a combination of layer Id, host address and requested
backup level, i.e., $L1::A::P$.
%\LR{More details here. Why is that they key? Why is the backup level relevant?}.
%\NZ{The slices at the same backup level can build a complete layer.}
%
Based on the recipe, the slice constructor creates a slice tar file by concatenating each file header and the corresponding
file contents; it then compresses the slice and passes it to the master.
%rebuilds each file entry in the slice tar archive according to the slice recipe.
%
The master concatenates all the compressed slices into a single
layer compressed tarball and sends it back to the client.
% as shown in Figure~\ref{fig:construct}.
%As a result no intermediate file will be created or stored on disk. % subil: whoa!


%\paragraph{Optimizations}
The layer restoration performance is critical to keep \texttt{pull} latencies low. Hence,
\sysname uses several additional optimizations to further speed up the process:
%Besides parallelizing layer reconstruction across D-servers, \sysname also
it parallelizes slice reconstruction on a single node and
avoids generating intermediate files on disk to reduce disk I/O.
%
%To avoid generating any intermediate files stored on disk to reduce disk I/Os
%and improve restoring performance, the slice restoring process is a stateless streaming
%slice construction process.
%
%All the involved data processing operations are preceded in memory as streaming. 

%Specifically, a slice constructor first reads each entry in the slice recipe
%and gets each header and its corresponding content pointer.
%
%Then, the slice constructor reads the corresponding physical files addressed by the content
%pointers in parallel from the D-server's file store
%
%\LR{The following is unclear. What does ``asynchronously'' mean in this case? Is every
%file written to its own buffer in parallel and then combined in a single archive buffer?}
%\NZ{It's synchronously.}
%and writes each file and it's corresponding header to a slice archive buffer in memory.
%Before writing file content to the archive buffer, slice constructor first writes its
%associated file header into the archive buffer according to the slice recipe.
%
%After archiving all the files along with their headers in the slice archive buffer,
%the archive buffer will be divided into several chunks, compressed in parallel,
%concatenated again into a single compressed slice, and sent to the master.
%
%Then, it writes the header in the slice archive.
%After that it put the associated file content in the slice archive
%by reading the corresponding physical file addressed by the content pointer.
%fetches files pointed by content
%and builds a slice archive.
%
%Next, the slice constructor sends the slice stream to the layer constructor on restoring master
%via network.
%
%first, slice constructor loads file in parallel from file store based on the slice recipe.
% along with their corresponding headers saved in slice recipe.
%
%\LR{The following paragraph is a little bit out of context. Why are we suddenly talking about
%stateless streaming layer construction? Need to tie it in better with the rest of the text and
%motivate our design decisions better.}
%
%\paragraph{Streaming layer construction}
%\LR{What exactly do you mean by ``Through network transfer''? Are the different slices kept inmemory at the master for concatenation after they have been received?}
%\NZ{Yes. But before that, these slices need to be transferred through network to master.}
%After receiving all slices, layer constructor on \emph{A} will concatenate all slices into a compressed layer
%. 
%Overall, the layer restoring process is a stateless streaming layer construction process.
%
%All the involve data processing operations are preceded in memory as streaming 
%without creating any intermediate files stored on disk to reduce disk I/Os and improve
%performance.

%Furthermore, \sysname also provisions a small in-memory file cache on each D-server to cache hot files.
%This reduces disk I/O during restoration even further. The file cache uses Adaptive
%Replacement (ARC)~\cite{xxx} \todo{fill in more details on how the cache works, how big it is, etc.}
%%
%\LR{It would be good to have a graph in the evaluation, showing how each of those optimizations
%reduces reconstruction time.}
%\NZ{The file cache hit ratio is very low, around 30\%
%because less files are shared among the 
%concurrent incoming requested layers.
%We can remove file cache from design.}

%headers according to $Dests$,
%after that it writes file contents into the archive
 
%Slice restoring process has four suboperations: 
%slice recipe lookup,
%slice file copying,
%slice compression, and
%slice network transfer. 
%To measure the overhead for each suboperation, 
%we implemented layer deduplication and parallel slice
%restoring on a 4-node registry cluster. 
%We first warmup the cluster by pushing 200 layers to the cluster
%and initiating layer deduplication process.
%The layers were randomly selected from our layer dataset detailed in xxx limited to 50MB.
%After finishing layer deduplication,
%we sent 400 \texttt{pull slice} requests to the cluster with 10 \texttt{pull slice} requests issued at a time.
%Figure~\ref{fig:slice-restoring-breakdown} shows the CDFs of the latencies for each suboperation.
%We see that across the four suboperations,
%the duration for slice compress is the shortest.
%Slice compression only took less than 0.001 s because a slice is a smaller unit. 
%The next shortest suboperation is network transfer since we pulled layer slice through Ethernet.
%90\% of slice recipe lookups took less than 0.1 s while 
%the highest slice recipe lookup duration almost reaches 0.8 s,
%which is caused by high concurrent lookup requests 
%%(note that we use redis to store metadata \NZ{use mongodb instead}).
%The most time consuming suboperation is slice file copying, which involves 
%copying all 
%the files that belong to the slice to their destination directory based on the slice recipe.
%Note that we implemented a thread pool on each registry server to read files in parallel
%and write data in RAMdisk to reduce disk IOs.
%40\%of slice file copying duration is greater than 1 s and 
%10\% of slice file copying duration is higher than 10 s.
%This is because bigger slices contains more files and requires more disk IOs.
%The overhead of slice copying can be largely mitigated for a large-scale registry cluster
%since the size of slice roughly equals to $S_{l}/N$, where $S_{l}$ denotes the layer size and $N$ is size of registry cluster.
%However, it could be a bottleneck for slice restoring on a small-scale registry cluster.
%%and slice file copying duration depends on slice size.
%
%%\begin{algorithm}
\scriptsize 
	\caption{File cache assisted slice restoring}
	\label{alg:prefetch}
	\KwIn{\\
		$\theta_{rsfc}$: Slice restoring latency threshold. \\
		$s$: Slice to be restored. \\
	}

	\SetKwFunction{Fsub}{Restore}
	\SetKwProg{Fn}{Function}{:}{}

	\Fn{\Fsub{s}}{
		%{\tiny\texttt{/* Otherwise, it's a repull layer miss   /}}\\
		\eIf{files in s are cached in file cache}
		{
			slice $\gets$ \texttt{RestoreSlice} \emph{s} \texttt{From} \emph{file cache + disk} 
		}
		{
			slice, $D_{rs}$ $\gets$ \texttt{RestoreSlice} \emph{s} \texttt{From} \emph{disk} \\
			\If{ $D_{rs} > \theta_{rsfc}$} 
			{ 
				\emph{file cache} $\gets$ \texttt{Cache} \texttt{Subsetof} \emph{s.files}
			}		
		}
	}

\end{algorithm}



%
%To reduce slice file copying overhead,
%\sysname~\filecachename~temporally cache a subset of unique files for bigger and popular slices that have a high slice restoring latency, ie., $D_{rs} > \theta_{rsfc}$, 
%where $D_{rs}$ is the slice restoring latency and $\theta_{rsfc}$ is the restoring latency threshold for 
%caching
%a subset of files from the slice to help improve its restoring performance as shown in Algorithm~\ref{alg:file-cache}.
%Upon a \texttt{pull slice} request for those slices, 
%\dedupname~ system fetches a subset of its containing files from \filecachename~and
%the remaining files from disk for slice restoring.
%
%To identify which slices have a high slice restoring latency,
%\dedupname~system monitors slice restoring performance and 
%maintains a restoring performance  profile for each slice that has been restored,
%% as shown in Figure~\ref{fig:xxx},
%which contains the latency breakdown of slice restoring
%% (,and a decompression latency updated by layer decompression process) 
%and its containing files' sizes.
%All the slice restoring performance profiles are also stored in distributed  databases,
% and addressed by slice digests. 
%To estimate the restoring latency for a slice $i$ that hasn't been restored, 
%\dedupname system~first lookups the slice restoring performance profiles by slice size,
% then selects a slice $x$ that is most similar in size to $i$,
% and estimates $i$'s restoring latency as: 
% $D_{rs}(i) \approx D_{rs}(x) + \Phi_{rs}(\Delta_{S})$,
% where $\Delta_{S}$ is the size different between two slices.
% $\Phi_{rs}(\Delta_{S})$ denotes a slice restoring latency function of slice size variation.
%  $\Phi_{rs}(\Delta_{S})$ is generated by using linear regression~\cite{xxx}.
% %$\varepsilon_{rs}$ is the standard error of restoring latency estimation for the layers similar in size.
%If the estimated slice restoring $D_{rs}(i) > \theta_{rs}$,
%then, \dedupname~lookups the slice restoring performance profiles by slice size,
%selects a slice $y$ that has a acceptable restoring latency and
%most similar in size to $i$.
%Next, \dedupname system~caches a subset of files $F$ for slice $i$, so that
%$D_{rs}(i) - \Phi_{rs}(\Sigma_{S}(F)) \approx D_{rs}(y)$,
%where $\Sigma_{S}(F)$ is the sum size of files in $F$.
%
%Note that \filecachename~size is limited so that \filecachename~only caches 
%subsets of files for big slices that belongs to popular layers.
%%that will be accessed later. 
%\cref{sec:cache-design} will describe how to determine popular layers based on user access patterns.
%Note that the slices for the same layer have similar sizes, restoring latencies, and popularity 
%because of unique file
%distribution. 
%Thus, once a layer is determined as popular layer, 
%\dedupname~will cache similar amount of files for its slices.
%Note that all the files in file cache are unique and can be shared for restoring different slices.
%
%For on-premise or private registry cluster, the network transfer speed is usually faster than remote cloud.
%Thus, slice compression is less important for medium to small size slices, 
%especially for the slices that have a high decompression latency, 
%i.e., $D_{stt} < \theta_{stt}$ and $D_{dc} > \theta_{dc}$, where $D_{stt}$ and $D_{dc}$ denote slice transfer duration
%and decompression duration respectively; 
%$\theta_{stt}$ and $\theta_{dc}$ denote thresholds for them respectively.
%Consequently, \dedupname system~only archives these slices without compressing them and directly sends
%these archival files back to the clients to eliminate clients' decompression latency.



%}



\subsection{User behavior based layer preconstruction cache}
\label{sec:cache-design}

In the following, we present our \sysname~\preconstructcachename design.

%Each \texttt{pull layer} request has a precedent \texttt{pull manifest} request.
%Upon receiving a \texttt{pull manifest} request, 
%\sysname~sends the updated \sysname~manifest to client.
%After receiving a \sysname~manifest, the client parses the manifest
%and sends either a \texttt{pull layer} request if the layer hasn't been deduplicated,
%or a list of \texttt{pull slice} requests if a list of corresponding slices presents in the manifest.
%Those \texttt{pull slice} requests will then be
%forwarded to all the registry servers that store the requested
%layer's slices as shown in Figure~\ref{fig:sys-overview}. 

\subsubsection{User access pattern based preconstruction}

Docker client stores images as lists of layers and layers are shared among different repositories, which is similar to Docker registry.
When a client \texttt{pull}s an image from a repository, 
it will first \texttt{pull} the manifest of the image~\cite{docker}~\cite{dockerworkload} and 
parse the manifest to get the layer digests,
then lookup each layer digest against a \emph{local layer digest index}.
After that it only \emph{pulls} the layers that have \emph{not been stored locally}.
Theoretically, clients only pull layer once. 
However, some clients may delete some local images and \emph{repull} layers for these images.
%Moreover, kubernetes allows users always \emph{repull} layers no matter these layers locally available or not~\cite{docker}.  
Here, a \emph{repull layer} means users pull this layer multiple times,
and a \emph{non-repull layer} means users only pull this layer once.

\preconstructcachename~starts layer restoring when a \texttt{pull} manifest request is received, which is called layer preconstruction.
These preconstructed layers are temporally stored in the cache for later use.
In this case, layer restoring process and overhead can be avoided if requested layer is found in the cache.
Next, we analyze user access patterns to identify which layers in the repository will be pulled by users.

\paragraph{User access patterns}
Figure~\ref{xxx} shows the CDF of layer repull count.
We see that majority of users don't \emph{repull} layers frequently.
For \texttt{Syd}, only 4\% of layers are repulled by the same clients.
\texttt{Dev} has the highest repull layer ratio of 36\% while 83\% of the repull layers are only repulled twice.
Majority of repulled layers are repulled infrequently.
For example, only 3\% of layers from \texttt{Syd} are repulled more than twice.
Layer from \texttt{Prestage} and \texttt{Lon} have the highest repull frequency.
5\% of layers are pulled more than 6 times.
We also observe that few clients \emph{repull} layers continuously.
The highest layer repull count is 19,300 from \texttt{Lon}.
We think these clients probably deploy containers on a shared platform such as Cloud,
and run ephemeral jobs such as stateless microservices. 
Once the applications are finished, the container images are automatically deleted.

When different clients \texttt{pull} the same repository, 
they will fetch different amount of layers from the repository based on the availability of their local layer dataset.
Even the same clients \texttt{pull} the same repository at different times, 
they will fetch different amount of layers from the repository because their local layer dataset changes over time.
Therefore, a \texttt{pull manifest} requests doesn't usually result in pulling/repulling the layers in the repository. 
Here, we define \emph{repulling repository} as 
repulling the layer in the repository for the same client.
Figure~\ref{xxx} shows the CDF of the probability of repository repulling.
The probability of repository repulling is calculated 
as the number of \texttt{pull manifest} resulting in repository repulling divided by 
the total number of \texttt{pull manifest} requests issued by the same client for the same repository.
We see that majority of repositories aren't repulled.
The repull repository ratio ranges from 15\% for \texttt{Prestage} to 43\% for \texttt{Prestage}.
Majority of repull repositories have a low repulling probability.
Only 20\% of repositories from  \texttt{Prestage}, \texttt{Stage}, and 
\texttt{Syd} have a repulling probability higher than 0.5.
And only 20\% of repositories from the rest 4 workloads have a repulling probability higher than 0.33.
We also observe that few repositories' repulling probability are 1, meaning 
every time clients pull these repositories, they always repull the layers in these repositories. 
 
Figure~\ref{xxx} shows the client repulling probability.
Client repulling probability is calculated as the number of \emph{repull} layer requests divided by
the number of \texttt{pull} layer requests issued by the same client.
We see that majority of clients do repull layers but the probability is low.
60\% of clients from \texttt{Prestage}, \texttt{Dev}, \texttt{Lon}, and \texttt{Fra} have a repulling probability lower than 0.1.
55\% of clients from both \texttt{Dal} and \texttt{Stage} have a repulling probability lower than 0.1.
Less clients have repulling probability range between 0.1 to 0.7.
10\%-30\% of clients have a repulling probability ranged from 0.1-0.7 across 7 workloads.
We find few clients repull layers continuously.
2\%-12\% of clients have a repulling probability higher than 0.9 from workloads:
\texttt{Dal}, \texttt{Dev}, \texttt{Fra}, \texttt{Prestage},
\texttt{Stage}, \texttt{Syd}, and \texttt{Lon}.

\paragraph{User profiles}

\preconstructcachename~maintains a RLMap for recording repository-layer relationship.
If a user~\emph{u} \emph{pushes} a  layer~\emph{l} to a repository~\emph{r},
\preconstructcachename~will add an new entry (\emph{l}) in RLMap[\emph{r}]. 
Based on user access patterns,
\preconstructcachename~ also maintains URLMap for keeping track of user access patterns.
To identify a user, 
we extract \emph{user end host address} (\emph{r.client}) from each request (\emph{r}). 
If a user~\emph{u} \texttt{pull}s a layer~\emph{l} from a repository~\emph{r},
\preconstructcachename will update entry URLMap[\emph{u}].
%Each node records the following history information: (\emph{Get\_cnt}, \emph{Put\_cnt}, \emph{last\_access\_time}). a child node layer~\emph{L} to parent node~\emph{R}
Each URLMap entry maintains a user profile for each client as shown in Figure~\ref{xxx}.
User profile contains a list of repo profiles for each accessed repository,
and each repo profile contains a list of accessed layer profiles.
Note that layer profiles can be shared among different repo profiles for the same user.
User profile records the user repulling probability $u.rp$ for client $u$.
If $u.rp$ is greater than threshold $\theta_{crp}$, this user has a high probability of repulling layers.
Repo profile records repository repulling probability $u.r.rp$ for repository $r$.
If $u.r.rp$ is greater than threshold $\theta_{rrp}$, this repository is a popular repulling repository for this user.
Layer profile records layer repull count $u.l.R$ for layer $l$.
If $u.l.R$ is greater than threshold $\theta_{R}$, this layer is a popular repull layer for this user.

\paragraph{Preconstruction algorithm}
\begin{algorithm}
\scriptsize 
	\caption{Preconstruction}
	\label{alg:prefetch}
	\KwIn{\\
		%$\theta_{rp}$: Threshold for popular repulling clients. \\
		%{\tiny\texttt{/*when $L\_timer[layer] > L\_thresh$, layer is evicted or demoted to Flashcache */}}\\
		$\theta_{rpc}$: Threshold for repull layers to be preconstructed and cached. \\
		%$\theta_{R}$: Threshold for popular repulling layers. \\
		$RLmap$: Repository to layers map.\\
		$URLmap$: User to user profile map. \\
	}
	\While{true}{
		\emph{r} $\leftarrow$ \texttt{request received}\\
		\uIf{r = GET manifest}
		{
			newLayers $\gets$ \emph{RLmap[r.repo]} $-$ \emph{URLmap[r.client].layers} \\
			oldLayers $\gets$ \emph{RLmap[r.repo]} $\bigcap$ \emph{URLmap[r.client].layers} \\
			%\emph{layers} $\gets$ \emph{RLmap[r.repo]} $-$ \emph{URLmap[r.client]} \\
			%\emph{OnTimelayers}, \emph{NotOnTimelayers} $\gets$ \emph{OnTimeCalculation(layers)} \\
			%\emph{MEMcache} $\gets$ \emph{Prefetch(OnTimelayers)} \\
			%\emph{FLASHcache} $\gets$ \emph{Prefetch(NotOnTimelayers)} \\
			%\texttt{set} \emph{L\_timer[layer] for each layer in layers} \\
			\emph{cache} $\leftarrow$ \texttt{Preconstruct} newLayers \\
			\If{URLmap[r.client, r.repo, oldLayers].rp $>$ $\theta_{rpc}$} 
			{ 
				\emph{cache} $\leftarrow$ \texttt{Preconstruct} oldLayers \\
			}
		}
		\uElseIf{r = PUT layer }
		{
				%\texttt{update} \emph{URLmap[(r.client, r.repo, r.layer)]} \\
				\texttt{update} \emph{RLmap[r.repo, r.layer]} \\
				\emph{cache} $\leftarrow$ \texttt{cache} \emph{r.layer} \\
				%\texttt{set} \emph{L\_PUT\_timer[r.layer]} \\
				%{\tiny\texttt{/*when $L\_timer[layer] > L\_thresh$, layer is evicted/}}	 \\
			}
		\ElseIf{r = GET layer}
		{
				\uIf{r.layer in cache}
				{
					\emph{serve from cache} \\
					\emph{hit++} \\
				} 
			   %{ 
			   	    {\tiny\texttt{/* Otherwise, it's a repull layer miss   /}}\\
					\uElseIf{ files in r.layers are cached in file cache}
					{
						layer $\gets$ \texttt{Restoring from} \emph{file cache + Disks} 
					}
					\Else{
						layer $\gets$ \texttt{Restoring from} \emph{Disks} \\
					%\uIf{ $D_{rs}$ $>$ $\theta_{rs}$} 
					%{ 
						\emph{cache} $\gets$ layer
					%}		
				}
			%}
		
		}
	\texttt{update} \emph{URLmap[r.client, r.repo, r.layer]} \\
	\texttt{set} \emph{Timer[r.client, r.repo, layer]} \\
	}
\end{algorithm}

%
\begin{algorithm}
	\scriptsize 
	\caption{Eviction}
	\label{alg:eviction}
	\KwIn{\\
		$\theta_{C}$: Capacity threshold for layer cache to trigger replacement. \\
		%$T_{flash}$: Capacity threshold for FLASH cache to trigger eviction. \\
		%$\theta_{nrpe}$: Threshold for non-repull layer to be evicted.  \\
		$LFRUlayer$: LFRU of layers. \\
		%$URLmap$: User to user profile map. \\
		$Timer$: Timer for layers in layer cache.
	}
	
	\ForEach{layer in LFRUlayer} 
	{
		{\tiny $/*$\textit{Timeout eviction   /}} \\
		\If{Timeout == Timer[layer]}
			{
				\texttt{Evict} layer \\
			}
	}

	\While{freeSpace $<$ $\theta_{C}$}
	{
		{\tiny $/*$\textit{Replacement   /}} \\
		layer $\gets$ \emph{LFRUlayer.last\_item()}\\

		\texttt{Evict} layer \\
		\emph{freeSpace} $+=$ \texttt{sizeof} layer \\

  }
\end{algorithm}		
%		if layer.rrp < x:
%		remove the layers
%		
%		\If{layer exclusively belongs to last\_usr}{
%			%{\scriptsize $/*$\textit{If layer is not shared with other users, layer is deleted}}\\
%			%	last\_layer $\gets$ \emph{LayerLRU[last\_usr]}\\			
%			FLASHcache $\gets$ \emph{Demote(last\_layer)} \\
%			\emph{free\_MEM} $+=$ \emph{sizeof(last\_layer)} \\
%		}
%	}
%	
%}

%	\ForEach{layer in cache} 
%	{
%		{\tiny $/*$\textit{Non-repull layer eviction/}} \\
%		and URL[clients, repos, layer].rp <
%	}
%	for layer in cache:
%			if  and 
%				evict layer
%	for repo in timeout repo:
%	if repo.rrp < x:
%	remove the layer in the repo whoes repo all timeout and repo.rrp < x.
%	evictedLayer <-
%	
%	for user in timeout user:
%	if user.rrp < x:
%	remove the layers
	
	


%	\While{free\_FLASH $<$ $T_{flash}$}{
%	\emph{last\_repo} $\gets$ \emph{RepoLRU.last\_item()}\\
%	\For{last\_layer $\gets$ \emph{LayerLRU[last\_repo].last\_item()}}{
%		\If{layer exclusively belongs to last\_repo}{
%			%{\scriptsize $/*$\textit{If layer is not shared with other users, layer is deleted}}\\
%			%	last\_layer $\gets$ \emph{LayerLRU[last\_usr]}\\			
%			\emph{Discard(last\_layer)} \\
%			\emph{free\_FLASH} $+=$ \emph{sizeof(last\_layer)} \\
%		}
%	}
%}




%\begin{algorithm}
%    \tiny 
%	\caption{User access history based eviction}
%	\label{alg:prefetch}
%	%\SetAlgoLined
%	\KwIn{\\
%		$L_{thresh}$: Threshold for duration to keep a prefetched layer. \\
%		$RLMap$: Repository to layers map.\\
%		$URLmap$: User to layers map. \\
%	}
%	\While{true}{
%		\emph{r} $\leftarrow$ \texttt{request received}\\
%		\uIf{r = GET manifest}
%		{
%%			layerlst $\leftarrow$ URLmap[(r.client, r.repo)]
%			\emph{layers} $\gets$ \emph{RLMap[r.repo]} $-$ \emph{URLmap[r.client]} \\
%			\emph{OnTimelayers}, \emph{NotOnTimelayers} $\gets$ \emph{OnTimeCalculation(layers)} \\
%			\emph{MEMcache} $\gets$ \emph{Prefetch(OnTimelayers)} \\
%			\emph{FLASHcache} $\gets$ \emph{Prefetch(NotOnTimelayers)} \\
%			\texttt{set} \emph{L\_timer[layer] for each layer in layer} \\
%			{\tiny\texttt{/*when $L\_timer[layer] > L\_thresh$, layer is evicted/}}
%			}
%		\uElseIf{r = PUT layer }
%		{
%				\texttt{update} \emph{URLmap[(r.client, r.layer)]} \\
%				\texttt{update} \emph{RLMap[(r.repo, r.layer, put)]} \\
%				\emph{MEMcache} $\leftarrow$ \texttt{buffer} \emph{r.layer} \\
%				\texttt{set} \emph{L\_PUT\_timer[r.layer]} \\
%				{\tiny\texttt{/*when $L\_timer[layer] > L\_thresh$, layer is evicted/}}	 \\
%			}
%		\ElseIf{r = GET layer}
%		{
%				\eIf{r.layer in MEMcache or r.layer in FLASHcache}
%				{
%					\emph{serve from MEMcache or FLASHcache} \\
%					\texttt{update} \emph{URLmap[(r.client, r.layer)]} \\
%					\texttt{Reset} \emph{L\_timer[r.layer]}\\
%					\emph{hit++} 
%				}
%			   {
%					\emph{serve from backend storage system} \\
%					\texttt{update} \emph{URLmap[(r.client, r.layer, repulled)]} \\
%					\emph{RepulledLayers} $\gets$ \emph{RLMap[r.repo]} \\
%					\emph{FLASHcache} $\gets$ \emph{Prefetch(RepulledLayers)} \\
%					\texttt{set} \emph{L\_timer[layer] for each layer in layers} \\
%					{\tiny\texttt{/*when $L\_timer[layer] > L\_thresh$, layer is evicted/}}
%			}
%		}
%	}
%\end{algorithm}

%U_thresh
%
%while true do
%	r <- request received
%		if r = Get manifest then
%			layerlst <- Urlmap[(r.client, r.repo)]
%			layers <- choose_popular(layerlst) 
%			PrefetchedLayers <- Prefetch(layers) 
%			set L_timer[layer] for each layer in layers
%			/when L_timer[layer] > L_thresh, layer is evicted/
%		if r = Put layer then
%			update Urlmap[(r.client, r.repo, r.layer)]
%			Layerbuffer <- buffer r.layer
%			set L_timer[r.layer] for r.layer
%			/when L_timer[layer] > L_thresh, layer is evicted/		
%		if r = Get layer then
%			if r.layer in PrefetchedLayers then
%				serve from PrefetchedLayers[r.layer]
%				update Urlmap[(r.client, r.repo, r.layer)]
%				prefetch_hit++
%			else
%				server from backend storage system
%				update Urlmap[(r.client, r.repo, r.layer)]
%				layerlst <- Urlmap[(r.client, r.repo)]
%				layers <- choose_popular(layerlst) 
%				PrefetchedLayers <- Prefetch(layers) and 
%				set L_timer[layer] for each layer in layers
%				/when L_timer[layer] > L_thresh, layer is evicted/


As shown in algorithm~\ref{alg:prefetch}, 

%keep track of all layers that have been downloaded by \emph{r.client}. 
%Note that for a layer node, \emph{Put\_cnt} $=$ 1 or $=$ 0.
%When a GET or PUT layer request is received, \sysname~will update the URLMap of the associating user, repository, and layer. 
When either a GET manifest request is received or 
%a GET layer request is miss,
%a requested layer is not cached or prefetched  (
a miss on a GET layer request happens,
\sysname will lookup RLMap and get the requested repository's containing layers,
and compare against the layers that are already \emph{pulled} by the user by looking up URLMap,
%select a certain number of \emph{popular} layers from the client's 
%previously accessed layers by lookup 
%URLMap
then prefetch the layers that have not been \emph{pulled}. 
We set a timer for each cached layer and evict it when its timer is $>U_{thresh}$.
To incorporate the algorithm with \sysname,
we prefetch \texttt{slices} in parallel from backend servers,  
%dedup storage system,
buffer them in the layer buffer first, then evict them into the file cache after they \emph{cool down}.

\subsubsection{User access pattern  based cache replacement}

\begin{algorithm}
	\scriptsize 
	\caption{Eviction}
	\label{alg:eviction}
	\KwIn{\\
		$\theta_{C}$: Capacity threshold for layer cache to trigger replacement. \\
		%$T_{flash}$: Capacity threshold for FLASH cache to trigger eviction. \\
		%$\theta_{nrpe}$: Threshold for non-repull layer to be evicted.  \\
		$LFRUlayer$: LFRU of layers. \\
		%$URLmap$: User to user profile map. \\
		$Timer$: Timer for layers in layer cache.
	}
	
	\ForEach{layer in LFRUlayer} 
	{
		{\tiny $/*$\textit{Timeout eviction   /}} \\
		\If{Timeout == Timer[layer]}
			{
				\texttt{Evict} layer \\
			}
	}

	\While{freeSpace $<$ $\theta_{C}$}
	{
		{\tiny $/*$\textit{Replacement   /}} \\
		layer $\gets$ \emph{LFRUlayer.last\_item()}\\

		\texttt{Evict} layer \\
		\emph{freeSpace} $+=$ \texttt{sizeof} layer \\

  }
\end{algorithm}		
%		if layer.rrp < x:
%		remove the layers
%		
%		\If{layer exclusively belongs to last\_usr}{
%			%{\scriptsize $/*$\textit{If layer is not shared with other users, layer is deleted}}\\
%			%	last\_layer $\gets$ \emph{LayerLRU[last\_usr]}\\			
%			FLASHcache $\gets$ \emph{Demote(last\_layer)} \\
%			\emph{free\_MEM} $+=$ \emph{sizeof(last\_layer)} \\
%		}
%	}
%	
%}

%	\ForEach{layer in cache} 
%	{
%		{\tiny $/*$\textit{Non-repull layer eviction/}} \\
%		and URL[clients, repos, layer].rp <
%	}
%	for layer in cache:
%			if  and 
%				evict layer
%	for repo in timeout repo:
%	if repo.rrp < x:
%	remove the layer in the repo whoes repo all timeout and repo.rrp < x.
%	evictedLayer <-
%	
%	for user in timeout user:
%	if user.rrp < x:
%	remove the layers
	
	


%	\While{free\_FLASH $<$ $T_{flash}$}{
%	\emph{last\_repo} $\gets$ \emph{RepoLRU.last\_item()}\\
%	\For{last\_layer $\gets$ \emph{LayerLRU[last\_repo].last\_item()}}{
%		\If{layer exclusively belongs to last\_repo}{
%			%{\scriptsize $/*$\textit{If layer is not shared with other users, layer is deleted}}\\
%			%	last\_layer $\gets$ \emph{LayerLRU[last\_usr]}\\			
%			\emph{Discard(last\_layer)} \\
%			\emph{free\_FLASH} $+=$ \emph{sizeof(last\_layer)} \\
%		}
%	}
%}




%\begin{algorithm}
%    \tiny 
%	\caption{User access history based eviction}
%	\label{alg:prefetch}
%	%\SetAlgoLined
%	\KwIn{\\
%		$L_{thresh}$: Threshold for duration to keep a prefetched layer. \\
%		$RLMap$: Repository to layers map.\\
%		$URLmap$: User to layers map. \\
%	}
%	\While{true}{
%		\emph{r} $\leftarrow$ \texttt{request received}\\
%		\uIf{r = GET manifest}
%		{
%%			layerlst $\leftarrow$ URLmap[(r.client, r.repo)]
%			\emph{layers} $\gets$ \emph{RLMap[r.repo]} $-$ \emph{URLmap[r.client]} \\
%			\emph{OnTimelayers}, \emph{NotOnTimelayers} $\gets$ \emph{OnTimeCalculation(layers)} \\
%			\emph{MEMcache} $\gets$ \emph{Prefetch(OnTimelayers)} \\
%			\emph{FLASHcache} $\gets$ \emph{Prefetch(NotOnTimelayers)} \\
%			\texttt{set} \emph{L\_timer[layer] for each layer in layer} \\
%			{\tiny\texttt{/*when $L\_timer[layer] > L\_thresh$, layer is evicted/}}
%			}
%		\uElseIf{r = PUT layer }
%		{
%				\texttt{update} \emph{URLmap[(r.client, r.layer)]} \\
%				\texttt{update} \emph{RLMap[(r.repo, r.layer, put)]} \\
%				\emph{MEMcache} $\leftarrow$ \texttt{buffer} \emph{r.layer} \\
%				\texttt{set} \emph{L\_PUT\_timer[r.layer]} \\
%				{\tiny\texttt{/*when $L\_timer[layer] > L\_thresh$, layer is evicted/}}	 \\
%			}
%		\ElseIf{r = GET layer}
%		{
%				\eIf{r.layer in MEMcache or r.layer in FLASHcache}
%				{
%					\emph{serve from MEMcache or FLASHcache} \\
%					\texttt{update} \emph{URLmap[(r.client, r.layer)]} \\
%					\texttt{Reset} \emph{L\_timer[r.layer]}\\
%					\emph{hit++} 
%				}
%			   {
%					\emph{serve from backend storage system} \\
%					\texttt{update} \emph{URLmap[(r.client, r.layer, repulled)]} \\
%					\emph{RepulledLayers} $\gets$ \emph{RLMap[r.repo]} \\
%					\emph{FLASHcache} $\gets$ \emph{Prefetch(RepulledLayers)} \\
%					\texttt{set} \emph{L\_timer[layer] for each layer in layers} \\
%					{\tiny\texttt{/*when $L\_timer[layer] > L\_thresh$, layer is evicted/}}
%			}
%		}
%	}
%\end{algorithm}
\paragraph{Temporal trend}

Figure~\ref{xxx} shows the CDF of layer popularity.
We observe a heavy layer access skewness for \texttt{Fra}, \texttt{Syd}, \texttt{Dal}, \texttt{Stage}, 
and \texttt{Lon}.
We see that 80\%, 70\%, and 60\% of the \texttt{pull} layer requests access only 10\% of layers, 
for \texttt{Fra}, \texttt{Syd}, \texttt{Dal}, \texttt{Stage}, 
and \texttt{Lon}.
Figure~\ref{xxx} shows the CDF of repository popularity.
Compare to layer popularity, 
repository access skewness is heavier acorss 7 workloads.
~90\% of \texttt{pull} layer requests access only 10\% of repositories for 
\texttt{Dev}, \texttt{Fra}, \texttt{Prestage}, \texttt{Syd}, and \texttt{Stage}.
~75\% of \texttt{pull} layer requests access only 10\% of repositories for
both \texttt{Dal} and \texttt{Lon}.
Figure~\ref{xxx} shows the CDF of client popularity.
\texttt{Dal}, \texttt{Dev}, \texttt{Fra}, \texttt{Lon}, \texttt{Prestage}, and \texttt{Stage}
shows a heavey client access skewness.
10\% of clients send 95\% \texttt{pull} layer requests for \texttt{Lon}.
\texttt{Syd} shows a slight client skewness. 70\% of requests are sent by 36\% of clients.

Figure~\ref{xxx} shows the CDF of layer reuse time. 
Layer reuse time means the duration between two consecutive \texttt{pull} requests to the same layer.
We see that layer reuse time distribution varies among different workloads.
For \texttt{Fra}, \texttt{Syd}, and \texttt{Stage},
half of the layers' reuse time is shorter than 6 minutes.
While half of layers from \texttt{Dal} and \texttt{Lon} have a reuse time higher than 1 hour.
Half of layers from both \texttt{Prestage} and \texttt{Dev} are not accessed for over 100 hours.

Figure~\ref{xxx} shows the CDF of repository reuse time.
Repository reuse time means the duration between two consecutive \texttt{pull manifest} requests to the same repository.
We find that repository reuse time is much shorter than layer reuse time.
80\% of repositories are re-accessed within 2-12 minutes for the 7 workloads.

Figure~\ref{xxx} shows the CDF of client access intervals.
Client access intervals are much shorter than repository reuse time.
80\% of client are active for at least 1 - 3 minutes for the 7 workloads. 





