%\section{Conclusion}
%\label{sec:conclusion}
%
%Data deduplication has proven itself as a highly effective technique for
%eliminating data redundancy.
%%
%In spite of being successfully applied to numerous real datasets, deduplication
%bypassed the promising area of Docker images.
%%
%In this paper, we propose to fix this striking omission.
%%
%We analyzed over 1.7 million real-world Docker image layers and identified that
%file-level deduplication can eliminate 96.8\% of the files resulting in
%a capacity-wise deduplication ratio of 6.9$\times$.
%%
%We proceeded with a simulation-based evaluation of the impact of deduplication
%on the Docker registry performance.
%%
%We found that restoring large layers from registry can slow down \texttt{pull}
%performance due to compression overhead. To speed up \sysname, we suggested several
%optimizations.
%%
%%\VT{After Section 4 is ready, we might add here one interesting finding from
%%simulation.}\NZ{addressed}
%%
%Our findings justify and lay way for integrating deduplication in the Docker registry.
%
%\paragraph{Future work}
%%
%In the future, we plan to investigate the effectiveness of sub-file deduplication for
%Docker images and to extend our analysis to more image tags rather than just the \texttt{latest} tag.
%%
%We also plan to proceed with a complete implementation of \sysname.
%
