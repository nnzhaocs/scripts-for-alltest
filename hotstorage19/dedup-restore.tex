\subsection{Restoring layers}
\label{sec:restore-desgin}
%\paragraph{Parallel slice restoring}
%\label{subsubsec:slice-restoring}



\begin{figure}[t]
	\centering
	\centering
	\includegraphics[width=\columnwidth]{graphs/sift-layer-construct-new.pdf}
	\caption{Parallel streaming layer construction.}
	\label{fig:construct}
\end{figure}

\begin{figure}[t]
	\centering
	\centering
	\includegraphics[width=\columnwidth]{graphs/sift-cache-bigger.pdf}
	\caption{Tiered storage architecture.}
	\label{fig:tieredstore}
\end{figure} 

%When a~\texttt{pull} layer request is received and its associated 
%P-servers fail, D-servers will rebuild a layer for the request.
%\sysname will first search layer preconstruct cache on D-servers.
%If not found,
%\sysname will rebuild the layer according to the layer recipe. 
%the \dedupname~system 
%first prepares a directory structure for the slice, based on the slice recipe.
%Then, it copies the files into the directory tree.
%Next, it compresses the slice's directory tree into a slice tarball,
%and directly sends it back to the client.

%\LR{Are we only considering the case when a P-server fails or
%are we also considering, if a P-server gets overloaded?}

To restore a layer, each D-server in \sysname uses  a slice and a
layer constructor. Considering Figure~\ref{fig:replication-partition} as
an illustration, the restoring process works as follows:

First, the layer constructor fetches the layer recipe from the metadata database.
%According to $L1$'s layer recipe, 
%(Figure~\ref{fig:replication-partition})
%Figure~\ref{fig:construct}
According to $L1$'s layer recipe,  the restoring workers are D-servers $A$, $B$, and $C$. The node with the largest slice is picked as the restoring master ($A$ in the example).
Since $A$ is the restoring master
%\LR{Explain how the master is picked},
%\NZ{We select the server which stores the biggest slice as master to reduce restoring overhead.}
it sends \texttt{GET slice} requests for the primary slices to $B$ and $C$.
%Note that the master first tries to rebuild the layer from primary slices.
If a primary slice is missing, the master locates its corresponding backup
slice and send a \texttt{GET slice} request to the corresponding D-server.

%\paragraph{Parallel slice construction}

%Upon a \texttt{pull} layer request fails on P-servers,
%\sysname initiates a layer restoring process on D-servers for it.
%which involves two modules: slice constructor and layer constructor. 
After a \texttt{GET slice} request has been received, 
$B$'s and $C$'s slice constructors start rebuilding their primary slices and send them to $A$ as shown in Figure~\ref{fig:construct}.
Meanwhile $A$ instructs its local slice constructor to restore its primary slice for $L1$.
%
To construct a layer slice, a slice constructor first gets the associate slice recipe
from the metadata database. The recipe is keyed by a combination of layer Id, host address and requested
backup level, i.e., $L1::A::P$.
%\LR{More details here. Why is that they key? Why is the backup level relevant?}.
%\NZ{The slices at the same backup level can build a complete layer.}
%
Based on the recipe, the slice constructor creates a slice tar file by concatenating each file header and the corresponding
file contents; it then compresses the slice and passes it to the master.
%rebuilds each file entry in the slice tar archive according to the slice recipe.
%
The master concatenates all the compressed slices into a single
layer compressed tarball and sends it back to the client.
% as shown in Figure~\ref{fig:construct}.
%As a result no intermediate file will be created or stored on disk. % subil: whoa!


%\paragraph{Optimizations}
The layer restoration performance is critical to keep \texttt{pull} latencies low. Hence,
\sysname uses several additional optimizations to further speed up the process:
%Besides parallelizing layer reconstruction across D-servers, \sysname also
it parallelizes slice reconstruction on a single node and
avoids generating intermediate files on disk to reduce disk I/O.
%
%To avoid generating any intermediate files stored on disk to reduce disk I/Os
%and improve restoring performance, the slice restoring process is a stateless streaming
%slice construction process.
%
%All the involved data processing operations are preceded in memory as streaming. 

%Specifically, a slice constructor first reads each entry in the slice recipe
%and gets each header and its corresponding content pointer.
%
%Then, the slice constructor reads the corresponding physical files addressed by the content
%pointers in parallel from the D-server's file store
%
%\LR{The following is unclear. What does ``asynchronously'' mean in this case? Is every
%file written to its own buffer in parallel and then combined in a single archive buffer?}
%\NZ{It's synchronously.}
%and writes each file and it's corresponding header to a slice archive buffer in memory.
%Before writing file content to the archive buffer, slice constructor first writes its
%associated file header into the archive buffer according to the slice recipe.
%
%After archiving all the files along with their headers in the slice archive buffer,
%the archive buffer will be divided into several chunks, compressed in parallel,
%concatenated again into a single compressed slice, and sent to the master.
%
%Then, it writes the header in the slice archive.
%After that it put the associated file content in the slice archive
%by reading the corresponding physical file addressed by the content pointer.
%fetches files pointed by content
%and builds a slice archive.
%
%Next, the slice constructor sends the slice stream to the layer constructor on restoring master
%via network.
%
%first, slice constructor loads file in parallel from file store based on the slice recipe.
% along with their corresponding headers saved in slice recipe.
%
%\LR{The following paragraph is a little bit out of context. Why are we suddenly talking about
%stateless streaming layer construction? Need to tie it in better with the rest of the text and
%motivate our design decisions better.}
%
%\paragraph{Streaming layer construction}
%\LR{What exactly do you mean by ``Through network transfer''? Are the different slices kept inmemory at the master for concatenation after they have been received?}
%\NZ{Yes. But before that, these slices need to be transferred through network to master.}
%After receiving all slices, layer constructor on \emph{A} will concatenate all slices into a compressed layer
%. 
%Overall, the layer restoring process is a stateless streaming layer construction process.
%
%All the involve data processing operations are preceded in memory as streaming 
%without creating any intermediate files stored on disk to reduce disk I/Os and improve
%performance.

%Furthermore, \sysname also provisions a small in-memory file cache on each D-server to cache hot files.
%This reduces disk I/O during restoration even further. The file cache uses Adaptive
%Replacement (ARC)~\cite{xxx} \todo{fill in more details on how the cache works, how big it is, etc.}
%%
%\LR{It would be good to have a graph in the evaluation, showing how each of those optimizations
%reduces reconstruction time.}
%\NZ{The file cache hit ratio is very low, around 30\%
%because less files are shared among the 
%concurrent incoming requested layers.
%We can remove file cache from design.}

%headers according to $Dests$,
%after that it writes file contents into the archive
 
%Slice restoring process has four suboperations: 
%slice recipe lookup,
%slice file copying,
%slice compression, and
%slice network transfer. 
%To measure the overhead for each suboperation, 
%we implemented layer deduplication and parallel slice
%restoring on a 4-node registry cluster. 
%We first warmup the cluster by pushing 200 layers to the cluster
%and initiating layer deduplication process.
%The layers were randomly selected from our layer dataset detailed in xxx limited to 50MB.
%After finishing layer deduplication,
%we sent 400 \texttt{pull slice} requests to the cluster with 10 \texttt{pull slice} requests issued at a time.
%Figure~\ref{fig:slice-restoring-breakdown} shows the CDFs of the latencies for each suboperation.
%We see that across the four suboperations,
%the duration for slice compress is the shortest.
%Slice compression only took less than 0.001 s because a slice is a smaller unit. 
%The next shortest suboperation is network transfer since we pulled layer slice through Ethernet.
%90\% of slice recipe lookups took less than 0.1 s while 
%the highest slice recipe lookup duration almost reaches 0.8 s,
%which is caused by high concurrent lookup requests 
%%(note that we use redis to store metadata \NZ{use mongodb instead}).
%The most time consuming suboperation is slice file copying, which involves 
%copying all 
%the files that belong to the slice to their destination directory based on the slice recipe.
%Note that we implemented a thread pool on each registry server to read files in parallel
%and write data in RAMdisk to reduce disk IOs.
%40\%of slice file copying duration is greater than 1 s and 
%10\% of slice file copying duration is higher than 10 s.
%This is because bigger slices contains more files and requires more disk IOs.
%The overhead of slice copying can be largely mitigated for a large-scale registry cluster
%since the size of slice roughly equals to $S_{l}/N$, where $S_{l}$ denotes the layer size and $N$ is size of registry cluster.
%However, it could be a bottleneck for slice restoring on a small-scale registry cluster.
%%and slice file copying duration depends on slice size.
%
%%\begin{algorithm}
\scriptsize 
	\caption{File cache assisted slice restoring}
	\label{alg:prefetch}
	\KwIn{\\
		$\theta_{rsfc}$: Slice restoring latency threshold. \\
		$s$: Slice to be restored. \\
	}

	\SetKwFunction{Fsub}{Restore}
	\SetKwProg{Fn}{Function}{:}{}

	\Fn{\Fsub{s}}{
		%{\tiny\texttt{/* Otherwise, it's a repull layer miss   /}}\\
		\eIf{files in s are cached in file cache}
		{
			slice $\gets$ \texttt{RestoreSlice} \emph{s} \texttt{From} \emph{file cache + disk} 
		}
		{
			slice, $D_{rs}$ $\gets$ \texttt{RestoreSlice} \emph{s} \texttt{From} \emph{disk} \\
			\If{ $D_{rs} > \theta_{rsfc}$} 
			{ 
				\emph{file cache} $\gets$ \texttt{Cache} \texttt{Subsetof} \emph{s.files}
			}		
		}
	}

\end{algorithm}



%
%To reduce slice file copying overhead,
%\sysname~\filecachename~temporally cache a subset of unique files for bigger and popular slices that have a high slice restoring latency, ie., $D_{rs} > \theta_{rsfc}$, 
%where $D_{rs}$ is the slice restoring latency and $\theta_{rsfc}$ is the restoring latency threshold for 
%caching
%a subset of files from the slice to help improve its restoring performance as shown in Algorithm~\ref{alg:file-cache}.
%Upon a \texttt{pull slice} request for those slices, 
%\dedupname~ system fetches a subset of its containing files from \filecachename~and
%the remaining files from disk for slice restoring.
%
%To identify which slices have a high slice restoring latency,
%\dedupname~system monitors slice restoring performance and 
%maintains a restoring performance  profile for each slice that has been restored,
%% as shown in Figure~\ref{fig:xxx},
%which contains the latency breakdown of slice restoring
%% (,and a decompression latency updated by layer decompression process) 
%and its containing files' sizes.
%All the slice restoring performance profiles are also stored in distributed  databases,
% and addressed by slice digests. 
%To estimate the restoring latency for a slice $i$ that hasn't been restored, 
%\dedupname system~first lookups the slice restoring performance profiles by slice size,
% then selects a slice $x$ that is most similar in size to $i$,
% and estimates $i$'s restoring latency as: 
% $D_{rs}(i) \approx D_{rs}(x) + \Phi_{rs}(\Delta_{S})$,
% where $\Delta_{S}$ is the size different between two slices.
% $\Phi_{rs}(\Delta_{S})$ denotes a slice restoring latency function of slice size variation.
%  $\Phi_{rs}(\Delta_{S})$ is generated by using linear regression~\cite{xxx}.
% %$\varepsilon_{rs}$ is the standard error of restoring latency estimation for the layers similar in size.
%If the estimated slice restoring $D_{rs}(i) > \theta_{rs}$,
%then, \dedupname~lookups the slice restoring performance profiles by slice size,
%selects a slice $y$ that has a acceptable restoring latency and
%most similar in size to $i$.
%Next, \dedupname system~caches a subset of files $F$ for slice $i$, so that
%$D_{rs}(i) - \Phi_{rs}(\Sigma_{S}(F)) \approx D_{rs}(y)$,
%where $\Sigma_{S}(F)$ is the sum size of files in $F$.
%
%Note that \filecachename~size is limited so that \filecachename~only caches 
%subsets of files for big slices that belongs to popular layers.
%%that will be accessed later. 
%\cref{sec:cache-design} will describe how to determine popular layers based on user access patterns.
%Note that the slices for the same layer have similar sizes, restoring latencies, and popularity 
%because of unique file
%distribution. 
%Thus, once a layer is determined as popular layer, 
%\dedupname~will cache similar amount of files for its slices.
%Note that all the files in file cache are unique and can be shared for restoring different slices.
%
%For on-premise or private registry cluster, the network transfer speed is usually faster than remote cloud.
%Thus, slice compression is less important for medium to small size slices, 
%especially for the slices that have a high decompression latency, 
%i.e., $D_{stt} < \theta_{stt}$ and $D_{dc} > \theta_{dc}$, where $D_{stt}$ and $D_{dc}$ denote slice transfer duration
%and decompression duration respectively; 
%$\theta_{stt}$ and $\theta_{dc}$ denote thresholds for them respectively.
%Consequently, \dedupname system~only archives these slices without compressing them and directly sends
%these archival files back to the clients to eliminate clients' decompression latency.



%}
