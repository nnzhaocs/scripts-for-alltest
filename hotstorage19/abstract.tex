\begin{abstract}

\vspace{-6pt}
The fast growing number of container images and the associated
performance and storage capacity demands are key obstacles to
sustaining and scaling Docker registries.  
The inability to efficiently and effectively deduplicate 
layers that are stored in a compressed format is a major obstacle.
%   Moreover, the layers  are not amenable to direct application of
%   existing deduplication techniques efficiently.
% 
In this paper, we propose a new Docker registry architecture,
\sysname, that integrates caching and deduplication with Docker
registries and reduces the storage requirements while mitigating any
performance overhead. \sysname uses a highly-effective
user-access-history-based prefetch algorithm, and a two two-tier
heterogeneous cache comprising memory and flash storage. \sysname achieves 
a $96$\% hit ratio and saves $56$\% more cache space.

%Based on our 

%sustaining and scaling container systems in the face of exponential growth is challenging. 
%%
%For example, Docker Hub~\cite{docker-hub}---a popular public container registry---stores more than~2 million public repositories. These repositories have grown at the rate of about $1$ million annually---requiring provisioning an additional 2.5~TB of storage per week on average---and the rate is expected to increase.
%%
%This puts intense pressure on Docker registry storage infrastructure, but the problem has so far remained largely unexplored.
%


\end{abstract}

%we predict  
%Overall,  
%to model the deduplication for estimating the deduplication effect on performance and storage savings, especially in terms of deduplication rate and deduplication overhead. We propose to use Markov decision process to find optimal solution that can maximize the storage saving and minimizing the cost in terms of performance degradation. Our solution will largely reduce the amount of redundant data in container storage systems and outperform the state-of-art deduplication techniques without any performance overhead.

%and evaluate
%the potential of file-level deduplication in the registry.
%
%Our analysis reveals that 
%
%We then present the design of \sysname---a Docker registry with file deduplication
%support---and conduct a simulation-based analysis of its performance implications.
