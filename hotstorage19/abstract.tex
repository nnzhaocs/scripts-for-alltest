\begin{abstract}

%\vspace{-5pt}

The rise of containers has lead to a steep proliferation of container images.
%
The associated storage performance and capacity requirements place high
pressure on the registries, which store and serve images, and are key obstacles
to efficiently scale these registries.
%
%The proliferation of container images and the associated storage performance
%and capacity requirements are key obstacles to efficiently scale Docker
%registries.
%
Exploiting the high file redundancy in real-world images through deduplication is
a promising approach to drastically reduce the necessary storage capacity of image
registries
%
%Though real-world images contain a tremendous amount of duplicate data,
%
However, modern registries are currently not able to effectively eliminate the duplicates.
%due to the compressed format of the images.

In this paper, we propose \sysname, a new Docker registry architecture, which
natively integrates deduplication into the registry.
%
\sysname supports several configurable \emph{deduplication modes}, which provide
different levels of storage efficiency, durability, and performance, as
required by different use cases.
%
To further mitigate the negative impact of deduplication on the image download
and upload times, \sysname introduces a \emph{two-tier cache hierarchy} with a novel
prefetch algorithm based on user access histories.
%
Under real workloads, \sysname saves up to \gap of storage space while keeping
the latencies within \gap of the original registry.

\end{abstract}
